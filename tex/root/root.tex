\documentclass{article}
\usepackage{listings, xcolor}
\usepackage[hybrid]{markdown}
\usepackage[a4paper, margin=2.5cm]{geometry}
\usepackage[italian]{babel}
\usepackage[colorlinks=true,linkcolor=blue,urlcolor=blue]{hyperref}

\definecolor{codegreen}{rgb}{0,0.6,0}
\definecolor{codegray}{rgb}{0.5,0.5,0.5}
\definecolor{codepurple}{rgb}{0.58,0,0.82}
\definecolor{backcolour}{rgb}{0.95,0.95,0.92}

\lstdefinestyle{mystyle}{
    backgroundcolor=\color{backcolour},   
    commentstyle=\color{codegreen},
    keywordstyle=\color{magenta},
    numberstyle=\tiny\color{codegray},
    stringstyle=\color{codepurple},
    identifierstyle=\color{cyan},
    basicstyle=\ttfamily\footnotesize\color{black},
    breakatwhitespace=false,         
    breaklines=true,                 
    captionpos=b,                    
    keepspaces=true,                 
    numbers=left,                    
    numbersep=5pt,                  
    showspaces=false,                
    showstringspaces=false,
    showtabs=false,                  
    tabsize=2
}

\lstset{style=mystyle}

\title{Collezione di macro di ROOT}
\author{Luca Zoppetti}
\date{17/01/2024}

\begin{document}

\maketitle

\tableofcontents

\section*{Introduzione}
Questo documento contiene una collezione di macro di \href{https://root.cern}{ROOT} che ho scritto per esercitarmi in vista degli esami di Laboratorio del primo e del secondo anno della laurea in Fisica all'Università di Bologna. Tutto il codice è pubblicato anche in una \href{https://github.com/LuckeeDev/root}{\emph{repository}} pubblica su GitHub, per cui è possibile scaricarlo ed eseguirlo se si vuole capire come funziona ciascun esempio riportato di seguito. Tutto quello che pubblico sulla mia pagina è gratuito e ad uso libero per il proprio studio, tuttavia se apprezzassi il mio lavoro e ti andasse di offrirmi un caffè, questo è il mio profilo PayPal: \href{https://paypal.me/lucazoppetti}{paypal.me/lucazoppetti}. Buono studio!

\section{Esami del primo anno}

\markdownInput{code/exams/first_year/180608/README.md}
\lstinputlisting[language=C++]{code/exams/first_year/180608/macro.C}

\markdownInput{code/exams/first_year/190207/README.md}
\lstinputlisting[language=C++]{code/exams/first_year/190207/macro.C}

\markdownInput{code/exams/first_year/190610/README.md}
\lstinputlisting[language=C++]{code/exams/first_year/190610/macro.C}

\markdownInput{code/exams/first_year/210623/README.md}
\lstinputlisting[language=C++]{code/exams/first_year/210623/macro.C}

\markdownInput{code/exams/first_year/220712/README.md}
\lstinputlisting[language=C++]{code/exams/first_year/220712/macro.C}

\markdownInput{code/exams/first_year/220906/README.md}
\lstinputlisting[language=C++]{code/exams/first_year/220906/macro.C}

\markdownInput{code/exams/first_year/230615/README.md}
\lstinputlisting[language=C++]{code/exams/first_year/230615/macro.C}

\section{Esami del secondo anno}

\markdownInput{code/exams/second_year/180115/README.md}
\subsubsection{Risposta al quesito 1 (Efficienza)}
\lstinputlisting[language=C++]{code/exams/second_year/180115/macro1.C}
\subsubsection{Risposta al quesito 2 (Fit)}
\lstinputlisting[language=C++]{code/exams/second_year/180115/macro2.C}
\subsubsection{Risposta al quesito 3 (\texttt{FillRandom})}
\lstinputlisting[language=C++]{code/exams/second_year/180115/macro3.C}

\markdownInput{code/exams/second_year/180626/README.md}
\subsubsection{Risposta al quesito 1 (Efficienza)}
\lstinputlisting[language=C++]{code/exams/second_year/180626/macro1.C}
\subsubsection{Risposta al quesito 2 (Fit)}
\lstinputlisting[language=C++]{code/exams/second_year/180626/macro2.C}
\subsubsection{Risposta al quesito 3 (\texttt{FillRandom})}
\lstinputlisting[language=C++]{code/exams/second_year/180626/macro3.C}

\markdownInput{code/exams/second_year/240119/README.md}
\subsubsection{Risposta al quesito 1 (Istogrammi e \texttt{TList})}
\lstinputlisting[language=C++]{code/exams/second_year/240119/macro1.C}
\subsubsection{Risposta al quesito 2 (Categorie)}
\lstinputlisting[language=C++]{code/exams/second_year/240119/macro2.C}
\subsubsection{Risposta al quesito 3 (PDF definita a tratti)}
\lstinputlisting[language=C++]{code/exams/second_year/240119/macro3.C}

\section{Esercizi}

\markdownInput{code/misc/benchmark/README.md}
\lstinputlisting[language=C++]{code/misc/benchmark/macro.C}

\markdownInput{code/misc/categories/README.md}
\lstinputlisting[language=C++]{code/misc/categories/macro.C}

\markdownInput{code/misc/efficiency/README.md}
\lstinputlisting[language=C++]{code/misc/efficiency/macro.C}

\markdownInput{code/misc/fill_random/README.md}
\lstinputlisting[language=C++]{code/misc/fill_random/macro.C}

\section{Macro realizzate a lezione}

\subsection{Riempimento e fit di istogrammi}
\lstinputlisting[language=C++]{code/lectures/fill_fit_histo.C}

\subsection{\texttt{TList}}
\lstinputlisting[language=C++]{code/lectures/list.C}

\subsection{Benchmark}
\lstinputlisting[language=C++]{code/lectures/benchmark.C}

\subsection{Verifica di correlazione}
\lstinputlisting[language=C++]{code/lectures/correlation.C}

\subsection{Efficienza}
\lstinputlisting[language=C++]{code/lectures/efficiency.C}

\subsection{Proporzioni}
\lstinputlisting[language=C++]{code/lectures/proportions.C}

\subsection{Risoluzione}
\lstinputlisting[language=C++]{code/lectures/resolution.C}

\subsection{Scrivere su \texttt{TTree}}
\lstinputlisting[language=C++]{code/lectures/make_tree.C}

\subsection{Leggere da \texttt{TTree}}
\lstinputlisting[language=C++]{code/lectures/read_tree.C}

\end{document}