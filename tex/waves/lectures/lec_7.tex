\section{Studio di un'onda progressiva}

\lecture{7}{19 marzo 2024}

Studiamo ora il caso generico di una forza applicata alla sorgente dell'onda su una corda. (Finora abbiamo studiato un moto noto sulla sorgente, non una forza generica).

\begin{figure}[H]
	\centering
	% TODO: edit to remove little arrow at the bottom
	\includegraphics[width=0.8\textwidth]{screenshots/2024-03-19-11-12-22.png}
\end{figure}

Impostando l'equazione della dinamica per il moto dell'anello posto in \(x=0\) otteniamo:
\[
	\vec{F_e} + \vec{T} +\vec{R_v}  = \mathrm{d}m \vec{a} \rightsquigarrow \begin{dcases}
		T\cos \theta + R_v = \mathrm{d} m a_x\\
		F_y + T\sin \theta =\mathrm{d} m a_y  \\
	\end{dcases}
\]

Facendo tendere \(\mathrm{d} m \to 0\), i membri di destra tendono a zero.
\[
	\begin{dcases}
		T\cos \theta + R_v = 0\\
		F_y + T\sin \theta = 0  \\
	\end{dcases}
	\rightsquigarrow F_y = -T\sin \theta \thickapprox -T \tan \theta = -T \frac{\partial \xi }{\partial x}
\]
Di conseguenza \(F_y = -T \left. \frac{\partial \xi }{\partial x}\right\vert_x=0 \). In questo sistema si producono esclusivamente onde progressive (\(f(x-vt)\) ), se ci fosse un + al posto del - tratteremmo invece onde regressive.
\begin{figure}[H]
	\centering
	\includegraphics[width=0.8\textwidth]{screenshots/2024-03-19-11-22-13.png}
\end{figure}
Per quanto scritto sopra,
\[
	\frac{\partial \xi }{\partial x} = - \frac{1}{v} \frac{\partial \xi }{\partial t} \rightsquigarrow F_y(t) = \frac{T}{v}\frac{\partial \xi (0,t)}{\partial t} 
\]
La derivata parziale di \(\xi \) calcolata nel punto 0 rappresenta la velocità verticale del punto sorgente, che verrà indicata con \(v_c(t)\). Inoltre possiamo definire l'impedenza meccanica della corda.
\begin{definition}
	[Impedenza meccanica]
	Si definisce impedenza meccanica:
	\[
		Z=\frac{T}{v}= T \sqrt{\frac{\mu }{T}} = \sqrt{\mu T}  
	\]
\end{definition} 
L'effetto della forzante è quindi quello di mettere in moto il primo punto della corda con una velocità proporzionale alla forza (di solito non è così!):
\[
	F_y(t) = Zv_c(t)
\]
Usare \(v\) e \(Z\) al posto di T e \(\mu \) ci permetterà di trovare risultati molto più generali.

\paragraph{C'è una forza viscosa?}

Sembra che nel sistema ci sia una forza viscosa che porta la forza a essere proporzionale alla velocità. Si possono interpretare le formule come se la corda fosse in grado di applicare una forza dipendente dalla velocità: \(F_v(t) = -Z v_c(t)\). L'energia non viene dissipata, è assorbita e utilizzata per mettere in moto punti sempre più lontani della corda.

\begin{note}
	\(v_c(t)\) è sempre in fase con la forzante, quindi siamo sistematicamente in una situazione di risonanza.
\end{note}

\subsection{Energia e potenza}

Il lavoro compiuto dalla forzante è \(\delta L = \vec{F} \cdot \mathrm{d} \vec{s} = F_y \mathrm{d} y = F_y \frac{\mathrm{d}\xi }{\mathrm{d}t} \mathrm{d} t = F_y v_c \mathrm{d} t \). Tuttavia, per quanto ricavato prima so che

\begin{figure}[H]
	\centering
	\includegraphics[width=0.8\textwidth]{screenshots/2024-03-19-11-33-56.png}
\end{figure}

E la potenza risulta quindi:
\[
	\mathcal{P} = \frac{\delta L}{\mathrm{d} t} = -T \left. \frac{\partial \xi }{\partial x} \right\vert_{x=0} \frac{\partial \xi (0,t)}{\partial t} 
\]
Tuttavia la relazione è più generale, che vale per ogni punto della corda:
\[
	\mathcal{P} = -T \frac{\partial \xi }{\partial x} \frac{\partial \xi }{\partial t}
\]
Per le onde progressive sappiamo già che 
\begin{figure}[H]
	\centering
	\includegraphics[width=0.8\textwidth]{screenshots/2024-03-19-11-37-15.png}
	\caption{Relazione fra la derivata temporale e spaziale per le onde progressive.}
	\label{fig:dt-dx-onda-prog}
\end{figure}
Di conseguenza la potenza può essere riscritta:
\begin{gather*}
	\mathcal{P} (x,t) = \frac{T}{v}\left(\frac{\partial \xi (x,t)}{\partial t}\right)^{2}  = Tv\left(\frac{\partial \xi (x,t)}{\partial x} \right)^{2} \\
	\mathcal{P} (x,t) = Z \left(\frac{\partial \xi (x,t)}{\partial t}\right)^{2}
\end{gather*}
Si vede subito che la potenza è sempre maggiore o uguale a zero. Quanto ricavato è valido in generale per tutte le onde meccaniche.

\paragraph{Energia cinetica}
Consideriamo un tratto di corda infinitesima in moto con velocità \(v=\frac{\partial \xi }{\partial t} \). 
\begin{figure}[H]
	\centering
	\includegraphics[width=0.8\textwidth]{screenshots/2024-03-19-11-44-58.png}
	\caption{L'immagine di riferimento per lo studio di energia cinetica e potenziale sulla corda.}
	\label{fig:energia-corda}
\end{figure}
Ottengo che 
\[
	\mathrm{d} K = \frac{1}{2} \mathrm{d} m v ^{2} = \frac{1}{2} \mu \left( \frac{\partial \xi }{\partial t}  \right) ^{2} \mathrm{d} x
\]
Da cui posso definire una densità lineare di energia cinetica:
\[
	u_K(x,t) = \frac{\mathrm{d}K}{\mathrm{d}x} = \frac{1}{2} \mu \left( \frac{\partial \xi }{\partial t}  \right) ^{2} 
\]

\paragraph{Energia potenziale}

Si fa riferimento sempre alla figura \ref{fig:energia-corda}. Il tratto infinitesimo di corda viene allungato facendo lavoro contro la tensione della corda: \(\delta L = -T (\mathrm{d}l - \mathrm{d} x ) = -T(\sqrt{\mathrm{d}x^{2} + \mathrm{d} y ^{2}   } - \mathrm{d} x )\). Applicando l'approssimazione delle piccole oscillazioni otteniamo che
\begin{gather*}
	\delta L = -T \mathrm{d} x \left( \sqrt{1+ \left( \frac{\partial \xi }{\partial x}  \right)^{2}  } -1  \right) \thickapprox -T \mathrm{d} x \left( 1+\frac{1}{2} \left( \frac{\partial \xi }{\partial x}  \right)^{2} + \cdots - 1  \right) \thickapprox \\
	\thickapprox -\frac{1}{2}T \left( \frac{\partial \xi }{\partial x}  \right)^{2} \mathrm{d} x\\
	u_P(x,t) = -\frac{\delta L}{\mathrm{d} x} = \frac{1}{2}T\left( \frac{\partial \xi }{\partial x}  \right) ^{2} 
\end{gather*}
Dove nell'ultima riga abbiamo definito la densità di energia potenziale.

\paragraph{Energia meccanica}
Possiamo definire la densità di energia meccanica:
\begin{definition}
	[Densità lineare di energia meccanica]
	\[
		u(x,t) = \frac{1}{2} \mu \left( \frac{\partial \xi }{\partial t}  \right) ^{2} + \frac{1}{2}T\left( \frac{\partial \xi }{\partial x}  \right) ^{2}
	\]
\end{definition}
Nel caso di onde progressive abbiamo le formule già viste che collegano la derivata in x alla derivata in t (Fig. \ref{fig:dt-dx-onda-prog}), quindi possiamo ricavare la seguente uguaglianza:
\begin{figure}[H]
	\centering
	\includegraphics[width=0.8\textwidth]{screenshots/2024-03-19-11-56-45.png}
\end{figure}
L'energia potenziale e l'energia cinetica sono uguali istante per istante in ogni punto! Quindi l'energia totale è il doppio dell'energia cinetica e il doppio dell'energia potenziale. La potenza e l'energia trasmessa sono inoltre proporzionali:
\begin{figure}[H]
	\centering
	\includegraphics[width=0.8\textwidth]{screenshots/2024-03-19-11-58-19.png}
\end{figure}