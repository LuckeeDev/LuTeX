\chapter{Ottica geometrica}
\lecture{23}{13 maggio 2024}
L'ottica geometrica si occupa di quali sono i sistemi e le regole con cui si formano le immagini di oggetti a valle di sistemi ottici. Esiste anche un'ottica "non geometrica" che si occupa solo di massimizzare l'intensità di un segnale luminoso a valle di sistemi ottici. Per la lezione di oggi serviranno alcune definizioni:
\begin{description}
	\item[Oggetto (O)]: è un corpo puntiforme o esteso che emette luce o che diffonde la luce di una sorgente. È detto "reale" quando i raggi partono da un punto (es.: oggetto fisico). L'oggetto potrebbe anche essere immagine di un sistema ottico e in tal caso si parla di oggetto virtuale.
	\item[Strumento ottico]: sistema che raccoglie i raggi uscenti da un oggetto e ne modifica la loro direzione.
	\item[Immagine (I)]: figura dell'oggetto realizzata attraverso i raggi deviati dagli strumenti ottici. È detta "reale" se i raggi deviati dallo strumento ottico convergono in un punto, mentre è detta "virtuale" se convergono i prolungamenti dei raggi (es.: specchio).
\end{description}
Gli strumenti ottici che studieremo sono "strumenti ottici stigmatici": i raggi generati da un punto di un oggetto convergono in un punto dell'immagine e i due punti sono detti "punti coniugati". Si ha uno strumento "astigmatico" quando non vi è corrispondenza biunivoca tra punto oggetto e punto immagine.

Si distinguono anche diversi casi di superfici:
\begin{description}
	\item[Superficie catottica] o specchio, quando il fenomeno che si realizza primariamente è quello della riflessione.
	\item[Superficie diottrica] o diottro, quando il fenomeno che si realizza primariamente è quello della rifrazione.
\end{description}
Per gli specchi la legge fisica è particolarmente semplice (\(\theta _i = \theta _r\)) e non dipende dalla pulsazione. Per i diottri è più complicata perché dipende anche dalla pulsazione/lunghezza d'onda. Questo fenomeno è detto "cromatismo" delle superfici diottriche. Si parla di "acromatismo" quando il sistema ottico è studiato in modo da minimizzare questo effetto.

\section{Specchi}
Consideriamo superfici approssimabili come parte di sfera. Per convenzione il raggio \(R\) è positivo se la curvatura è interna allo spazio immagine, mentre è negativo se la curvatura è interna allo spazio oggetto. Faremo riferimento alla seguente figura:
\begin{figure}[H]
	\centering
	\includegraphics[width=0.6\textwidth]{screenshots/2024-05-13-11-30-35.png}
\end{figure}
Facendo tendere \(O \to \infty \), lo specchio diventa uno specchio piano. I raggi che partono da P nello spazio oggetto e si riflettono sullo specchio piano, se prolungati, convergono nel punto Q appartenente allo spazio immagine. Quindi lo specchio è una superficie stigmatica. Introducendo un altro punto P' si nota che le distanze vengono mantenute dallo specchio piano:
\begin{figure}[H]
	\centering
	\includegraphics[width=0.6\textwidth]{screenshots/2024-05-13-11-38-55.png}
\end{figure}
Consideriamo ora una porzione di superficie sferica a specchio. Se è convessa, \(R>0\), se è concava \(R<0\). Per costruire le immagini si tracciano alcuni raggi particolari: paralleli all'asse ottico, passanti per il vertice, passanti per centri di curvatura, fuochi, ecc... Nelle superfici stigmatiche questi raggi che partono dal punto P convergono (se prolungati) in un punto Q coniugato a P.

\subsection{Specchio concavo}
Studiamo la seguente figura:
\begin{figure}[H]
	\centering
	\includegraphics[width=0.5\textwidth]{screenshots/2024-05-13-11-44-29.png}
\end{figure}
Osservando la figura si ottiene che \(\alpha + \theta = \beta \), \(\beta + \theta = \gamma \), da cui \(\alpha + \gamma = 2 \beta \). Approssimiamo per angoli piccoli: siano \(\tan \alpha = \quotient{HH^{\prime} }{PH^{\prime} } \implies \alpha \thickapprox \frac{h}{p} \), \(\tan \beta = \quotient{HH^{\prime} }{OH^{\prime} } \implies \beta \thickapprox \frac{h}{\vert R \vert }= -\frac{h}{R} \), \(\tan \gamma = \quotient{HH^{\prime} }{QH^{\prime} }\implies \gamma \thickapprox \frac{h}{\vert q \vert }=-\frac{h}{q} \) dove i cambi di segno sono dovuti al fatto che siamo nello spazio oggetto e che lo specchio è concavo. Otteniamo quindi
\begin{gather}
	\alpha + \gamma = 2\beta \implies \frac{h}{p} - \frac{h}{q} = -\frac{2h}{R}\\
	\frac{1}{p} - \frac{1}{q}= - \frac{2}{R} \coloneqq -\frac{1}{f}
\end{gather}
\begin{definition}
	[Fuoco dello specchio]
	Nella formula appena scritta si è definito il "fuoco" dello specchio:
	\begin{equation}
		f=\frac{R}{2}
	\end{equation}
	Si avrà \(f<0\) per uno specchio concavo e \(f>0 \) per uno specchio convesso.
\end{definition}
Il fuoco ha la proprietà di essere il punto in cui convergono i raggi generati da punti con \(p\to +\infty \). Questi punti dividono lo spazio in quattro parti: per P a sinistra di O, Q sarà tra O ed F; per P tra O e F, Q sarà a sinistra di O; per P tra F e V, Q sarà a destra di V; per P a destra di V, Q sarà tra F e V. Studiamo ora l'ingrandimento dovuto a uno specchio concavo di un oggetto a sinistra di O:
\begin{figure}[H]
	\centering
	\includegraphics[width=0.4\textwidth]{screenshots/2024-05-13-11-57-01.png}
\end{figure}
VPP' e VQQ' sono triangoli simili, quindi l'ingrandimento è \(m= \frac{QQ^{\prime} }{PP^{\prime} }=\frac{\vert q \vert }{\vert p \vert }= -\frac{q}{p}\). L'immagine è reale, ma è rovesciata. Se ora l'oggetto è tra F e V, la sua immagine sarà virtuale nello spazio immagine. Inoltre sarà ingrandita e diritta. Facendo il disegno si ricava facilmente che
\begin{equation}
	m = \frac{QQ^{\prime} }{PP^{\prime} }=\frac{q}{p}
\end{equation}
Questa costituisce da ora la definizione di ingrandimento. Si avrà \(m> 0\) per un'immagine dritta, \(m<0\) per un'immagine rovesciata.

% TODO: se necessario guarda le slide
\subsection{Specchio convesso}
Per lo specchio convesso si ottengono le stesse formule dello specchio concavo. Si fa il disegno e si ricava facilmente tutto quanto. L'unica differenza è che si avranno \(R<0\) e \(q>0\). Si verifica anche che per P nello spazio oggetto, Q è sempre tra V e il fuoco e che l'ingrandimento è sempre \(0 \leq m \leq 1\).