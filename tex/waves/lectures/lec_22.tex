\lecture{22}{9 maggio 2024}
\paragraph{Reticolo}
Un reticolo è un dispositivo caratterizzato da un numero N di fenditure grandi. Detto \(N^{\prime} \) il numero di fenditure per unità di lunghezza, si ha \(d= \quotient{1}{N^{\prime} } \). Per il reticolo si definisce la dispersione, che è la capacità di deviare la luce al variare della lunghezza d'onda: \(\mathcal{D} = \frac{\mathrm{d}\theta }{\mathrm{d} \lambda } = \frac{m}{d \cos (\theta_{max})} \) usando la formula dei massimi. Si definisce anche il potere risolutivo \(\mathcal{R} =\frac{\lambda }{\mathrm{d}\lambda } \) come la capacità di vedere come separati massimi a lunghezze d'onda molto vicine. Quindi non conta solo la distanza fra i picchi ma anche quanto è largo un picco. Considerato \(\mathrm{d} \theta \thickapprox \Delta \theta = \frac{\lambda }{N d}\) come distanza angolare fra i picchi, otteniamo
\begin{equation}
	\mathcal{R} = \frac{\lambda }{\mathrm{d} \lambda } = \frac{Nm}{\cos (\theta_{max})} \thickapprox Nm
\end{equation}

\begin{note}
	Nelle slide ci sono esempi di applicazioni dell'interferenza con film sottili (L30).
\end{note}