\chapter{Misura e integrazione secondo Lebesgue}

\section{Misura di Lebesgue}

\begin{definition}
	[Misura esterna]
	Siano $X$ un insieme e $\calP(X)=\{A:A\subseteq X\}$. $\mu^*:\calP(X)\to [0,+\infty]$ è una misura esterna se soddisfa le seguenti condizioni:
	\begin{enumerate}
		\item $\mu^*(\varnothing)=0$
		\item (Monotonia) Se $A\subseteq B$, allora $\mu^*(A) \leq \mu^*(B)$
		\item (Sub-additività numerabile) $\mu^*\left(\bigcup\limits_{i=1}^\infty A_i\right)\leq \sum\limits_{i=1}^\infty \mu^*(A_i)$
	\end{enumerate}
\end{definition}

\begin{definition}
	[$\sigma$-algebra]
	Sia $X$ un insieme. $\A \subseteq \calP(X)$ è una $\sigma$-algebra se:
	\begin{enumerate}
		\item $\varnothing, X \in \A$
		\item Se $A\subset \A$, allora $X \setminus A \in \A$
		\item Se $A_j \in \A \with j \in \N$, allora $\bigcup\limits_{i=1}^\infty A_i, \bigcap\limits_{i=1}^\infty A_i \in \A$
	\end{enumerate}
\end{definition}

\begin{prop}
	Se $C \subseteq \calP(X)$, allora $C$ può essere sempre completato a una $\sigma$-algebra ed esiste il completamento minimo $\A(C)$ che è la $\sigma$-algebra più piccola in $\calP(X)$ contenente $C$.
	$$
		\A(C)=\cap\{\A:\A \text{ è una } \sigma \text{-algebra in } \calP(X), C \subseteq \A\}
	$$
	\qed
\end{prop}

\begin{remark}\leavevmode
	\begin{enumerate}
		\item $\{\varnothing,X\}$ è la più piccola $\sigma$-algebra associata ad $X$
		\item $\calP(X)$ è la più grande $\sigma$-algebra associata ad $X$
	\end{enumerate}
\end{remark}

\begin{definition}
	[$\sigma$-algebra di Borel]
	Sia $(X,\tau)$ uno spazio topologico. Si definisce $\sigma$-algebra di Borel la più piccola $\sigma$-algebra in $X$ contenente $\tau$.
\end{definition}

\begin{definition}
	[Spazio misurabile]
	Si definisce spazio misurabile la coppia $(X,\A)$, dove $X$ è un insieme e $\A$ una $\sigma$-algebra contenuta in $\calP(X)$.
\end{definition}

\begin{definition}
	[Misura]
	Sia $(X,\A)$ uno spazio misurabile. $\mu:\A \to [0,+\infty]$ è una misura se soddisfa le seguenti richieste:
	\begin{enumerate}
		\item $\mu(\varnothing)=0$
		\item (Additività numerabile) Se $A_i \in \A \ \forall i \in \N$ sono insiemi a due a due disgiunti, allora $$\displaystyle\mu\left(\bigcup_{i=1}^\infty A_i \right) = \sum_{i=1}^\infty \mu(A_i)$$
	\end{enumerate}
	$\mu$ si dice finita se $\mu(X)<+\infty$. $\mu$ si dice $\sigma$-finita se esistono $A_i \with i \in \N$ insiemi misurabili di misura finita e $X=\bigcup\limits_{i=1}^\infty A_i$.
\end{definition}

\begin{remark}
	La proprietà di monotonia discende dal fatto che la misura è sempre definita a partire da una misura esterna.
\end{remark}

\begin{definition}
	[Spazio di misura]
	$(X,\A,\mu)$ è detto spazio di misura.
\end{definition}

\begin{definition}
	[Spazio di misura completo]
	$(X,\A,\mu)$ è uno spazio di misura completo se $\forall N \in \A \tc \mu(N)=0$ tutti i suoi sottoinsiemi sono misurabili, cioè $\forall M\subseteq N, \ M\in \A$.
\end{definition}

\begin{theorem}
	[Successioni monotone in uno spazio di misura]
	Sia $(X,\A,\mu)$ uno spazio di misura.
	\begin{enumerate}
		\item Se $A_i \in \A$ è una successione crescente, cioè $A_i \subseteq A_{i+1} \ \forall i \in \N$, allora $\mu\left(\bigcup\limits_{i=1}^\infty A_i\right) = \lim\limits_{i\to + \infty}\mu(A_i)$
		\item Se $A_i \in \A$ è una successione decrescente, cioè $A_i \supseteq A_{i+1} \ \forall i \in \N$ e $\exists j \in \N \tc \mu(A_j) < + \infty$, allora $\mu\left(\bigcap\limits_{i=1}^\infty A_i\right)=\lim\limits_{i\to +\infty}\mu(A_i)$
	\end{enumerate}
	\qed
\end{theorem}

\begin{definition}
	[Intervallo]
	Sia $(\R^n,\tau)$ lo spazio topologico euclideo. Si definisce intervallo $R=[a_1,b_1]\times\cdots[a_n,b_n] \subseteq \R^n$. La misura elementare di $R$ è $\mu(R)=\prod\limits_{j=1}^n(b_j-a_j)$. Si indica con $\calR$ l'insieme degli intervalli di $\R^n$: $\calR=\{A \in \calP(\R^n):A \text{ è un intervallo}\}$.
\end{definition}

\begin{definition}
	[Misura esterna di Lebesgue]
	Si definisce la misura esterna di Lebesgue $\mu^*:\calP(\R^n)\to [0,+\infty]$ tale che, se $E \in \calP(\R^n)$,
	$$
		\mu^*(E)=\inf \left\{\sum\limits_{j=1}^\infty \mu(R_j):R_j \in \calR \ \forall j \in \N \e E \subseteq \bigcup\limits_{j=1}^\infty R_j\right\}
	$$
\end{definition}

\begin{theorem}
	$\mu^*$ è una misura esterna, cioè
	\begin{enumerate}
		\item $\mu^*(\varnothing)=0$
		\item Se $A \subseteq B$, allora $\mu^*(A)\leq \mu^*(B)$
		\item Se $A_i \subseteq \R^n, i \in \N$, allora $\mu^*\left(\bigcup\limits_{i=1}^\infty A_i\right)\leq \sum\limits_{i=1}^\infty \mu^*(A_i)$
	\end{enumerate}
\end{theorem}

\begin{proof}
	[Dimostrazione del punto 3.]
	Sia $\varepsilon>0$. $\forall i \in \N$ è possibile trovare un ricoprimento di $A_i$ tale che
	$$A_i\subseteq \bigcup\limits_{j=1}^\infty R_{ij} \e \sum\limits_{j=1}^\infty \mu^*(R_{ij})\leq \mu^*(A_i)+\frac{\varepsilon}{2^i}$$
	Si noti che l'unico caso da dimostrare è quello in cui $\mu^*(A_i)<+\infty \ \forall i \e \sum\limits_{i=1}^\infty \mu^*(A_i)<+\infty$, perché altrimenti è ovvio che $\mu^*\left(\bigcup\limits_{i=1}^\infty A_i\right)\leq +\infty$. $\bigcup\limits_{i,j=1}^\infty R_{ij}$ è un ricoprimento di $E=\bigcup\limits_{i=1}^\infty A_i$.
	\begin{gather*}
		\mu^*(E)=\mu^*\left(\bigcup_{i=1}^\infty A_i\right)\leq \sum_{i,j=1}^{\infty}\mu^*(R_{ij})=\sum_{i=1}^\infty\left(\sum_{j=1}^{\infty}\mu^*(R_{ij})\right)\leq\\
		\leq\sum_{i=1}^{\infty}\left(\mu^*(A_i)+\frac{\varepsilon}{2^i}\right)=\sum_{i=1}^\infty \mu^*(A_i)+\varepsilon\\
		\then \mu^*\left(\bigcup_{i=1}^\infty A_i\right)\leq \sum_{i=1}^\infty \mu^*(A_i)
	\end{gather*}
\end{proof}

\begin{prop}
	La misura elementare di un intervallo $R \in \calR$ coincide con la misura esterna di Lebesgue dell'intervallo.
	\qed
\end{prop}

\subsection{Metodo di Carathéodory}

\begin{definition}
	[Insieme misurabile secondo Carathéodory]
	Siano $X$ un insieme e $\mu^*:\calP(X) \to [0,+\infty]$ una misura esterna su $X$. $A\subseteq X$ è misurabile secondo Carathéodory se $\forall E \subseteq X, \ \mu^*(E)=\mu^*(A\cap E) + \mu^*\left(A^C \cap E\right)$, dove $A^C=X\setminus A$.
\end{definition}

\begin{theorem}
	[Costruzione della misura]
	Siano $X$ un insieme e $\mu^*$ una misura esterna su $\calP(X)$. Allora l'insieme $\A \subseteq \calP(X)$ degli insiemi misurabili secondo Carathéodory è una $\sigma$-algebra e $\restr{\mu^*}{\A}$ è una misura.
	\qed
\end{theorem}

\begin{definition}
	[Misura di Lebesgue in $\R^n$]
	Sia $\mathcal{L}(\R^n)$ la $\sigma$-algebra degli insiemi misurabili secondo Carathéodory rispetto alla misura esterna di Lebesgue. La misura di Lebesgue è $\mu=\restr{\mu^*}{\mathcal{L}(\R^n)}$.
\end{definition}

\begin{theorem}
	[Completezza della misura di Lebesgue]
	La misura di Lebesgue è una misura completa, cioè:
	\begin{enumerate}
		\item $\forall N \subseteq \R^n$ con misura esterna nulla, $N \in \mathcal{L}(\R^n)$
		\item $\forall M \subseteq N \with \mu(N)=0, \ M \in \mathcal{L}(\R^n)$
	\end{enumerate}
\end{theorem}

\begin{proof}
	Si dimostra ciascun punto separatamente.
	\begin{enumerate}
		\item Siano $\mu^*(N)=0 \e E \subseteq \R^n$. Si noti che per la proprietà di subadditività è ovvio che $\mu^*(E)\leq \mu^*(E\cap N)+\mu^*\left(E\cap N^C\right)$, quindi resta da dimostrare solo che $\mu^*(E)\geq\mu^*\left(E\cap N^C\right)+\mu^*(E\cap N)$. $E \cap N \subseteq N\then \mu^*(E\cap N)\leq\mu^*(N)=0\then \mu^*(E\cap N)=0$. Inoltre $E\supseteq E\cap N^C$, quindi $\mu^*(E)\geq \mu^*\left(E\cap N^C\right)=\mu^*\left(E\cap N^C\right)+\mu^*(E\cap N)$. Quindi $N$ è misurabile secondo Carathéodory.
		\item $\forall M\subseteq N \ \mu^*(M)\leq\mu^*(N)=0$, quindi anche $M\in \mathcal{L}(\R^n)$.
	\end{enumerate}
\end{proof}

\begin{remark}
	Nella dimostrazione non viene usata la definizione di misura esterna di Lebesgue. Tutte le misure costruite con il metodo di Carathéodory sono complete.
\end{remark}

\begin{definition}
	[Insieme di misura nulla]
	$E \subseteq \R^n$ ha misura di Lebesgue nulla se e solo se $\forall \varepsilon > 0$ esiste un ricoprimento numerabile di $E$ con intervalli tale che $E \subseteq \bigcup\limits_{j=1}^\infty R_j, R_j \in \calR \e \sum\limits_{j=1}^\infty \mu(R_j)<\varepsilon$.
\end{definition}

\begin{theorem}
	[Relazione fra $\mathcal{B}(\R^n)$ e $\mathcal{L}(\R^n)$]
	Sia $\mathcal{B}(\R^n)$ la $\sigma$-algebra di Borel in $\R^n$. Allora
	\begin{enumerate}
		\item $\mathcal{B}(\R^n)$ è la più piccola $\sigma$-algebra che contenga $\mathcal{R}$, cioè gli intervalli di $\R^n$
		\item $\mathcal{B}(\R^n)\subset \mathcal{L}(\R^n)$
		\item $\mathcal{B}(\R^n)$ non è completo rispetto alla misura di Lebesgue
		\qed
	\end{enumerate}
\end{theorem}

\begin{theorem}
	[Misura con aperti e compatti]\leavevmode
	\begin{enumerate}
		\item Sia $A\subseteq\R^n$, allora $\mu^*(A)=\inf\{\mu(U):A\subseteq U \with U \text{ aperto}\}$. In particolare, se $A$ è misurabile allora $\mu^*(A)=\mu(A)$
		\item Se $A \in \mathcal{L}(\R^n)$, allora $\mu(A)=\sup\{\mu(K):K \text{ compatto}, K \subseteq A\}$
		\qed
	\end{enumerate}
\end{theorem}

\begin{corollary}
	$A \in \mathcal{L}(\R^n) \iff \forall \varepsilon > 0 \ \exists V \text{ chiuso},\ U \text{ aperto},\ V \subseteq A \subseteq U \tc \mu(U\setminus V) < \varepsilon$.\qed
\end{corollary}

\begin{definition}
	[$G_\delta \e F_\sigma$]
	Si definiscono i seguenti insiemi:
	\begin{itemize}
		\item $G_\delta = \left\{G \subseteq \R^n:G=\bigcap\limits_{j=1}^\infty U_j, U_j \text{ aperto } \forall j \in \N\right\} \subseteq \mathcal{B}(\R^n)$
		\item $F_\sigma = \left\{F \subseteq \R^n:F=\bigcup\limits_{j=1}^\infty C_j, C_j \text{ chiuso } \forall j \in \N\right\} \subseteq \mathcal{B}(\R^n)$
	\end{itemize}
\end{definition}

\begin{theorem}
	$\forall A \in \mathcal{L}(\R^n) \ \exists G \in G_\delta, F \in F_\sigma \tc F \subseteq A \subseteq G \e \mu(G\setminus A)=\mu(A\setminus F)=0$. In altre parole, ogni insieme misurabile secondo Lebesgue è approssimato a meno di insiemi di misura nulla da un insieme in $F_\sigma$ per difetto e da un insieme in $G_\delta$ per eccesso.
	\qed
\end{theorem}

\begin{corollary}
	Il completamento della $\sigma$-algebra di Borel è la $\sigma$-algebra degli insiemi misurabili secondo Lebesgue. Per ottenerla vengono aggiunti gli insiemi di misura nulla.\qed
\end{corollary}

\begin{theorem}
	[Invarianza della misura di Lebesgue per isometrie]\leavevmode
	\begin{enumerate}
		\item Siano $A\subseteq \R^n$ e $\vb{h} \in \R^n$. Sia $A_{\vb{h}}=\{\vb{x}+\vb{h}:\vb{x}\in A\}$ il traslato di $A$. Allora $A \in \mathcal{L}(\R^n)\iff A_{\vb{h}} \in \mathcal{L}(\R^n)$ e in tal caso $\mu(A)=\mu(A_{\vb{h}})$.
		\item Siano $Q\in \mathcal{O}(n)$ (il gruppo delle matrici ortogonali in $\R^n$) e $E\subseteq \R^n$. Allora $E \in \mathcal{L}(\R^n) \iff QE \in \mathcal{L}(\R^n)$ e in tal caso $\mu(E)=\mu(QE)$.
		\item Siano $T: \R^n \to \R^n$ un'applicazione lineare con $\det T \neq 0$ e $A \subseteq \R^n$. Allora $A \in \mathcal{L}(\R^n) \iff TA \in \mathcal{L}(\R^n)$ e in tal caso $\mu(TA) = \abs{\det T}\mu(A)$.
		\qed
	\end{enumerate}
\end{theorem}

\section{Integrale di Lebesgue}

\begin{definition}
	[Funzione semplice]
	Siano $E_j \subseteq \R^n \with j=1,\dots,k$ insiemi misurabili secondo Lebesgue e $C_j \in \R$. Sia inoltre $\chi_{E_j}$ la funzione caratteristica dell'insieme $E_k$ definita come segue:
	$$
		\chi_{E_j}(\vb{x}) =
		\begin{cases}
			1 & \vb{x} \in E_j\\
			0 & \vb{x} \notin E_j
		\end{cases}
	$$
	$\varphi: \R^n \to [0, +\infty)$ definita come $\varphi(\vb{x})=\sum\limits_{j=1}^k C_j \chi_{E_j}(\vb{x})$ è detta funzione semplice. $\varphi$ è una funzione semplice positiva se $C_j \geq 0 \ \forall j=1,\dots,k$.
\end{definition}

\begin{definition}
	[Integrale secondo Lebesgue di una funzione semplice positiva]
	Sia $\varphi(\vb{x})=\sum\limits_{j=1}^k C_j \chi_{E_j}(\vb{x})$ con $C_j \geq 0 \ \forall j=1,\dots,k$. Si definisce integrale secondo Lebesgue di $\varphi$
	$$
		\idotsint_{\R^n}\varphi(x_1,\dots,x_n)\dd x_1 \cdots \dd x_n = \sum_{j=1}^kC_j\mu(E_j)
	$$
	con la convenzione che se $C_j=0$ e $\mu(E_j)=+\infty$, allora $C_j\mu(E_j)=0$.
	Inoltre $\varphi$ si dice sommabile se $\int_{\R^n} \varphi < + \infty$.
\end{definition}

\begin{theorem}
	[Proprietà dell'integrale di Lebesgue sulle funzioni semplici positive]\leavevmode
	\begin{enumerate}
		\item (Linearità) Se $\varphi,\psi$ sono funzioni semplici positive e $c\in \R^+$, allora $c\varphi + \psi$ è una funzione semplice positiva e $\int(c\varphi + \psi)=c\int\varphi +\int\psi$
		\item (Monotonia) Se $\varphi, \psi$ sono funzioni semplici positive e $\varphi \leq \psi \ \forall \vb{x} \in \R^n$, allora  $\int \varphi \leq \int \psi$
		\qed
	\end{enumerate}
\end{theorem}

\begin{definition}
	[Funzione misurabile secondo Lebesgue]
	Siano $A \in \mathcal{L}(\R^n) \e f:A\to \R$. $f$ è misurabile secondo Lebesgue se $\forall \beta \in \R, \ \{\vb{x}\in A:f(\vb{x})<\beta\} \in \mathcal{L}(\R^n)$.
\end{definition}

\begin{remark}
	Le funzioni semplici sono misurabili secondo Lebesgue.
\end{remark}

\begin{theorem}
	[Caratterizzazione delle funzioni misurabili positive tramite funzioni semplici positive]
	Siano $A \subseteq \R^n$ misurabile secondo Lebesgue e $f:A\to [0,+\infty]$ misurabile secondo Lebesgue. Allora esiste una successione crescente di funzioni semplici non negative $\varphi_k:A\to [0,+\infty)$ convergenti ad $f$ puntualmente. Inoltre se $f$ è limitata, la convergenza delle $\varphi_k$ a $f$ è uniforme.
\end{theorem}

\begin{proof}
	Per ogni $k\in \N$, si consideri l'intervallo $[0,2^k]$ nell'insieme dei valori di $f$ e lo si divida in $2^{2k}$ sottointervalli di ampiezza $2^{-k}$.
	$$
		I_{k,l}=\left[\frac{l}{2^k},\frac{l+1}{2^k}\right) \with l=1,\dots,2^{2k}-1
	$$
	Si definisca $J_k=[2^k,+\infty]$. Siano inoltre
	$$
		E_{k,l}=f^{-1}(I_{k,l}) \e F_k=f^{-1}(J_k)
	$$
	$E_{k,l} \e F_k$ sono misurabili. Allora la successione
	$$
		\varphi_k=\sum_{l=0}^{2^{2k}-1}\frac{l}{2^k}\chi_{E_{k,l}}+2^k\chi_{F_k}
	$$
	soddisfa alla tesi del teorema.   
\end{proof}

\begin{definition}
	[Integrale di Lebesgue di una funzione misurabile non negativa]
	Se $A\in \mathcal{L}(\R^n) \e f:A\to [0,+\infty)$ è misurabile, allora $\int_Af=\sup\{\int_A \varphi_k:\varphi_k \leq f, \varphi_k \text{ è una funzione semplice positiva}\}$. In particolare, $f$ è detta sommabile se $\int_A f<+\infty$.
\end{definition}

\begin{theorem}
	[Proprietà dell'integrale di funzioni misurabili non negative]\label{thm:int_nonneg}
	\leavevmode
	\begin{enumerate}
		\item\label{item:mis_int} Ogni funzione misurabile non negativa definita in un insieme misurabile è integrabile secondo Lebesgue
		\item (Linearità) Se $f,g:A\to [0,+\infty]$ sono misurabili in $A\in \mathcal{L}(\R^n)$, allora $f+cg \with c \in \R^+$ è misurabile e $\int_A (f+cg)=\int_Af+c\int_Ag$
		\item (Monotonia) Se $f,g:A\to[0,+\infty]$ sono misurabili in $A\in\mathcal{L}(\R^n) \e f\leq g$, allora $\int_Af\leq\int_Ag$
		\item Se $B\subseteq A \with B\in \mathcal{L}(\R^n)$, allora $\int_Bf=\int_A\chi_Bf$
		\qed
	\end{enumerate}
\end{theorem}

\begin{theorem}
	[Proprietà delle funzioni misurabili]
	Siano $A \in \mathcal{L}(\R^n), \ f,g:A\to\R$ misurabili e $c \in \R$. Allora $f+g,\ cf,\ f/g,\ \abs{f},\ \max\{f,g\} \e \min\{f,g\}$ sono misurabili in $A$. Nel caso della divisione, $g(\vb{x})\neq 0 \ \forall \vb{x} \in A$.
	\qed
\end{theorem}

\begin{theorem}
	[Misurabilità e convergenza puntuale]\label{thm:mis_conv}
	Siano $A \in \mathcal{L}(\R^n)$, $f_k:A\to\R \with k \in \N$ una successione di funzioni misurabili e $f(\vb{x})=\lim\limits_{k\to+\infty}f_k(\vb{x}) \ \forall \vb{x} \in A$. Allora $f$ è misurabile secondo Lebesgue.
	\qed
\end{theorem}

\begin{definition}
	[Funzione integrabile di segno variabile]
	Siano $A\in\mathcal{L}(\R^n) \e f:A\to\R$ misurabile secondo Lebesgue. Si definiscono la parte positiva $f_+$ e la parte negativa $f_-$ di $f$ come nella Definizione \ref{def:ppos_pneg} del Capitolo \ref{chap:peano_jordan}. $f_+ \e f_-$ sono misurabili. Se almeno uno fra $\int_A f_+ \e \int_A f_-$ è finito, si definisce $\int_A f= \int_A f_+ - \int_A f_-$. Se $\int_A f_+,\int_A f_- <+\infty, \ f$ è detta sommabile.
\end{definition}

\begin{theorem}[Proprietà dell'integrale di Lebesgue]
	Se $A \in \mathcal{L}(\R^n), \ f,g:A \to \overline{\R}$ sono misurabili e sommabili secondo Lebesgue, allora:
	\begin{enumerate}
		\item (Linearità) $\forall c \in \R, \ f+cg$ è sommabile e $\int_A(f+cg)=\int_Af+c\int_Ag$
		\item (Monotonia) Se $f\leq g$, allora $\int_Af\leq\int_Ag$
		\item $\abs{f}$ è sommabile e $\abs{\int_Af}\leq\int_A\abs{f}$
		\qed
	\end{enumerate}
\end{theorem}

\begin{theorem}[Integrale di Riemann e di Lebesgue]\leavevmode
	\begin{enumerate}
		\item Se $f:[a,b]\to\R$ è integrabile secondo Riemann, allora $f$ è integrabile secondo Lebesgue e $$\int_{a(\mathcal{R})}^bf=\int_{a(\mathcal{L})}^{b}f$$
		\item Sia $f:[a,b]\to\R$ misurabile secondo Lebesgue. Allora $f$ è integrabile secondo Riemann se e solo se soddisfa queste condizioni:
		\begin{enumerate}[a.]
			\item $f$ è limitata
			\item L'insieme dei punti di discontinuità di $f$ ha misura di Lebesgue nulla
			\qed
		\end{enumerate}
	\end{enumerate}
\end{theorem}

Le funzioni $f:[a,b]\to\R$ continue in $[a,b]$ oppure monotone in $[a,b]$ sono integrabili secondo Riemann e quindi anche secondo Lebesgue. La funzione di Dirichlet ($\chi_{\mathbb{Q}\cap[0,1]}$) non è integrabile secondo Riemann, ma lo è secondo Lebesgue e ha integrale nullo. La funzione
$$
	f(x)=\frac{\dd}{\dd x}\left[x\cos(\frac{1}{x})\right] \with x \in [0,1]
$$
non è integrabile secondo Lebesgue, ma è integrabile in senso generalizzato secondo Riemann.

\begin{theorem}
	[di Fubini]
	Sia $f:\R^n\times\R^m\to\R$ misurabile secondo Lebesgue. $\abs{f}$ è sommabile in $\R^n\times\R^m$ se e solo se almeno uno dei seguenti integrali esiste finito: $\int_{\R^n}\left(\int_{\R^m}\abs{f}\right)<+\infty$ oppure $\int_{\R^m}\left(\int_{\R^n}\abs{f}\right)<+\infty$. In tal caso $f$ è sommabile e 
	$$
		\iint_{\R^n\times\R^m}f(\vb{x},\vb{y})\dd \vb{x} \dd \vb{y} = \int_{\R^n}\dd \vb{x}\left(\int_{\R^m} \dd \vb{y} f(\vb{x},\vb{y})\right)=\int_{\R^m}\dd \vb{y} \left(\int_{\R^n}\dd \vb{x} f(\vb{x},\vb{y})\right)
	$$
	\qed
\end{theorem}

\subsection{Lebesgue e i limiti di successioni}

\begin{theorem}
	[Successioni di funzioni sommabili uniformemente convergenti]
	Siano $A\in \mathcal{L}(\R^n)$ limitato e $f_k,f:A\to\R \with k \in \N$ misurabili. Se $f_k \rightrightarrows f$ in $A$ e $f_k$ è sommabile $\forall k \in \N$, allora $f$ è sommabile e $\displaystyle\int_A f = \lim\limits_{k\to+\infty}\int_Af_k$.
\end{theorem}

\begin{proof}
	$f$ è sommabile:
	$$\displaystyle\abs{\int_A (f-f_k)}\leq \int_A\abs{f-f_k}\leq \int_A\sup_{x \in A}\abs{f-f_k}=M_k\mu_n(A)\xrightarrow{k\to+\infty}0$$
	Inoltre
	$$
		\int_Af=\int_A(f-f_k)+\int_Af_k\then \int_Af=\lim_{k\to+\infty}\int_Af_k
	$$
\end{proof}

\begin{theorem}
	[Convergenza monotona]
	Siano $A \in \mathcal{L}(\R^n)$ e $f_k:A\to[0,+\infty)$ una successione crescente di funzioni misurabili. Allora
	\begin{enumerate}
		\item $f_k$ convergono puntualmente a $f: A\to[0,+\infty]$: $$f(\vb{x})=\lim\limits_{k\to+\infty}f_k(\vb{x})=\sup_kf_k(\vb{x})$$
		\item $f$ è integrabile e $\displaystyle\int_Af=\lim_{k\to+\infty}\int_Af_k$
		\qed
	\end{enumerate}
\end{theorem}

\begin{remark}
	Si noti che le $f_k$ sono integrabili per il punto \textit{\ref{item:mis_int}} del teorema \ref{thm:int_nonneg}. Inoltre $f$ è misurabile per il teorema \ref{thm:mis_conv} e quindi integrabile per il punto \textit{\ref{item:mis_int}} del teorema \ref{thm:int_nonneg}.
\end{remark}

\begin{theorem}
	[della convergenza dominata]
	Siano $A\in \mathcal{L}(\R^n)$ e $f_n:A\to\R$ misurabili. Sotto le seguenti ipotesi:
	\begin{enumerate}[a.]
		\item $\exists g:A\to[0,+\infty)$ sommabile tale che $\abs{f_n}\leq g\ \forall n \in \N \e \int_Ag<+\infty$
		\item $f_n\xrightarrow[n\to+\infty]{A}f$, cioè che $f$ sia il limite puntuale di $f_n$
	\end{enumerate}
	allora $f_n,f$ sono sommabili e $\displaystyle\int_Af=\lim\limits_{n\to+\infty}\int_Af_n$.
	\qed
\end{theorem}
