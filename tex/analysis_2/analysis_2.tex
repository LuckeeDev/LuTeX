% Basic packages
\documentclass[openany]{book}
\usepackage[italian]{babel}
\usepackage[a4paper,margin=1in]{geometry}
\usepackage{amsthm, amssymb, amsmath, bm, physics, esint, graphicx}

% Links
\usepackage{hyperref}
\usepackage[all]{hypcap}

% Define useful shortcuts
\newcommand{\R}{\mathbb{R}}
\newcommand{\N}{\mathbb{N}}
\newcommand{\K}{\mathbb{K}}
\newcommand{\J}{\mathcal{J}}
\newcommand{\C}[1]{\mathcal{C}^{(#1)}}
\newcommand{\calP}{\mathcal{P}}
\newcommand{\A}{\mathcal{A}}
\newcommand{\calR}{\mathcal{R}}
\newcommand{\vecf}{\bm f}
\newcommand{\then}{\Longrightarrow}
\newcommand{\tc}{\text{ tale che }}
\newcommand{\with}{\text{ con }}
\newcommand{\e}{\text{ e }}

% Declare additional math operators
\DeclareMathOperator{\sez}{sez}
\DeclareMathOperator{\inj}{1-1}
\DeclareMathOperator{\su}{su}
\DeclareMathOperator{\injarrow}{\xrightarrow{\inj}}
\DeclareMathOperator{\suarrow}{\xrightarrow{\su}}
\DeclareMathOperator{\bijarrow}{\xrightarrow[\su]{\inj}}
\DeclareMathOperator{\Area}{Area}
\DeclareMathOperator{\divop}{div}
\DeclareMathOperator{\rot}{\vb{rot}}

\newcommand\restr[2]{{% we make the whole thing an ordinary symbol
  \left.\kern-\nulldelimiterspace % automatically resize the bar with \right
  #1 % the function
  \vphantom{\big|} % pretend it's a little taller at normal size
  \right|_{#2} % this is the delimiter
}}

% Enable lettered enumerations
\usepackage[shortlabels]{enumitem}

% Remove dot after theorem number
\usepackage{xpatch}
\makeatletter
\AtBeginDocument{\xpatchcmd{\@thm}{\thm@headpunct{.}}{\thm@headpunct{}}{}{}}
\makeatother

% Style normal pages
\usepackage{fancyhdr}
\pagestyle{fancy}
\fancyhf{}
\fancyhead[LE, RO]{\nouppercase{\leftmark}}
\fancyhead[RE, LO]{\nouppercase{\rightmark}}
\fancyfoot[LE, RO]{\thepage}

% Style chapter page
\fancypagestyle{plain}{%
	\fancyhf{}
	\fancyfoot[LE,RO]{\thepage}
	\renewcommand{\headrulewidth}{0pt}
}

% Do not show chapter number in section
\renewcommand{\thesection}{\arabic{section}}

% Define available theorems and their styles
\newtheorem{theorem}{Teorema}[section]
\newtheorem{corollary}{Corollario}[theorem]
\newtheorem{lemma}[theorem]{Lemma}
\newtheorem{prop}[theorem]{Proposizione}
\newtheorem*{axiom}{Assioma}

\theoremstyle{definition}
\newtheorem{definition}{Definizione}[section]

\theoremstyle{remark}
\newtheorem*{remark}{Osservazione}

% Change QED symbol to #
\renewcommand\qedsymbol{\#}

% Setup document
\title{Analisi Matematica 2}
\author{Luca Zoppetti}
\date{\today}

\begin{document}

\maketitle
\thispagestyle{empty}
\newpage

\
\thispagestyle{empty}
\newpage 

\tableofcontents
\thispagestyle{empty}
\newpage

\
\thispagestyle{empty}
\newpage

\chapter{Topologia negli spazi metrici}

\section{Elementi di topologia}

\begin{definition}[Spazio metrico]
    Si consideri un insieme $X$ e una funzione $d: X \to [0, +\infty)$ con le seguenti proprietà $\forall x,y,z \in X$:
    \begin{enumerate}
        \item $d(x,y) \geq 0$ e $d(x,y) \iff  x = y$
        \item $d(x,y) = d(y,x) \  \forall x, y \in X$
        \item $d(x,y) \leq d(x,z) + d(z,y)$
    \end{enumerate}
    Chiamiamo spazio metrico la coppia $(X, d)$.
\end{definition}

\begin{definition}[Topologia]
    Si consideri un insieme $X$ e una famiglia\footnote{Una famiglia consiste in un insieme, detto insieme di indici, e in una mappa che ad ogni indice associa un unico elemento della famiglia.} di suoi sottoinsiemi $\tau$. La coppia $(X, \tau)$ è uno spazio topologico se $\tau$, che nel caso è detta topologia, ha le seguenti proprietà:
    \begin{enumerate}
        \item $\varnothing, X \in \tau$
        \item $A_{\alpha} \in \tau \ \forall \alpha \then \bigcup\limits_\alpha A_\alpha \in \tau$ (unione anche infinita)
        \item $A_1,\dots,A_k \in \tau \with k \in \N \then \bigcap\limits_{j=1}^k A_j \in \tau$ (intersezione finita)
    \end{enumerate}
\end{definition}

\begin{definition}[Intorno circolare aperto o palla aperta]
    Sia dato $x_0 \in X$. Si definisce intorno circolare aperto di $x_0$ di raggio $r$ l'insieme $B_r(x_0)=\{x \in X : d(x, x_0) < r\}$.
\end{definition}

\begin{definition}
    [Insieme aperto]
    $A \subseteq X$ è un insieme aperto se $\forall x \in A \ \exists r>0 \tc B_r(x) \subseteq A$.
\end{definition}

\begin{prop}
    Ogni intorno circolare aperto è un aperto in $(X,d)$. \qed
\end{prop}

\begin{definition}
    [Insieme chiuso]
    $A \subseteq X$ è un insieme chiuso se il suo complementare $X \setminus A$ è un insieme aperto.
\end{definition}

\begin{remark}
    Esistono insiemi che non sono né aperti né chiusi.
\end{remark}

\begin{prop}
    [Proprietà degli insiemi chiusi] Gli insiemi chiusi hanno le seguenti proprietà:
    \begin{enumerate}
        \item $\varnothing, X$ sono insiemi chiusi
        \item $A_1,\dots,A_k \in X \with k \in \N \then \bigcup\limits_{j=1}^k A_j$ è un insieme chiuso (unione finita)
        \item $A_\lambda, \lambda \in \Lambda\footnote{Con $\Lambda$ si indica un insieme di indici, anche infinito, per l'insieme $X$.} \then \bigcap\limits_{\lambda \in \Lambda} A_\lambda$ è un insieme chiuso (intersezione anche infinita)
        \qed
    \end{enumerate}
\end{prop}

\begin{definition}
    [Distanza di un punto da un insieme]
    Sia $(X,d)$ spazio metrico. Si considerino $A \subseteq X, x \in X$. La distanza di $x$ da $A$ è $d(x,A)=\inf\{d(x,y) : y \in A\}$.
\end{definition}

\begin{definition}
    [Insieme limitato]
    Sia $(X, d)$ uno spazio metrico. $A \subseteq X$ si dice limitato se $\exists M > 0, x_0 \in X \tc A \subseteq B_M(x_0)$.
\end{definition}

\begin{definition}
    [Insieme connesso]
    Sia $(X,d)$ uno spazio metrico. $Y \subseteq X$ è connesso se non esiste una partizione propria di $Y$, ovvero se non esistono due aperti disgiunti $A_1, A_2 \in X$ tali che:
    \begin{itemize}
        \item $A_1, A_2 \neq \varnothing$
        \item $Y \cap A_1, Y \cap A_2 \neq \varnothing$
        \item $(Y\cap A_1)\cup(Y \cap A_2)=Y$
    \end{itemize}
    In altre parole, significa che non è possibile separare i punti di $Y$ usando due aperti in modo non banale.
\end{definition}

\begin{definition}
    [Ricoprimento]
    Sia $(X, \tau)$ uno spazio topologico. Si definisce ricoprimento di $A$ una famiglia $U_\alpha, \alpha \in \Lambda \with U_\alpha \in \tau \ \forall \alpha$ (quindi aperti rispetto a $\tau$)$\tc A \subseteq \bigcup\limits_\alpha U_\alpha$. Si dice sottoricoprimento (finito) una sottofamiglia (finita) di $U_\alpha$ che sia ancora ricoprimento di $A$.
\end{definition}

\begin{definition}
    [Insieme compatto]
    Sia $(X, \tau)$ uno spazio topologico. $A \subseteq X$ è compatto se da ogni ricoprimento di $A$ è possibile estrarre un sottoricoprimento finito di A.
\end{definition}

\begin{theorem}
    [di Heine-Borel]
    Si consideri $(\R^n, d_2)$, ovvero lo spazio euclideo. $A \subseteq \R^n$ è compatto se e solo se $A$ è limitato e chiuso. \qed
\end{theorem}

\section{Chiusura, interno e frontiera}

\begin{definition}
    [Chiusura di un insieme]
    Sia $(X, d)$ uno spazio metrico e $A \subseteq X$. Si dice chiusura dell'insieme $A$ e si indica con $\overline{A}$ l'insieme chiuso più piccolo contenente $A$, ovvero $\overline{A} = \cap\{C\subseteq X : C \text{ è chiuso}, C \supseteq A\}$.
\end{definition}

\begin{definition}
    [Interno di un insieme]
    Sia $(X, d)$ uno spazio metrico e $A \subseteq X$. Si dice interno dell'insieme $A$ e si indica con $\mathring{A}$ l'insieme aperto più grande contenuto in $A$, ovvero $\mathring{A} = \cup\{U\subseteq X: U \text{ è aperto}, U \supseteq A\}$.
\end{definition}

\begin{prop}
    Sia $(X, d)$ uno spazio metrico. $A \subseteq X, \ x,y \in X \then \abs{d(x,A)-d(y,A)}\leq d(x,y)$. \qed
\end{prop}

\begin{definition}
    [Diametro]
    Siano $(X,d)$ uno spazio metrico e $A \subseteq X$. Si dice diametro di $A$ $\text{diam}(A)=\sup\limits_{x,y \in A}d(x,y)$.
\end{definition}

\begin{definition}
    [Punto aderente a un insieme]
    Siano $(X,d)$ uno spazio metrico e $A \subseteq X$. $x \in X$ è aderente ad $A$ se $\forall \varepsilon > 0 \ B_\varepsilon(x)\cap A \neq \varnothing$.
\end{definition}

\begin{definition}
    [Punto di frontiera]
    Siano $(X,d)$ uno spazio metrico e $A \subseteq X$. $x \in X$ è di frontiera per $A$ se è aderente sia ad $A$ che a $X \setminus A$.
\end{definition}

\begin{definition}
    [Frontiera]
    La frontiera dell'insieme $A$, indicata con $\partial A$, è l'insieme dei punti di frontiera per l'insieme $A$.
\end{definition}

\begin{definition}
    [Punto interno]
    Siano $(X,d)$ uno spazio metrico e $A \subseteq X$. $x_0 \in A$ è interno ad $A$ se $\exists \varepsilon > 0 \tc B_\varepsilon(x_0)\subseteq A$.
\end{definition}

\begin{theorem}
    [di caratterizzazione della chiusura]\label{thm:chiusura}
    Siano $(X,d)$ uno spazio metrico e $A \subseteq X$. Allora:
    \begin{enumerate}
        \item $A \subseteq \overline A$
        \item $A, B \subseteq X \then \overline{A} \cup \overline{B} = \overline{A \cup B}$
        \item $\overline{\overline{A}} = \overline{A}$
        \item $A$ chiuso $\iff A = \overline{A}$
        \item $\overline{A} = \{x\in X : d(x,A) = 0\}$
        \item $\overline{A} = \{x \in X : x\text{ aderente ad }A\}$
    \end{enumerate}
\end{theorem}

\begin{proof} Si dimostra ciascun punto separatamente.
    \begin{enumerate}
        \item Ovvio per definizione.
        \item $A, B \subseteq A \cup B \subseteq \overline{A \cup B}$ per il punto 1 del teorema.\\
        Dato che i chiusi che contengono $A \cup B$ contengono sia $A$ che $B$, $\overline A, \overline B \subseteq \overline{A \cup B} \then \overline A \cup \overline B \subseteq \overline{A \cup B}$.\\
        Viceversa, $A \subseteq \overline{A}, B \subseteq \overline{B} \then A \cup B \subseteq \overline{A} \cup \overline{B} \then \overline{A \cup B} \subseteq \overline{A} \cup \overline{B}$.\\
        Di conseguenza, $\overline{A} \cup \overline{B} = \overline{A \cup B}$.
        \item $\overline{A}$ è chiuso, quindi è ovvio usando la definizione.
        \item Segue dal punto 3 del teorema e dalle definizioni di chiuso e chiusura.
    \end{enumerate}
    I punti 5 e 6 non vengono dimostrati, tuttavia si può notare che se $x$ è punto di aderenza per $A$, allora $\forall \varepsilon > 0 \ B_\varepsilon(x) \cap A \neq \varnothing \iff d(x,A)=0$. In altre parole, i punti 5 e 6 sono equivalenti fra loro.
\end{proof}

\begin{theorem}
    [di caratterizzazione di $\mathring A, \overline{A}, \partial A$]
    Siano $(X,d)$ uno spazio metrico e $A \subseteq X$. Allora:
    \begin{enumerate}
        \item $\mathring A = \{x \in A : x \text{ è punto interno ad }A\}$
        \item $\mathring A = A \setminus \partial A$
        \item $\overline{A} = A \cup \partial A$
        \item $\partial A = \partial (X\setminus A)= \overline{X\setminus A}\cap\overline{A}$ (si noti che $\partial A$ è un insieme chiuso)
        \item $A$ è chiuso $\iff A \supseteq \partial A$
    \end{enumerate}
\end{theorem}

\begin{proof}
    Si dimostra ciascun punto separatamente.
    \begin{enumerate}
        \item $x \in \mathring A \subseteq A \then \exists U \text{ aperto} \subseteq X \tc x \in U \subseteq A \then \exists B_\varepsilon(x) \subseteq U \subseteq A$, ovvero $x$ è punto interno ad $A$. Viceversa, se $x \in A$ è punto interno $\exists B_\varepsilon(x)\subseteq A$, quindi $x \in \mathring A$.
        \item $x \in A\setminus \partial A$, quindi $x \in A$. Poiché $x \notin \partial A$, definendo $\delta = d(x, X\setminus A) > 0$, $\forall \ 0 < \varepsilon < \delta \ B_\varepsilon(x) \subseteq A \then x \in \mathring A \then A \setminus \partial A \subseteq \mathring A$.

        Viceversa, se $x \in \mathring A \subseteq A \ \exists B_\varepsilon(x) \subseteq A \then d(x, X\setminus A) \geq \varepsilon > 0$, quindi $x \notin \partial A$, cioè $\mathring A \subseteq A \setminus \partial A$.

        Di conseguenza, $\mathring A = A \setminus \partial A$.
        \item $x \in A \cup \partial A$, allora o $x \in A \subseteq \overline{A}$ o $x \in X \setminus A$ e $x$ aderente ad A. In entrambi i casi, per il teorema \ref{thm:chiusura} $x \in \overline{A}$, ovvero $A \cup \partial A \subseteq \overline{A}$.

        Viceversa, se $x \in \overline{A}$ allora $x$ è aderente ad A (sempre per il teorema \ref{thm:chiusura}). Si verificano due casi:
        \begin{enumerate}[a.]
            \item $x \in A$. Non c'è nient'altro da dimostrare.
            \item $x \in X \setminus A$, ma $x$ dev'essere aderente ad $A$, quindi $x \in \partial A$.
        \end{enumerate}
        Da questo segue che $\overline{A} \subseteq A \cup \partial A \then \overline{A} = A \cup \partial A$.
        \item Ovviamente, $\partial A = \partial (X \setminus A)$, per cui $\overline{X \setminus A} = (X \setminus A) \cup \partial (X \setminus A) = (X \setminus A) \cup \partial A$. Considerando che $\overline{A} = A \cup \partial A$, si vede facilmente che $\overline{A} \cap \overline{X \setminus A} \supseteq \partial A$. Si vuole ora dimostrare l'uguaglianza fra questi due insiemi. Se $x \notin \partial A$ si presentano due casi:
        \begin{enumerate}[a.]
            \item $x \in A$ e $d(x, X \setminus A) > 0 \then x \notin \overline{X \setminus A}$
            \item $x \in X \setminus A$ e $d(x, A)> 0 \then x \notin \overline{A}$
        \end{enumerate}
        In entrambi i casi, $x \notin \overline{A} \cap \overline{X \setminus A}$, quindi $\partial A = \overline{A} \cap \overline{X \setminus A}$.
        
        \item Segue dal punto 3 del teorema e dal teorema \ref{thm:chiusura}. $A \text{ chiuso} \iff A = \overline{A} = A \cup \partial A \iff A \supseteq \partial A$.
    \end{enumerate}
\end{proof}

\begin{definition}
    [Punto di accumulazione]
    Siano $(X, d)$ uno spazio metrico e $A \subseteq X$. $x_0 \in X$ è un punto di accumulazione per $A$ se $\forall \varepsilon > 0 \ (B_\varepsilon(x_0)\setminus \{x_0\})\cap A \neq \varnothing$.
\end{definition}

\begin{definition}
    [Derivato di un insieme]
    Si definisce derivato di $A$ e si indica con $\text{D}(A)$ l'insieme dei punti di accumulazione di $A$.
\end{definition}

\begin{theorem}
    [di Bolzano-Weierstrass]
    Si consideri lo spazio metrico euclideo $(\R^n, d_2)$. Ogni insieme infinito e limitato possiede almeno un punto di accumulazione. \qed
\end{theorem}

\begin{definition}
    [Punto isolato]
    $x \in A$ è un punto isolato di $A$ se $x \notin \text{D}(A)$, ovvero se non è un punto di accumulazione per $A$.
\end{definition}

\section{Spazi normati}

\begin{definition}
    [Norma]
    Sia $X$ uno spazio vettoriale definito sul campo $\K$. $\norm{\cdot} : X \to [0, +\infty)$ è una norma se soddisfa le seguenti proprietà $\forall \vb{x},\vb{y} \in X$ e $\forall \lambda \in \K$:
    \begin{enumerate}
        \item $\norm{\vb{x}} = 0 \iff \vb{x} = \vb{0}$
        \item $\norm{\lambda \vb{x}} = \abs{\lambda}\norm{\vb{x}}$
        \item $\norm{\vb{x} + \vb{y}} \leq \norm{\vb{x}} + \norm{\vb{y}}$
    \end{enumerate}
\end{definition}

\begin{definition}
    [Prodotto interno]
    Sia $X$ uno spazio vettoriale definito sul campo $\K$. $\innerproduct{\cdot}{\cdot}: X \times X \to \R$ è detto prodotto interno se soddisfa le seguenti proprietà:
    \begin{enumerate}
        \item $\forall \vb{x} \in X, \innerproduct{\vb{x}} \geq 0$ e $\innerproduct{\vb{x}} = 0 \iff \vb{x} = \vb{0}$
        \item $\forall \vb{x}, \vb{y} \in X \innerproduct{\vb{x}}{\vb{y}}=\innerproduct{\vb{y}}{\vb{x}}$
        \item $\forall \vb{x}, \vb{y}, \vb{z}, \innerproduct{\vb{x}}{\vb{y}+\vb{z}}=\innerproduct{\vb{x}}{\vb{y}} + \innerproduct{\vb{x}}{\vb{z}}$
        \item $\forall \vb{x}, \vb{y} \in X, \ \forall \lambda \in \K, \innerproduct{\lambda \vb{x}}{\vb{y}} = \lambda \innerproduct{\vb{x}}{\vb{y}} $
    \end{enumerate}
\end{definition}

\begin{definition}
    [Ortogonale]
    Siano $\vb{x}, \vb{y} \in X$. Si dice che $\vb{x}$ è ortogonale a $\vb{y}$ e si scrive $\vb{x} \perp \vb{y}$ se $\innerproduct{\vb{x}}{\vb{y}}$.
\end{definition}

\section{Successioni}

\begin{definition}
    [Successione in uno spazio metrico]
    Sia $(X, d)$ uno spazio metrico. Si definisce successione $f: \N \to X$. Si definisce insieme dei valori della successione l'insieme $\{x_n\}$ dove $x_n = f(n)$.
\end{definition}

\begin{definition}
    [Successione convergente]
    Siano $(X, d)$ uno spazio metrico, $a \in X$ e $x_n \in X \ \forall n \in \N$. Si dice che $x_n$ converge ad $a$ rispetto a $d$ se $\lim\limits_{n \to + \infty}d(x_n, a) = 0$, cioè $\forall \varepsilon > 0 \ \exists m \in \N \tc d(x_n, a) < \varepsilon \ \forall n \geq m$.
\end{definition}

\begin{remark}
    Se $X$ è uno spazio normato, allora $X$ è anche uno spazio metrico con $d(\vb{x},\vb{y}) = \norm{\vb{x}-\vb{y}}$. In $\R^n$ tutte le norme sono equivalenti, cioè se una successione di elementi in $\R^n$ converge rispetto a una norma allora converge rispetto a qualunque altra norma.
\end{remark}

\begin{definition}
    [Successione di funzioni]
    Si chiama successione di funzioni una successione definita da $\N$ all'insieme $X=\{f: I \to \R\} \with I \subseteq \R$ intervallo.
\end{definition}

\begin{definition}
    [Convergenza uniforme]
    Si dice che $f_n$ converge uniformemente a $f$ e si indica con $f_n \rightrightarrows f$ se $d_\infty(f_n, f) = \sup\limits_{x \in I}\abs{f_n(x) - f(x)} \xrightarrow{n \to + \infty} 0$.
\end{definition}

\begin{definition}
    [Convergenza in media]
    Si dice che $f_n$ converge in media a $f$ se $\displaystyle \lim_{n\to + \infty}d_1(f_n,f)=\lim_{n \to + \infty} \int_I\abs{f_n(t) - f(t)}dt = 0$.
\end{definition}

\begin{definition}
    [Convergenza puntuale]
    Si dice che $f_n$ converge puntualmente a $f$ e si indica con $f_n \to f$ se $\displaystyle \lim_{n\to +\infty} f_n(t) = f(t) \ \forall t \in I$.
\end{definition}

\begin{theorem}
    [Proprietà della convergenza uniforme]
    Siano $f_n, f \in B(I, \R) = \{f: I \to \R \text{ limitate}\}$ e sia che $f_n \rightrightarrows f$. Allora:
    \begin{enumerate}
        \item Se $f_n$ è continua $\forall n$ in $I \then f$ è continua in $I$
        \item $f_n$ converge puntualmente a $f$
        \item Se $I$ è un insieme limitato e $\displaystyle \int_I\abs{f_n(t)}dt < + \infty \ \forall n \then \int_I\abs{f(t)dt} < +\infty$ e $f_n \xrightarrow{d_1} f$ (ovvero, $f_n$ converge in media a $f$)
    \end{enumerate}
\end{theorem}

\begin{proof}
    % TODO
\end{proof}

\begin{theorem}
    [Scambio fra limite e derivata]
    Siano $f_n, f, f_n', g: I \to \R$ tali che $d_\infty(f_n, f) \xrightarrow{n\to + \infty} 0$ e $d_\infty(f_n', g) \xrightarrow{n \to +\infty} 0$. Allora $f$ è derivabile in $I$ e $f'(t) = g(t)$. \qed
\end{theorem}

\chapter{Funzioni fra spazi metrici}

\section{Funzioni e regolarità}

\begin{definition}
    [Funzione fra spazi metrici]
    Siano $(X, d_x), (Y, d_y)$ spazi metrici. Si dice funzione $f: X \to Y$ che a $x \in X$ associa $y = f(x) \in Y$.
\end{definition}

\begin{definition}
    [Funzione continua fra spazi metrici]
    Siano $(X, d_x), (Y, d_y)$ spazi metrici. $f: X \to Y$ si dice continua se $\forall x_0 \in X, \ \forall \varepsilon > 0 \ \exists \delta > 0 \tc d_y(f(x), f(x_0)) < \varepsilon \ \forall x \in A \tc d_x(x, x_0) < \delta$.
\end{definition}

\begin{theorem}
    [Caratterizzazioni equivalenti della continuità]
    Siano $(X, d_x), (Y, d_y)$ spazi metrici. Allora le seguenti proposizioni sono equivalenti:
    \begin{enumerate}
        \item $f$ è continua da $X$ in $Y$
        
        \item $f$ trasforma successioni convergenti in $X$ in successioni convergenti in $Y$, cioè $\forall \{x_n\}_{n \in \N}, x_n \in X \ \forall n \tc \exists x \in X$ per cui $\displaystyle \lim_{n \to +\infty}d_x(x_n, x)=0$, si ha che $\displaystyle \lim_{n \to +\infty}d_y(f(x),f(x_n))=0$.
        
        \item $\forall x \in X, \forall V \subseteq Y$ intorno circolare aperto di $f(x)$, $\exists U \subseteq X$ intorno circolare aperto di $x$ tale che $f(U) \subseteq f(V)$.

        \item $\forall V$ aperto in $Y$, $f^{-1}(V)$ è aperto in $X$, dove $f^{-1}(V)=\{x \in X : f(x) \in V\}$.
        \qed
    \end{enumerate}
\end{theorem}

\begin{remark}
    Si noti che il punto 4 del teorema fa uso esclusivamente del concetto di insieme aperto senza fare riferimento alle strutture metriche, ma solo alla struttura topologica. Esso costituisce la definizione di continuità in spazi topologici non metrici.
\end{remark}

\begin{definition}
    [Uniforme continuità]
    Siano $(X, d_x), (Y, d_y)$ spazi metrici. $f : X \to Y$ si dice uniformemente continua se $\forall \varepsilon > 0 \ \exists \delta = \delta (\varepsilon) > 0 \tc d_y(f(x), f(x_0)) < \varepsilon \ \forall x,x_0 \in X \tc d_x(x, x_0) < \delta$.
\end{definition}

\begin{remark}
    Il concetto di uniforme continuità rende il valore di $\delta$ indipendente dalla scelta di $x_0$. Esso dipende quindi solo dalla scelta di $\varepsilon$.
\end{remark}

\begin{definition}
    [Funzione lipschitziana]
    Siano $(X, d_x),\ (Y, d_y)$ spazi metrici. $f: X \to Y$ si dice lipschitziana se $\exists L > 0 \tc \forall x, x_0 \in X \ d_y(f(x), f(x_0)) \leq Ld_x(x, x_0)$. Se $L<1$, $f$ si chiama contrazione.
\end{definition}

\begin{definition}
    [Isometria]
    Siano $(X, d_x),\ (Y, d_y)$ spazi metrici. $f: X \to Y$ è un'isometria se $\forall x, y \in X \ d_y(f(x), f(y)) = d_x(x,y)$.
\end{definition}

\begin{theorem}
    [Relazioni fra regolarità delle funzioni]
    Siano $(X, d_x), (Y, d_y)$ spazi metrici e $f:x \to Y$. Allora:
    \begin{enumerate}
        \item Se $f$ è uniformemente continua, allora è continua.
        \item Se $f$ è lipschitziana, allora è uniformemente continua.
        \item Se $f$ è un'isometria, allora è lipschitziana.
        \qed
    \end{enumerate}
\end{theorem}

\begin{theorem}
    Si consideri lo spazio euclideo $(\R^n,d)$.
    \begin{enumerate}
        \item $x \in \overline{A} \iff \exists \{x_k\}_{k\in \N}, x_k \in A \ \forall k \in \N \tc d(x, x_k) \xrightarrow{k \to + \infty} 0$

        \item $A \subseteq \R^n$ è compatto $\iff \forall\{x_k\}_{k\in \N}, x_k \in A \ \forall k \in \N$ esiste una sottosuccessione $x_{k_l}$ convergente in $A$.
    \end{enumerate}
\end{theorem}

\begin{proof}
    
\end{proof}

\begin{theorem}
    [di Weierstrass]
    Siano $(\R^n, d), (\R^m, d)$ spazi metrici euclidei, $K \subseteq \R^n$ un insieme compatto e $f: K \to \R^m$ una funzione continua. Allora $f(K) \subseteq \R^m$ è un insieme compatto e, se $m=1$, $\exists \min\limits_Kf, \max\limits_Kf$.
    \qed
\end{theorem}

\begin{theorem}
    [di Bolzano]
    Siano $(\R^n, d), (\R^m, d)$ spazi metrici euclidei, $A \subseteq \R^n$ un insieme connesso, $f: A \to \R^m$ una funzione continua. Allora $f(A)$ è connesso e, se $m=1$, $f(A)$ è un intervallo.
    \qed
\end{theorem}

\begin{definition}
    [Insieme connesso per archi]
    Sia $(\R^n, d)$ lo spazio metrico euclideo. $A \subseteq \R^n$ è un insieme connesso per archi se $\forall x, y \in A \ \exists r: [0,1] \to A$ continua tale che $r(0)=x,\ r(1)=y$.
\end{definition}

\begin{theorem}
    Sia $(\R^n, d)$ lo spazio metrico euclideo.
    \begin{enumerate}
        \item Se $A \subseteq \R^n$ è connesso per archi, allora è connesso
        \item Se $A \subseteq \R^n$ è aperto, allora $A$ è connesso $\iff A$ è connesso per archi
        \qed
    \end{enumerate}
\end{theorem}

\section{Spazi metrici completi}

\begin{definition}
    [Successione di Cauchy]
    Sia $(X, d)$ uno spazio metrico. $x_n \in X \ \forall n \in \N, \{x_n\}_{n \in \N}$ è di Cauchy se $\forall \varepsilon > 0 \ \exists \overline{n} \in \N \tc d(x_m, x_p) < \varepsilon \ \forall m, p \geq \overline{n}$.
\end{definition}

\begin{theorem}
    Se $\{x_n\}_{n \in \N}$ converge in $(X, d)$, allora è di Cauchy.
\end{theorem}

\begin{proof}
    
\end{proof}

\begin{definition}
    [Spazio metrico completo]
    $(X, d)$ è uno spazio metrico completo se ogni successione di Cauchy in $X$ ha limite in $X$ rispetto a $d$.
\end{definition}

\begin{definition}
    [Insieme denso]
    Sia $(X,d)$ uno spazio metrico completo. $Z \subseteq X$ è denso in $X$ se $\overline{Z} = X$.
\end{definition}

\begin{theorem}
    [del colapasta]
    Ogni spazio metrico $(Y,d)$ è completabile in modo univoco a meno di isometria. Cioè esiste $X$ tale che $\overline{Y} = X$. Cioè esiste $Z \subseteq X$ denso in $X$ e isometrico a $Y$.
    \qed
\end{theorem}

\section{Proprietà delle funzioni continue}

\begin{theorem}
    [Continuità della somma e del prodotto per scalare]
    $\mathcal{F}=\{f: A \subseteq \R^n \to \R^p\}$ è uno spazio vettoriale, poiché
    \begin{enumerate}[a.]
        \item $(f+g)(x)=f(x)+g(x) \in \R^p$
        \item $(\lambda f)(x)= \lambda f(x) \in \R^p \with \lambda \in \R$
    \end{enumerate}
    Inoltre se $f,g$ sono continue, allora $f+g, \lambda f$ sono continue.
    \qed
\end{theorem}

\begin{theorem}
    [Continuità della composta]
    Siano $A \subseteq \R^n, \ B \subseteq \R^p$, $f: A \to B$ continua e $g: B \to \R^k$ continua, allora $f\circ g: A \to \R^k$ è continua in $A$.
    \qed
\end{theorem}

\begin{theorem}
    [di permanenza del segno]
    Sia $f: A \subseteq \R^n \to \R$ continua in $x_0 \in A$. Se $f(x_0)>0$, allora $\exists \varepsilon > 0 \tc f(x) > 0 \ \forall x \in B_\varepsilon(x_0)$.
    \qed
\end{theorem}

\chapter{Calcolo differenziale per funzioni in più variabili}\label{chap:nvars}

\section{Funzioni differenziabili}

\begin{definition}
    [Derivata parziale]
    Sia $f: A \subseteq \R^n \to \R$ con $A$ aperto. Si definisce derivata parziale $j$-esima di $f$ in $\vb{x_0}$ il valore del seguente limite, se esiste finito:
    $$
        \frac{\partial f}{\partial x_j} (\vb{x_0}) = \displaystyle \lim_{h\to 0} \frac{f(\vb{x_0} + h\vb{\hat{e}_j}) - f(\vb{x_0})}{h} \in \R
    $$
    dove $\vb{\hat{e}_j}$ è il j-esimo vettore della base canonica.
\end{definition}

\begin{definition}
    [Gradiente]
    Siano $A \subseteq \R^n$ un aperto, $\vb{x_0} \in A$ e $f:A \to \R$. Se esistono $\frac{\partial f}{\partial x_1}(\vb{x_0}),\dots,\\\frac{\partial f}{\partial x_n}(\vb{x_0})$, allora si definisce gradiente di $f$ in $\vb{x_0}$ il vettore $\bm{\nabla}f = \left( \frac{\partial f}{\partial x_1} (\vb{x_0}), \dots, \frac{\partial f}{\partial x_n}(\vb{x_0}) \right)$.
\end{definition}

\begin{definition}
    [Derivata direzionale]
    Siano $A \subseteq \R^n$ un aperto, $\vb{x_0} \in A$, $f:A \to \R$ e $\bm{\hat{\nu}} = (\nu_1,\dots,\nu_n) \in \R^n \tc \norm{\bm{\hat{\nu}}}=1$. Si definisce derivata direzionale di $f$ in $\vb{x_0}$ rispetto alla direzione $\bm{\hat{\nu}}$ il valore del seguente limite, se esiste finito:
    $$
        \frac{\partial f}{\partial \bm{\hat{\nu}}} (\vb{x_0}) = \displaystyle \lim_{h\to 0} \frac{f(\vb{x_0} + h\bm{\hat{\nu}}) - f(\vb{x_0})}{h} \in \R
    $$
\end{definition}

\begin{remark}
    La derivabilità in tutte le direzioni non assicura la continuità della funzione $f$ nel punto $\vb{x_0}$. Essa implica infatti solo che $f$ sia continua in $\vb{x_0}$ per ogni restrizione a rette passanti in $\vb{x_0}$, ma non in tutto un intorno di $\vb{x_0}$. È necessario un concetto più forte che viene definito di seguito.
\end{remark}

\begin{definition}
    [Funzione differenziabile]
    Siano $A \subseteq \R^n$ un aperto, $\vb{x_0} \in A$ e $f:A \to \R$. La funzione $f$ è differenziabile in $\vb{x_0}$ se esiste $\vb{m} = (m_1,\dots,m_n) \in \R^n \tc f(\vb{x}) = f(\vb{x_0}) + \ip{\vb{m}}{\vb{x} - \vb{x_0}} + o(\norm{\vb{x} - \vb{x_0}}) \with \norm{\vb{x}-\vb{x_0}}\to 0$.
\end{definition}

\begin{definition}
    [Operatore differenziale]
    Si definisce operatore differenziale e si indica con $\dd{f}_{\vb{x_0}} : \R^n \to \R$ l'operatore $\dd{f}_{\vb{x_0}}(\vb{h})=\ip{\vb{m}}{\vb{h}}$.
\end{definition}

\begin{theorem}
    [Proprietà delle funzioni differenziabili]\label{thm:prop_diff}
    Siano $A \subseteq \R^n$ un aperto, $\vb{x_0} \in A$ e $f:A \to \R$ una funzione differenziabile. Allora:
    \begin{enumerate}
        \item $f$ è continua in $\vb{x_0}$
        \item $f$ è derivabile parzialmente in $\vb{x_0}$ e $\vb{m} = \bm{\nabla}f(\vb{x_0})$
        \item $f$ è derivabile in qualunque direzione $\bm{\hat{\nu}} \tc \norm{\bm{\hat{\nu}}}=1$ e $\frac{\partial f}{\partial \bm{\hat{\nu}}} = \ip{\grad f(\vb{x_0})}{\bm{\hat{\nu}}}$
    \end{enumerate}
\end{theorem}

\begin{proof}
    Si dimostra ciascun punto separatamente.
    \begin{enumerate}
        \item Si verifica che la distanza fra $f(\vb{x})$ e $f(\vb{x_0})$ tende a $0$ se $\vb{x}$ tende a $\vb{x_0}$.
        \begin{align*}
            \abs{f(\vb{x})-f(\vb{x_0})}&=\abs{\ip{\vb{m}}{\vb{x}-\vb{x_0}}+o(\norm{\vb{\vb{x}-\vb{x_0}}})}\leq\\
            &\leq\abs{\ip{\vb{m}}{\vb{\vb{x}-\vb{x_0}}}}+o(\norm{\vb{\vb{x}-\vb{x_0}}})\leq\\
            &\leq\norm{\vb{m}}\norm{\vb{x}-\vb{x_0}}+o(\norm{\vb{x}-\vb{x_0}})\xrightarrow{\vb{x}\to\vb{x_0}} 0
        \end{align*}

        \item Si studia il limite che definisce la derivata parziale $j$-esima di $f$ in $\vb{x_0}$.
        \begin{align*}
            \frac{\partial f}{\partial x_j}&=\lim_{t\to 0}\frac{f(\vb{x_0}+t\hat{\vb{e}}_j)-f(\vb{x_0})}{t}=\\
            &=\lim_{t\to 0}\frac{\ip{\vb{m}}{t\hat{\vb{e}}_j} + o(\abs{t})}{t}=\\
            &=\ip{\vb{m}}{\hat{\vb{e}}_j}=m_j
        \end{align*}
        La derivata parziale di $f$ rispetto a $x_j$ è la $j$-esima componente del vettore $\vb{m}$, quindi $\vb{m}=\grad f(\vb{x_0})$.

        \item Come nel punto \textit{2}, si studia il limite che definisce la derivata direzionale di $f$ in $\vb{x_0}$.
        \begin{align*}
            \frac{\partial f}{\partial \hat{\bm \nu}}&=\lim_{t\to 0}\frac{f(\vb{x_0}+t\hat{\bm \nu})-f(\vb{x_0})}{t}=\\
            &=\lim_{t\to 0}\frac{\ip{\grad f(\vb{x_0})}{t\hat{\bm \nu}} + o(\abs{t})}{t}=\\
            &=\ip{\grad f(\vb{x_0})}{\hat{\bm \nu}}
        \end{align*}
    \end{enumerate}
\end{proof}

\begin{remark}
    Si noti che dal punto \textit{3} del teorema \ref{thm:prop_diff} si evince che la direzione di massima variazione della funzione nel punto $\vb{x_0}$ sia esattamente la direzione del gradiente di $f$ in $\vb{x_0}$.
\end{remark}

\begin{definition}
    [Insieme di livello]
    Siano $A \subseteq \R^n$ un aperto, $f:A \to \R$ e $c \in \R$. Si definisce insieme di livello l'insieme $M_c = \{ \vb{x} \in A : f(\vb{x})=c\}\subseteq A$.
\end{definition}

\begin{theorem}
    [del differenziale totale]\label{thm:diff_tot}
    Siano $A \subseteq \R^n$ un aperto e $f: A \to \R$ derivabile in $A$. Se $\frac{\partial f}{\partial x_j} (\vb{x})$ sono continue in $\vb{x_0} \ \forall j \in [n]$, allora $f$ è differenziabile in $\vb{x_0}$.
\end{theorem}

\begin{proof}[Dimostrazione per n=2.]
    Per la dimostrazione, si scrive $f(x,y)-f(x_0,y_0)=f(x,y)+f(x,y_0)-f(x,y_0)-f(x_0,y_0)$. Applicando il teorema del valor medio di Lagrange al primo e al terzo termine, si ottiene che
    \begin{equation}\label{eq:diff_tot_1}
        f(x,y)-f(x,y_0)=\frac{\partial f}{\partial y}(x,c)[y-y_0]=\left[\frac{\partial f}{\partial y}(x_0,y_0)+o(1)\right](y-y_0) \with (x,y)\to (x_0,y_0)
    \end{equation}
    Applicando lo stesso ragionamento al secondo e al quarto termine, si ottiene che:
    \begin{equation}\label{eq:diff_tot_2}
        f(x,y_0)-f(x_0,y_0)=\frac{\partial f}{\partial x}(d,y_0)[x-x_0]=\left[\frac{\partial f}{\partial x}(x_0,y_0)+o(1)\right](x-x_0) \with (x,y)\to(x_0,y_0)
    \end{equation}
    Unendo le equazioni \eqref{eq:diff_tot_1} e \eqref{eq:diff_tot_2}, si può scrivere il seguente limite:
    \begin{align*}
        &\lim_{\norm{(x-x_0,y-y_0)}\to 0}\frac{f(x,y)-f(x_0,y_0)-\ip{\grad f(x_0,y_0)}{(x-x_0,y-y_0)}}{\norm{(x-x_0,y-y_0)}}=\\
        =&\lim_{\norm{(x-x_0,y-y_0)}\to 0}o(1)\frac{x-x_0+y-y_0}{\norm{(x-x_0,y-y_0)}}=0
    \end{align*}
    Che è esattamente la definizione di differenziabilità, da cui la tesi. Si noti che la frazione nell'ultimo passaggio è limitata per $(x,y)\to(x_0,y_0)$.
\end{proof}

\begin{theorem}
    [Funzioni a gradiente nullo]
    Sia $A \subseteq \R^n$ aperto e connesso. Se $f: A \to \R$ è derivabile in $A$ e $\grad f (\vb{x})=(0,\dots,0) \ \forall \vb{x} \in A$, allora $f$ è costante.
\end{theorem}

\begin{proof}
    $f$ ha le derivate parziali continue in tutto $A$, quindi è differenziabile per il teorema \ref{thm:diff_tot}. Ai fini di questo passaggio, si consideri $A\subseteq\R^2$ e sia $(x_0,y_0)\in\R^2$. Sia $(x,y)$ un punto qualsiasi nella palla di raggio $\varepsilon>0$ centrata in $(x_0,y_0)$. Applicando il teorema del valor medio di Lagrange, si ottiene che
    \begin{align*}
        f(x,y)-f(x_0,y_0)&=f(x,y)+f(x,y_0)-f(x,y_0)-f(x_0,y_0)=\\
        &=\frac{\partial f}{\partial x}(c,y_0)[x-x_0]+\frac{\partial f}{\partial y}(x,d)[y-y_0]=0
    \end{align*}
    Quindi $\exists \varepsilon > 0 \tc \forall \vb{x}\in B_\varepsilon(\vb{x_0})\cap A\ f(\vb{x})=f(\vb{x_0})$.
    
    $A$ è un insieme aperto, quindi essendo connesso è anche connesso per archi. Questo significa che, per qualsiasi $\vb{x},\vb{y} \in A$ è possibile costruire un'applicazione $\vb{r}:[0,1]\to A$ continua tale che $\vb{r}(0)=\vb{x} \e \vb{r}(1)=\vb{y}$. Sia $\overline{t}=\sup\{t \in [a,b]:f(\vb{r}(t))=f(\vb{x})\}$. Per assurdo, si supponga che sia $\overline{t}<1$. Questo è in diretta contraddizione con quanto detto in precedenza, perché vorrebbe dire che non è possibile trovare una palla centrata in $\vb{r}(\overline{t})$ in cui la funzione è costante. Quindi $\overline{t}=1$ e la funzione è costante.
\end{proof}

\section{Derivate di ordine superiore}

\begin{definition}
    [Derivata di ordine $k$]
    Siano $A \subseteq \R^n$ un aperto, $\vb{x_0} \in A$ e $f:A \to \R$. Se $\displaystyle \exists \frac{\partial f}{\partial x_{i_1}}, \frac{\partial^2 f}{\partial x_{i_1} \partial x_{i_2}},\dots,\frac{\partial ^{k-1}f}{\partial x_{i_1} \cdots \partial x_{i_{k-1}}} : B_\varepsilon(\vb{x_0}) \to \R$, si definisce derivata $k$-esima di $f$ in $\vb{x_0}$ rispetto a $x_{i_1},\dots,x_{i_{k}}$ il valore del seguente limite se esiste finito:
    $$
        \lim_{h\to 0}\frac{\frac{\partial^{k-1}f}{\partial x_{i_1}\cdots \partial x_{i_{k-1}}}(\vb{x_0}+ h \vb{\hat{e}_{i_k}})- \frac{\partial^{k-1}f}{\partial x_{i_1}\cdots \partial x_{i_{k-1}}}(\vb{x_0})}{h} \in \R
    $$
\end{definition}

\begin{definition}
    [Insieme $\C{k}(A, \R)$]
    Si definisce l'insieme delle funzioni derivabili con continuità $k$ volte in $A\subseteq\R^n$ aperto:
    $$
        \displaystyle \C{k}(A,\R) = \left\{ f: A \to \R \tc \ \exists \frac{\partial f}{\partial x_i},\dots,\frac{\partial ^2 f}{\partial x_i \partial x_j}, \dots, \frac{\partial ^k f}{\partial x_{i_1}\cdots\partial x_{i_k}} \text{ continue in } A \right\}
    $$
\end{definition}

\begin{remark}
    Si noti che per il teorema \ref{thm:diff_tot} è sufficiente che siano continue le derivate $k$-esime, perché le altre lo sono di conseguenza. Devono comunque essere continue tutte le combinazioni possibili di derivate, che crescono molto velocemente ($n^k$ combinazioni per le derivate di ordine $k$).
\end{remark}

\begin{theorem}
    [di Schwarz]\label{thm:schwarz}
    Siano $A \subseteq \R^n$ un aperto e $f: A \to \R$ derivabile due volte in $A$. Se $\frac{\partial ^2 f}{\partial x_i \partial x_j}$ e $\frac{\partial^2 f}{\partial x_j \partial x_i}$ sono continue in $\vb{x_0} \in A$, allora
    $$
        \frac{\partial ^2 f}{\partial x_i \partial x_j}(\vb{x_0}) = \frac{\partial^2  f}{\partial x_j \partial x_i}(\vb{x_0})
    $$
\end{theorem}

\begin{proof}
    [Dimostrazione per n=2.]
    Sia $F(x)=f(x,y)-f(x,y_0)$. Applicando due volte il teorema del valor medio di Lagrange si ottiene che
    \begin{align*}
        F(x)-F(x_0)&=F'(c)[x-x_0]=\left(\frac{\partial f}{\partial x}(c,y)-\frac{\partial f}{\partial x}(c,y_0) \right)[x-x_0]=\\
        &=\frac{\partial^2 f}{\partial x \partial y}(c,d)[x-x_0][y-y_0]
    \end{align*}
    Inoltre,
    $$
        F(x)-F(x_0)=f(x,y)-f(x,y_0)-f(x_0,y)+f(x_0,y_0)=G(y)-G(y_0)
    $$
    dove $G(y)=f(x,y)-f(x_0,y)$. Applicando nuovamente Lagrange come in precedenza, si ottiene
    \begin{align*}
        G(y)-G(y_0)&=G'(\xi)[y-y_0]=\left(\frac{\partial f}{\partial y}(x,\eta)-\frac{\partial f}{\partial y}(x_0,\eta)\right)[y-y_0]=\\
        &=\frac{\partial ^2f}{\partial y\partial x}(\xi,\eta)[x-x_0][y-x_0]
    \end{align*}
    Essendo le derivate seconde continue, si possono mandare $(c,d)\to(x_0,y_0) \e (\xi,\eta)\to(x_0,y_0)$ e ottenere la tesi:
    $$
        \frac{\partial^2 f}{\partial x \partial y}(x_0,y_0)=\frac{\partial^2 f}{\partial y \partial x}(x_0,y_0)
    $$
\end{proof}

\begin{definition}
    [Matrice hessiana]
    Siano $A \subseteq \R^n$ un aperto e $f: A \to \R$ derivabile due volte in $A$. Si definisce matrice hessiana di $f$ in $\vb{x_0}$ la matrice
    $$
        H_f(\vb{x_0}) =
        \begin{bmatrix}
            \grad \frac{\partial f}{\partial x_1}(\vb{x_0})\\
            \vdots\\
            \grad \frac{\partial f}{\partial x_n}(\vb{x_0})
        \end{bmatrix}
    $$
\end{definition}

\begin{remark}
    Per il teorema \ref{thm:schwarz}, se $f$ è $\C{2}$ in $\vb{x_0} \in A$, allora $H_f(\vb{x_0})$ è simmetrica.
\end{remark}

\begin{theorem}
    [Formula di Taylor di ordine $k$]
    Sia $A \subseteq \R^n$ un aperto. Se $f \in \C{k}(A, \R)$, allora vale la seguente
    \begin{align*}
        f(\vb{x_0}+ \vb{h})= f(\vb{x_0})&+\sum_{j=1}^{n}\frac{\partial f}{\partial x_j}(\vb{x_0})h_j + \frac{1}{2}\sum_{i,j=1}^{n}\frac{\partial^2 f}{\partial x_i \partial x_j}(\vb{x_0})h_ih_j+\cdots+\\
        &+\frac{1}{k!}\sum_{i_1,\dots,i_k=1}^n\frac{\partial^k}{\partial x_{i_1}\cdots\partial x_{i_k}}(\vb{x_0})h_{i_1}\cdots h_{i_k} +\\
        &+o (\norm{\vb{h}}^k) \with \norm{\vb{h}} \to 0
    \end{align*}
    \qed
\end{theorem}

\begin{remark}
    In particolare, per $n=2$ vale
    $$
        f(\vb{x_0} + \vb{h}) = f(\vb{x_0})+\ip{\grad f(\vb{x_0})}{\vb{h}} + \frac{1}{2}\ip{\vb{h}}{H_f(\vb{x_0})\vb{h}}+o(\norm{\vb{h}}^2) \with \norm{\vb{h}} \to 0
    $$
\end{remark}

\section{Punti critici liberi}

\begin{definition}
    [Massimo (minimo) locale]\label{def:minmax}
    Sia $A \subseteq \R^n$ e $f: A \to \R$. Il punto $\vb{x_0}$ è un punto di massimo (minimo) locale per $f$ se $\exists \delta > 0 \tc f(\vb{x}) \leq f(\vb{x_0})\ (f(\vb{x})\geq f(\vb{x_0})) \ \forall \vb{x} \in A \cap B_\delta (\vb{x_0})$.
\end{definition}

\begin{definition}
    [Punto di sella]\label{def:sella}
    Sia $A \subseteq \R^n$ e $f: A \to \R$. Il punto $\vb{x_0}$ è un punto di sella per $f$ se $\grad f(\vb{x_0})=\vb{0}$ e $\forall \varepsilon > 0 \ \exists \vb{x_1}, \vb{x_2} \in A \cap B_\varepsilon(\vb{x_0}) \tc f(\vb{x_1}) > f(\vb{x_0}) > f(\vb{x_2})$.
\end{definition}

\begin{remark}
    Si noti che nelle definizioni \ref{def:minmax} e \ref{def:sella} non si richiede alcuna regolarità di $f$ e alcuna proprietà particolare di $A$.
\end{remark}

\begin{definition}
    [Punto critico (o stazionario)]
    Siano $A \subseteq \R^n$ un aperto e $f: A\to \R$. $\vb{x_0} \in A$ è un punto critico (o stazionario) di $f$ se $\exists \grad f(\vb{x_0})$ e $\grad f(\vb{x_0})=\vb{0}$.
\end{definition}

\begin{theorem}
    [di Fermat]\label{thm:fermat}
    Sia $A \subseteq \R^n$ un aperto e $f: A \to \R$. Se $\vb{x_0} \in A$ è massimo o minimo locale per $f$ ed esiste il gradiente di $f$ in $\vb{x_0}$, allora $\grad f(\vb{x_0})=\vb{0}$, cioè $\vb{x_0}$ è punto critico di $f$.
\end{theorem}

\begin{proof}
    Sia $F_j(t)=f(\vb{x_0}+t\hat{\vb{e}}_j):(-\varepsilon,\varepsilon)\to\R$, dove $\hat{\vb{e}}_j$ è il $j$-esimo vettore della base canonica. Per costruzione $t=0$ è un punto di massimo o minimo locale per $F_j$, di conseguenza per il teorema di Fermat in una dimensione $$\frac{\dd F_j}{\dd t}(0)=0$$
    Inoltre, si ha che
    \begin{align*}
        \frac{\dd F_j}{\dd t}(0) =\lim_{h\to0}\frac{F(h)-F(0)}{h}= \lim_{h\to 0}\frac{f(\vb{x_0}+h\hat{\vb{e}}_j)-f(\vb{x_0})}{h}=\frac{\partial f}{\partial x_j}(\vb{x_0})
    \end{align*}
    Dove nell'ultima uguaglianza si è usato il fatto che per ipotesi esistono tutte le derivate parziali di $f$. In conclusione, ogni componente del gradiente è nulla quindi si ha la tesi.
\end{proof}

\subsection{Forme quadratiche}

Sia $M \in \mathcal{M}_{n \times n}(\R)$ una matrice quadrata di ordine $n$. La seguente espressione si chiama forma quadratica associata alla matrice $M$:
$$
    q_M(\vb{h}) = \ip{\vb{h}}{M\vb{h}} \in \R
$$
La forma quadratica può essere:
\begin{itemize}
    \item Definita positiva se $\forall \vb{h}\neq \vb{0}, \ q_M(\vb{h})> 0$
    \item Definita negativa se $\forall \vb{h}\neq \vb{0}, \ q_M(\vb{h}) < 0$
    \item Semidefinita positiva se $\forall \vb{h}\in \R^n, \ q_M(\vb{h}) \geq 0$ e $\exists \vb{k} \tc q_M(\vb{k})=0$
    \item Semidefinita negativa se $\forall \vb{h}\in \R^n, \ q_M(\vb{h}) \leq 0$ e $\exists \vb{k} \tc q_M(\vb{k})=0$
    \item Indefinita se $\exists \vb{h}, \vb{k} \in \R^n \tc q_M(\vb{h}) < 0 < q_M(\vb{k})$
\end{itemize}

Matrici diverse possono essere associate alla stessa forma quadratica, infatti è sufficiente che abbiano gli stessi elementi sulla diagonale e che la somma $a_{ij}+a_{ji}=b_{ij}+b_{ji} \ \forall i\neq j\with i,j \in [n]$. Fra tutte le matrici associate alla stessa forma quadratica, ci si può restringere alle matrici simmetriche (che sono sempre diagonalizzabili!), per cui esistono criteri di classificazione efficaci.

\begin{theorem}
    [Classificazione delle forme quadratiche e segno degli autovalori]
    Sia $A \in \mathcal{M}_{n\times n}(\R)$ tale che $A=A^T$ e $q_A$ la forma quadratica associata ad $A$. Sia la coppia $(p,q)$ la segnatura della matrice $A$, dove $p$ rappresenta la somma delle molteplicità algebriche degli autovalori positivi e $q$ la somma delle molteplicità algebriche degli autovalori negativi. Allora:
    \begin{enumerate}
        \item $q_A$ è definita positiva $\iff (p,q)=(n,0)$
        \item $q_A$ è definita negativa $\iff (p,q)=(0,n)$
        \item $q_A$ è semidefinita positiva $\iff (p,q)=(m,0) \with 0< m < n$
        \item $q_A$ è semidefinita negativa $\iff (p,q)=(0,l) \with 0< l < n$
        \item $q_A$ è indefinita $\iff (p,q)=(m,l) \with 0<m<n,0<l<n$
        \qed
    \end{enumerate}
\end{theorem}

\begin{theorem}
    [Criterio di Sylvester]
    Sia $A \in \mathcal{M}_{n\times n}(\R) \tc A=A^T\neq \vb{O}$ e sia $A_k$ il minore principale nord-ovest di ordine $k$ ottenuto selezionando le prime $k$ righe e le prime $k$ colonne di $A$. Allora la matrice $A$ è:
    \begin{enumerate}
        \item Definita positiva $\iff \det A_k > 0 \ \forall k \in [n]$
        \item Definita negativa $\iff \det A_k < 0$ se $k$ è dispari e $\det A_k > 0$ se $k$ è pari $ \ \forall k \in [n]$
        \item Semidefinita positiva $\iff \det A_k \geq 0 \ \forall k \in [n]$ e $\exists j \in [n] \tc \det A_j = 0$
        \item Semidefinita negativa $\iff \det A_k \leq 0$ se $k$ è dispari e $\det A_k \geq 0$ se $k$ è pari $ \ \forall k \in [n]$ e $\exists j \in [n] \tc \det A_j=0$
        \item Indefinita altrimenti
        \qed
    \end{enumerate}
\end{theorem}

\subsection{Classificazione dei punti critici}

\begin{theorem}
    [Classificazione con la matrice hessiana]
    Sia $f \in \C{2}(A, \R) \with A\subseteq \R^n$ aperto. Allora:
    \begin{enumerate}
        \item Se $\vb{x_0} \in A$ è punto di minimo locale per $f$, allora $\grad f(\vb{x_0})=\vb{0}$ e $H_f(\vb{x_0})$ è o definita positiva o semidefinita positiva
        \item Se $\vb{x_0} \in A$ è punto di massimo locale per $f$, allora $\grad f(\vb{x_0})=\vb{0}$ e $H_f(\vb{x_0})$ è o definita negativa o semidefinita negativa
        \item Se $\grad f(\vb{x_0})=\vb{0}$ e $H_f(\vb{x_0})$ è definita positiva, allora $\vb{x_0}$ è punto di minimo locale per $f$
        \item Se $\grad f(\vb{x_0})=\vb{0}$ e $H_f(\vb{x_0})$ è definita negativa, allora $\vb{x_0}$ è punto di massimo locale per $f$
        \item Se $\grad f(\vb{x_0})=\vb{0}$ e $H_f(\vb{x_0})$ è indefinita, allora $\vb{x_0}$ è un punto di sella per $f$
    \end{enumerate}
\end{theorem}

\begin{proof}
    Si dimostrano i punti \textit{1} e \textit{3}. I punti \textit{2} e \textit{4} hanno la dimostrazione identica, il punto \textit{5} non è stato affrontato.
    \begin{enumerate}
        \item $\vb{x_0}$ è un punto di minimo locale per $f$ e il suo gradiente esiste perché $f\in\C{2}$, allora per il teorema \ref{thm:fermat} $\grad f(\vb{x_0}) = \vb{0}$. Sia $F:(-\varepsilon,\varepsilon)\to \R$ definita come segue:
        $$
            F(t)=f(\vb{x_0}+t\hat{\bm\nu})\with \hat{\bm\nu} \in \R^n\tc \norm{\hat{\bm\nu}}=1
        $$
        Allora per costruzione $t=0$ è un punto di minimo locale per $F$. Per questo motivo,
        \begin{align*}
            0&\leq F''(0)=\lim_{h\to 0}\frac{\frac{\partial f}{\partial \hat{\bm\nu}}(\vb{x_0}+h\hat{\bm\nu})-\frac{\partial f}{\partial \hat{\bm\nu}}(\vb{x_0})}{h}=\\
            &=\lim_{h\to 0}\sum_{i=1}^n\frac{[\partial_i f(\vb{x_0 + h\hat{\bm\nu}})-\partial_i f(\vb{x_0})]\nu_i}{h}=\\
            &=\sum_{i=1}^{n}\frac{\partial}{\partial \hat{\bm\nu}}\frac{\partial f}{\partial x_i}(\vb{x_0})\nu_i=\\
            &=\sum_{i=1}^n\sum_{j=1}^{n}=\frac{\partial^2 f}{\partial x_i \partial x_j}(\vb{x_0})\nu_i\nu_j=\ip{\hat{\bm\nu}}{H_f(\vb{x_0})\hat{\bm\nu}}
        \end{align*}
        Questo significa che $H_f(\vb{x_0})$ è definita positiva o semidefinita positiva.
        \addtocounter{enumi}{1}
        \item Si applichi la formula di Taylor alla funzione $f$ nel punto $\vb{x_0}$. Allora, essendo $\grad f(\vb{x_0})=\vb{0}$,
        $$
            f(\vb{x_0}+\vb{h})-f(\vb{x_0}) = \frac{1}{2}\ip{\vb{h}}{H_f(\vb{x_0})\vb{h}}+o(\norm{h}^2) \with \norm{h}\to 0
        $$
        Sia $\overline{\lambda}>0$ il minimo degli autovalori di $H_f(\vb{x_0})$, allora vale la seguente maggiorazione:
        \begin{gather*} 
            \ip{\vb{h}}{H_f(\vb{x_0})\vb{h}}\geq \overline{\lambda}\norm{\vb{h}^2}\\
            \then f(\vb{x_0}+\vb{h})-f(\vb{x_0})\geq \norm{h}^2\left(\frac{1}{2}\overline{\lambda}+o(1)\right)\geq 0
        \end{gather*}
        Per cui $\vb{x_0}$ è un punto di minimo locale per $f$.
    \end{enumerate}
\end{proof}

\chapter{Calcolo differenziale per funzioni a valori vettoriali}

\section{Funzioni derivabili e differenziabili}

\begin{definition}
    [Funzione derivabile]
    Sia $\vecf=(f_1,\dots,f_k): A\subseteq \R^n \to \R^k \with A$ aperto. $\vecf$ è derivabile in $\vb{x_0} \in A$ se $\exists \frac{\partial f_1}{\partial x_i}(\vb{x_0}),\dots,\frac{\partial f_k}{\partial x_i}(\vb{x_0}) \in \R \ \forall i \in [n]$. In questo caso si definisce la matrice jacobiana di $\vecf$ in $\vb{x_0}$ come segue:
    $$
        J_{\vecf}(\vb{x_0})=
        \begin{bmatrix}
            \grad f_1 (\vb{x_0})\\
            \vdots \\
            \grad f_k (\vb{x_0})
        \end{bmatrix}
        \in \mathcal{M}_{k\times n}(\R)
    $$
\end{definition}

\begin{definition}
    [Funzione differenziabile]
    Sia $\vecf=(f_1,\dots,f_k): A\subseteq \R^n \to \R^k \with A$ aperto. $\vecf$ è differenziabile in $\vb{x_0} \in A$ se $\exists M \in \mathcal{M}_{k\times n}(\R)$ tale che
    $$
        \vecf(\vb{x_0} + \vb{h})= \vecf(\vb{x_0}) + M\vb{h} + \bm o (\norm{\vb{h}}) \in \R^k \with \norm{\vb{h}}\to 0
    $$
\end{definition}

\begin{theorem}
    [Condizioni per differenziabilità]
    Sia $A \subseteq \R^n$ un aperto e $\vecf= (f_1,\dots,f_k):A \to \R^k$.
    \begin{enumerate}
        \item $\vecf$ è differenziabile in $\vb{x_0} \in A \iff f_1,\dots,f_k$ sono differenziabili in $\vb{x_0} \in A$. In tal caso
        $$
            M=J_{\vecf}(\vb{x_0})=
            \begin{bmatrix}
                \grad f_1(\vb{x_0})\\
                \vdots\\
                \grad f_k(\vb{x_0})
            \end{bmatrix}
            \in \mathcal{M}_{k\times n}(\R)
        $$
        \item Se $\vecf$ è differenziabile in $\vb{x_0} \in A$, allora $\vecf$ è continua in $\vb{x_0}$
        \item Se $\vecf$ è derivabile in $\vb{x_0} \in A$ e le derivate parziali sono tutte continue in $\vb{x_0}$, allora $\vecf$ è differenziabile
        \qed
    \end{enumerate}
\end{theorem}

\begin{theorem}
    [di differenziabilità della funzione composta]
    Siano $A\subseteq \R^n, B \subseteq \R^k$ aperti. Siano $\vecf: A \to B$ differenziabile in $\vb{x_0} \in A$ e $\bm g:B \to \R^l$ differenziabile in $\vb{y_0} = \vecf(\vb{x_0})$, allora $\bm g \circ \vecf : A \to \R^l$ è differenziabile in $\vb{x_0}$ e $J_{\bm g \circ \vecf}(\vb{x_0})=J_{\bm g}(\vecf (\vb{x_0}))\cdot J_{\vecf}(\vb{x_0}) \in \mathcal{M}_{l\times n}$.
    \qed
\end{theorem}

\begin{definition}
    [Diffeomorfismo]
    Siano $A,B \subseteq \R^n$ aperti. La funzione $\vecf: A \to B$ è un diffeomorfismo $\C{k}$ se $\vecf$ è iniettiva e suriettiva e $\vecf, \vecf^{-1}$ sono $\C{k}$ nei rispettivi domini.
\end{definition}

\begin{theorem}
    [Differenziale della funzione inversa]
    Siano $A,B \subseteq \R^n$ aperti e sia $\vecf:A\to B$ un diffeomorfismo $\C{1}$. Allora vale che
    $$
        J_{\vecf^{-1}}(\vb{y}) = \left(J_{\vecf}\left(\vecf^{-1}\left(\vb{y}\right)\right) \right)^{-1} \ \forall \vb{y} \in B
    $$
\end{theorem}

\begin{proof}
    $\vecf$ è invertibile perché è un diffeomorfismo. Di conseguenza,
    \begin{gather*}
        \vecf\left(\vecf^{-1}(\vb{y})\right)=\vb{y}
        \then J_{\vecf}\left(\vecf^{-1}(\vb{y})\right)\cdot J_{\vecf^{-1}}(\vb{y})=\text{id}_{n\times n}
        \then J_{\vecf^{-1}}(\vb{y})=\left(J_{\vecf}\left(\vecf^{-1}(\vb{y})\right)\right)^{-1}
    \end{gather*}
\end{proof}

\section{Varietà regolari}

\begin{definition}
    [Varietà regolare $(n-k)$-dimensionale]
    Siano $A \subseteq \R^n$, $\bm g=(g_1,\dots,g_k):A\to\R^k \in \C{1}$. Detto $\Gamma =\{\vb{x}\in A : g_1(\vb{x})=0,\dots,g_k(\vb{x})=0\}$ l'insieme delle soluzioni del sistema di equazioni, $\Gamma$ è una varietà regolare $(n-k)$-dimensionale se $\forall \vb{x} \in \Gamma \ J_{\bm g}(\vb{x})$ ha rango uguale a $k$.
\end{definition}

\begin{definition}
    [Vettore tangente]
    Il vettore $\vb{h} \in \R^n$ è tangente a $\Gamma$ in $\vb{x_0}$ se $\exists \varphi: (-\varepsilon, \varepsilon) \to \Gamma \subseteq \R^n$ di classe $\C{1} \tc \varphi (0) = \vb{x_0} \e \varphi ' (0) = \vb{h}$.
\end{definition}

\begin{definition}
    [Spazio tangente]
    Se $\Gamma$ è una varietà regolare $(n-k)$-dimensionale e $\vb{x_0} \in \Gamma$, si definisce lo spazio tangente come segue:
    $$
        T_{\vb{x_0}}\Gamma = \{ \vb{h}=(h_1,\dots,h_n) \in \R^n : \ip{\grad g_1(\vb{x_0})}{\vb{h}}=0,\dots,\ip{\grad g_k(\vb{x_0})}{\vb{h}}=0 \}
    $$
\end{definition}

\begin{remark}
    Si noti che $\grad g_1(\vb{x_0}),\dots,\grad g_k (\vb{x_0})$ costituisce un sistema di vettori linearmente indipendenti perché $\rank(J_{\bm g})=k \ \forall \vb{x} \in \Gamma$. Per questo motivo, $\dim T_{\vb{x_0}}\Gamma= n-k$.
\end{remark}

\begin{definition}
    [Spazio normale]
    Se $\Gamma$ è una varietà regolare $(n-k)$-dimensionale e $\vb{x_0} \in \Gamma$, si definisce lo spazio normale come segue:
    $$
        N_{\vb{x_0}}\Gamma = \left\{ \vb{h}=(h_1,\dots,h_n) \in \R^n : \exists \lambda_1, \dots, \lambda_k \in \R \e \vb{h}=\sum_{j=1}^n \lambda_j \grad g_j(\vb{x_0}) \right\}
    $$
    Si noti che $\dim N_{\vb{x_0}}\Gamma = k$.
\end{definition}

\begin{definition}
    [Iperpiano tangente]
    A partire dalla nozione di spazio tangente si può definire l'iperpiano tangente nel punto $\vb{x_0}$, che è costituito dall'insieme dei punti $\vb{x}$ tali che il loro vettore distanza dal punto $\vb{x_0}$ appartenga allo spazio tangente a $\Gamma$:
    $$
        I= \{ \vb{x} \in \R^n :  \ip{\grad g_1(\vb{x_0})}{\vb{x}-\vb{x_0}}=0,\dots,\ip{\grad g_k(\vb{x_0})}{\vb{x}-\vb{x_0}}=0 \}
    $$
\end{definition}

\section{Teorema di Dini}

\begin{theorem}
    [Caso con $1$ equazione in $2$ incognite]
    Sia $A \subseteq \R^2$ aperto, $g \in \C{1}(A, \R)$ e $(x_0,y_0) \in A \tc g(x_0, y_0)=0 \e \partial_y g(x_0,y_0)\neq 0$.
    Allora,
    \begin{enumerate}
        \item $\exists \delta, \varepsilon>0 \tc W=[x_0-\delta,x_0+\delta]\times[y_0-\varepsilon,y_0+\varepsilon] \subseteq A$
        \item $\exists f \in \C{1}([x_0-\delta,x_0+\delta],[y_0-\varepsilon,y_0+\varepsilon]) \tc g(x,y)=0 \with (x,y) \in W \iff y=f(x) \with x \in [x_0-\delta,x_0+\delta]$
    \end{enumerate}
    In altre parole, se si restringe l'equazione $g(x,y)=0$ al rettangolo $W$ l'insieme delle sue soluzioni è il grafico di $f$. Inoltre
    $$
        \frac{\dd f}{\dd x}(x)=-\frac{\partial_x g(x,f(x))}{\partial_y g(x,f(x))}
    $$
\end{theorem}

\begin{proof}
    [Dimostrazione per $n=2$ e $k=1$.]
    La dimostrazione si articola in tre passaggi: innanzitutto si costruisce $f$, poi ne si verifica la continuità e infine si calcola la sua derivata.


    Per ipotesi, $\partial_yg(x_0,y_0) \neq 0$. Ai fini della dimostrazione, si supponga che $\partial_yg(x_0,y_0) > 0$. Per il teorema di permanenza del segno (Thm. \ref{thm:sign}, Cap. \ref{chap:functions}), esiste una scatola $W=[x_0-\delta,x_0+\delta]\times[y_0-\varepsilon,y_0+\varepsilon]\with \delta,\varepsilon>0 \tc \partial_yg(x,y)>0\ \forall (x,y)\in W$. Sia $h(y)=g(x_0,y)\in\C{1}$. Per quanto appena detto, $h(y)$ è una funzione strettamente crescente nell'intervallo $[y_0-\varepsilon, y_0+\varepsilon]$, per cui $h(y_0-\varepsilon)<0=h(y_0)<h(y_0+\varepsilon)$. Si è appena dimostrato che nel punto medio delle basi del rettangolo la funzione $g$ assume segni discordi. Applicando nuovamente il teorema di permanenza del segno, questa volta alla funzione $g(x,y)$ sulle basi del rettangolo, si ha che esiste un intervallo $[x_0-\delta ', x_0+\delta ']$ all'interno del quale $g(x,y_0-\varepsilon)<0<g(x,y_0+\varepsilon)$. Si noti che $\delta$ potrebbe non coincidere con $\delta '$, quindi, detto $\delta=\min\{\delta,\delta '\}$, per comodità la base del rettangolo sarà ancora $[x_0-\delta,x_0+\delta]$.
    

    Sia ora la funzione $\tilde{h}(y)=g(\overline{x},y) \in \C{1} \with \overline{x} \in [x_0-\delta,x_0+\delta]$ definita per $y \in [y_0-\varepsilon,y_0+\varepsilon]$. Per quanto detto in precedenza, $\tilde{h}(y_0-\varepsilon)<0<\tilde{h}(y_0+\varepsilon)$. Di conseguenza, per il teorema degli zeri, $\exists!\ \overline{y} \in [y_0-\varepsilon,y_0+\varepsilon] \tc \tilde{h}(\overline{y})=g(\overline{x},\overline{y})=0$. È quindi possibile definire $f:[x_0-\delta,x_0+\delta]\to[y_0-\varepsilon,y_0+\varepsilon] \tc f(\overline{x})=\overline{y} \ \forall\, \overline{x} \in [x_0-\delta,x_0+\delta]$.


    Sia $(\overline{x},f(\overline{x}))\in W$. Per verificare la continuità di $f$, si fissi $\overline{\varepsilon}>0$. Allora, per lo stesso ragionamento applicato in precedenza, $g(\overline{x},f(\overline{x})-\overline{\varepsilon})<0=g(\overline{x},f(\overline{x}))<g(\overline{x}, f(\overline{x})+\overline{\varepsilon})$. Per questo motivo, $\exists \overline{\delta} \tc g(x,f(\overline{x})-\overline{\varepsilon}) < 0 < g(x,f(\overline{x})+\overline{\varepsilon}) \ \forall x \in [\overline{x}-\overline{\delta},\overline{x}+\overline{\delta}]\cap[x_0-\delta,x_0+\delta]$. Questo garantisce che $f(\overline{x})-\overline{\varepsilon} < f(x) < f(\overline{x})+\overline{\varepsilon}$, che è esattamente la definizione di continuità (Def. \ref{def:f_cont}, Cap. \ref{chap:functions}).


    Per calcolare la derivata di $f$ in $x_1$, che è $\displaystyle\lim_{x_2\to x_1}\frac{f(x_2)-f(x_1)}{x_2-x_1}$, si introduce la funzione ausiliaria $\bm \varphi=(\varphi_1,\varphi_2):[0,1]\to W$ definita come segue:
    $$
        \begin{cases}
            x=\varphi_1(t)=x_1+t(x_2-x_1)\\
            y=\varphi_2(t)=f(x_1)+t(f(x_2)-f(x_1))
        \end{cases}
    $$
    Sia ora $h(t)=(g\circ\bm\varphi)(t) \in \C{1}([0,1],\R)$. Per il teorema del valor medio di Lagrange, $h(1)-h(0)=h'(c) \with c \in [0,1]$.
    \begin{equation}
        h(1)-h(0)=g(\bm\varphi(1))-g(\bm\varphi(0))=g(x_2,f(x_2))-g(x_1,f(x_1))=0\label{eq:h1_h0}
    \end{equation}
    Inoltre,
    \begin{align}
        h'(c)
        &=\innerproduct{\grad g(\bm\varphi(c))}{\bm\varphi'(c)}=\notag\\
        &=\frac{\partial g}{\partial x}(\bm\varphi(c))[x_2-x_1]+\frac{\partial g}{\partial y}(\bm\varphi(c))[f(x_2)-f(x_1)]\label{eq:der_h}
    \end{align}
    Siano ora $\xi=\xi(c)=x_1+c(x_2-x_1) \e \eta=\eta(c)=f(x_1)+c(f(x_2)-f(x_1))$. Unendo i risultati delle equazioni \eqref{eq:h1_h0} e \eqref{eq:der_h} e considerando che $\partial_yg(x,y)\neq 0 \ \forall (x,y) \in W$, si conclude che
    \begin{gather*}
        \frac{\partial g}{\partial x}(\xi,\eta)[x_2-x_1]+\frac{\partial g}{\partial y}(\xi,\eta)[f(x_2)-f(x_1)]=0\\
        \frac{f(x_2)-f(x_1)}{x_2-x_1}=-\frac{\partial_x g(\xi,\eta)}{\partial_y g(\xi,\eta)}
    \end{gather*}
    Se $x_2$ tende a $x_1$, $f(x_2)\to f(x_1)$ e $\eta \to f(x_1)$ perché $f$ è continua e, essendo $\xi$ intermedio fra $x_2$ e $x_1$, $\xi \to x_1$. In conclusione,
    \begin{align*}
        \frac{\dd f}{\dd x}(x_1)&=\lim_{x_2\to x_1}-\frac{\partial_x g(\xi,\eta)}{\partial_y g(\xi,\eta)}=\\
        &=\lim_{(\xi,\eta)\to(x_1,f(x_1))}-\frac{\partial_x g(\xi,\eta)}{\partial_y g(\xi,\eta)}=\\
        &=-\frac{\partial_x g(x_1,f(x_1))}{\partial_y g(x_1,f(x_1))}
    \end{align*}
\end{proof}

\begin{remark}
    Questa è alta pasticceria.
\end{remark}

\begin{theorem}
    [Caso generale con $k$ equazioni in $n$ incognite]
    Siano $A\subseteq \R^n$ aperto, $\bm g \in \C{1}(A,\R^k)$, $(\vb{x^0}, \vb{y^0})=(x_1^0,\dots,x_{n-k}^0,y_1^0,\dots,y_{k}^0) \in A \tc \bm g(\vb{x^0}, \vb{y^0})=\vb{0} \e$
    $$
        \det\frac{\partial(g_1\cdots g_k)}{\partial (y_1 \cdots y_k)}(\vb{x^0}, \vb{y^0})=\\
        \det\begin{bmatrix}
            \frac{\partial g_1}{\partial y_1} & \cdots & \frac{\partial g_1}{\partial y_k}\\
            \vdots & \ddots & \vdots \\
            \frac{\partial g_k}{\partial y_1} & \cdots & \frac{\partial g_k}{\partial y_k}
        \end{bmatrix}
        (\vb{x^0}, \vb{y^0}) \neq 0
    $$
    Allora esistono $I_{\vb{x^0}} \in \R^{n-k}$ intorno circolare aperto di $\vb{x^0}$ e $J_{\vb{y^0}} \in \R^k$ intorno circolare aperto di $\vb{y^0}$ tali che:
    \begin{enumerate}
        \item $W=I_{\vb{x^0}}\times J_{\vb{y^0}} \subseteq A$
        \item $\exists \vecf \in \C{1}(I_{\vb{x^0}}, J_{\vb{y^0}}) \tc \bm g(\vb{x}, \vb{y})=\vb{0} \iff \vb{y} = \vecf (\vb{x}) \with \vb{x} \in I_{\vb{x^0}}$
    \end{enumerate}
    Inoltre
    $$
        \frac{\partial f_i}{\partial x_j}(\vb{x})=-\frac
        {\det \frac{\partial (g_1,\dots,g_k)}{\partial y_1,\dots,y_{i-1},x_j,y_{i+1},\dots,y_k}(\vb{x},\vecf(\vb{x}))}
        {\det \frac{\partial (g_1,\dots,g_k)}{\partial y_1,\dots,y_k}(\vb{x},\vecf(\vb{x}))}
    $$
    \qed
\end{theorem}

\section{Estremanti condizionati}

\begin{definition}
    [Punto estremante condizionato]
    Siano $A\subseteq\R^n$ aperto, $\Gamma \subseteq A$ una varietà regolare $(n-k)$-dimensionale e $f: A\to\R$. Il punto $\vb{x_0} \in \Gamma$ è un punto di minimo (massimo) locale per $f$ ristretta a $\Gamma$ se $\exists \varepsilon > 0 \tc f(\vb{x}) \geq f(\vb{x_0}) \ (f(\vb{x}) \leq f(\vb{x_0})) \ \forall \vb{x} \in \Gamma \cap B_\varepsilon(\vb{x_0})$.
\end{definition}

\begin{definition}
    [Punto critico condizionato]
    Siano $A\subseteq\R^n$ aperto, $\Gamma \subseteq A$ una varietà regolare $(n-k)$-dimensionale e $f: A\to\R$. Il punto $\vb{x_0} \in \Gamma$ è punto critico condizionato per $f$ ristretta a $\Gamma$ se $\forall \bm{\hat{\nu}} \in T_{\vb{x_0}}\Gamma, \ \displaystyle\frac{\partial f}{\partial \bm{\hat{\nu}}}(\vb{x_0})=0$. 
\end{definition}

\begin{theorem}
    [di Fermat condizionato]\label{thm:fermat_cond}
    Siano $A\subseteq\R^n$ aperto, $\Gamma \subseteq A$ una varietà regolare $(n-k)$-dimensionale e $f: A\to\R$. Se $\vb{x_0} \in \Gamma$ è punto estremante condizionato per $f$ a $\Gamma$, allora $\vb{x_0}$ è un punto critico condizionato per $f$ a $\Gamma$.
\end{theorem}

\begin{proof}
    % TODO
\end{proof}

\begin{corollary}
    Sotto le ipotesi del teorema \ref{thm:fermat_cond}, $\grad f (\vb{x_0}) \in N_{\vb{x_0}}\Gamma$.
    \qed
\end{corollary}

\begin{definition}
    [Funzione di Lagrange]
    Per lo studio dei punti critici condizionati di $f: A\to \R$ alla varietà regolare $\Gamma$ definita dall'equazione $\bm g (\vb{x})=\vb{0}$ si definisce la funzione di Lagrange, o lagrangiana:
    $$
        \displaystyle \mathcal{L}(\vb{x};\lambda_1,\dots,\lambda_k)=f(\vb{x})-\sum_{j=1}^k \lambda_j g_j(\vb{x}) : A \times \R^k \to \R
    $$
\end{definition}

\begin{theorem}
    [dei moltiplicatori di Lagrange]
    Siano $f,g_1,\dots,g_k \in \C{1}(A\subseteq\R^n, \R) \with A$ aperto in $\R^n$. Se $\Gamma = \{ \vb{x} \in A: g_1 (\vb{x})=0, \dots, g_k(\vb{x})=0 \}$ è una varietà regolare e $\vb{x_0} \in \Gamma$ è punto estremante condizionato di $f$ a $\Gamma$, allora $\exists (\overline{\lambda}_1, \dots,\overline{\lambda}_k)$ tale che il punto $(\vb{x_0}; \overline{\lambda}_1,\dots,\overline{\lambda}_k)$ è punto critico libero di $\mathcal{L}$.
\end{theorem}

\begin{proof}
    % TODO
\end{proof}

\begin{theorem}
    [Condizioni con funzione di Lagrange]
    Siano $f,g_1,\dots,g_k \in \C{2}(A\subseteq\R^n, \R) \with A$ aperto e $\Gamma = \{ \vb{x} \in A: g_1 (\vb{x})=0, \dots, g_k(\vb{x})=0 \}$ una varietà regolare.
    \begin{enumerate}
        \item Se $\vb{x_0} \in \Gamma$ è punto di minimo condizionato di $f$ a $\Gamma$, allora $\exists \overline{\lambda}_1,\dots,\overline{\lambda}_k \in \R \tc$
        \begin{enumerate}[a.]
            \item $\grad f(\vb{x_0})=\displaystyle \sum_{j=1}^k \overline{\lambda}_j \grad g_j(\vb{x_0})$
            \item $\ip{\vb{h}}{\left( H_f(\vb{x_0})-\sum_{j=1}^k\overline{\lambda}_j H_{g_j}(\vb{x_0}) \right)\vb{h}} \geq 0 \ \forall \vb{h} \in T_{\vb{x_0}}\Gamma$, ovvero la restrizione della forma quadratica associata alla matrice $\left( H_f(\vb{x_0})-\sum_{j=1}^k\overline{\lambda}_j H_{g_j}(\vb{x_0}) \right)$ allo spazio $T_{\vb{x_0}}\Gamma$ è definita positiva o semidefinita positiva 
        \end{enumerate}
        \item Se $\vb{x_0} \in \Gamma$ soddisfa per una qualche scelta di $(\overline{\lambda}_1,\dots,\overline{\lambda}_k)$:
        \begin{enumerate}[a.]
            \item $\grad f(\vb{x_0})=\displaystyle\sum_{j=1}^k\overline{\lambda}_j \grad g_j (\vb{x_0})$
            \item $\ip{\vb{h}}{\left( H_f(\vb{x_0})-\sum_{j=1}^k\overline{\lambda}_j H_{g_j}(\vb{x_0}) \right)\vb{h}} > 0 \ \forall \vb{h} \in T_{\vb{x_0}}\Gamma$
        \end{enumerate}
        allora $\vb{x_0}$ è punto di minimo condizionato di $f$ a $\Gamma$.
    \end{enumerate}
\end{theorem}

\begin{proof}
    [Dimostrazione del punto 2.] % TODO
\end{proof}

\chapter{Teoria della misura e dell'integrazione}\label{chap:peano_jordan}

\section{Misura di Peano-Jordan}

\begin{definition}
    [Intervallo semi-aperto superiormente]
    Si definisce intervallo semi-aperto superiormente di $\R^n$ il prodotto cartesiano di $n$ intervalli chiusi inferiormente e aperti superiormente:
    $$
        I=[a_1,b_1)\times\cdots\times[a_n,b_n) \with a_i \leq b_i \ \forall i \in [n]
    $$
    La misura elementare di $I$ è
    $$
        \mu_n(I)=\prod_{j=1}^n(b_j-a_j)
    $$
\end{definition}

\begin{definition}
    [Pluri-intervallo]
    Siano $I_1,\dots,I_k$ intervalli semi-aperti superiormente a due a due disgiunti. Si definisce pluri-intervallo l'unione $P=\bigcup\limits_{j=1}^kI_j$. La misura di $P$ è $\mu_n(P)=\sum\limits_{j=1}^k\mu_n(I_j)$. 
    L'insieme dei pluri-intervalli di $\R^n$ viene indicato con $\mathcal{P}$.
\end{definition}

\begin{lemma}
    Valgono le seguenti proposizioni:
    \begin{itemize}
        \item Se $P_1,\dots,P_k \in \mathcal{P}$, allora $\bigcup\limits_{j=1}^k P_j, \bigcap\limits_{j=1}^kP_j \in \mathcal{P}$
        \item Se $P_1,P_2 \in \mathcal{P}$, allora $P_2 \setminus P_1 \in \mathcal{P}$
        \qed
    \end{itemize}
\end{lemma}

\begin{theorem}
    [Caratterizzazione della misura in $\mathcal{P}$]
    Per la misura in $\mathcal{P}$ valgono le seguenti proprietà:
    \begin{enumerate}
        \item (Additività finita) Se $P_1,\dots,P_k \in \mathcal{P}$ sono a due a due disgiunti, allora
        $$
            \mu_n\left( \bigcup_{j=1}^k P_j \right) = \sum_{j=1}^k \mu_n(P_j)
        $$
        \item (Monotonia) Se $P, Q \in \mathcal{P} \e P \subseteq Q \then \mu_n(P)\leq \mu_n(Q)$
        \item (Modularità) Se $P,Q \in \mathcal{P}$, allora
        $$
            \mu_n(P\cup Q) + \mu_n (P \cap Q) = \mu_n (P) + \mu_n (Q)
        $$
        \item (Sottrattività) Se $P,Q \in \mathcal{P} \then \mu_n(P\setminus Q)=\mu_n(P)-\mu_n(P\cap Q)$. In particolare, se $Q \subseteq P$, allora $\mu_n(P\setminus Q) = \mu_n (P)-\mu_n(Q)$
        \qed
    \end{enumerate}
\end{theorem}

\begin{corollary}
    Se $P_1,\dots,P_k \in \mathcal{P}$, allora $\mu_n\left(\bigcup\limits_{j=1}^kP_j\right) \leq \sum\limits_{j=1}^k\mu_n(P_j)$. Ovvero, vale la proprietà di subadditività finita per $\mu_n$ su $\mathcal{P}$.
    \qed
\end{corollary}

\begin{lemma}
    Se $P \in \mathcal{P}$, allora $\partial P \in \mathcal{P} \e \mu_n(\partial P)=0$.
    \qed
\end{lemma}

\begin{corollary}
    Se $P \in \mathcal{P}$, allora $\mathring P, \overline{P}$ sono misurabili e $\mu_n(\mathring P)=\mu_n(\overline{P})=\mu_n(P)$.
    \qed
\end{corollary}

\begin{definition}
    [Misura interna ed esterna]
    Si definiscono gli insiemi dei pluri-intervalli interni all'insieme $X$, $\mathcal{P}^{(i)}$, e dei pluri-intervalli esterni all'insieme $X$, $\mathcal{P}^{(e)}$:
    \begin{align*}
        \mathcal{P}^{(i)}&=\{P\in \mathcal{P}:P\subseteq X\}\\
        \mathcal{P}^{(e)}&=\{Q \in \mathcal{P}: Q \supseteq X \}
    \end{align*}
    La misura interna di $X$ è $
        \mu_n^{(i)}(X)=\sup\limits_{P\in \mathcal{P}^{(i)}}\mu_n(P)
    $
    e la misura esterna di $X$ è $
        \mu_n^{(e)}(X)=\inf\limits_{Q \in \mathcal{P}^{(e)}}\mu_n(Q)
    $
\end{definition}

\begin{lemma}
    $\mu_n^{(i)}(X), \mu_n^{(e)}(X) \in \R_{\geq 0} \e \mu_n^{(i)}(X)\leq\mu_n^{(e)}(X)$.
    \qed
\end{lemma}

\begin{definition}
    [Insieme limitato misurabile secondo Peano-Jordan]
    Sia $X \subseteq \R^n$ un insieme limitato. Si dice che $X$ è misurabile secondo Peano-Jordan se $\mu_n^{(i)}(X)=\mu_n^{(e)}(X)$ e in tal caso si dice misura $n$-dimensionale di $X$ tale valore.
    L'insieme degli insiemi limitati misurabili secondo Peano-Jordan si indica con $\J_b(\R^n)$.
\end{definition}

\begin{lemma} Valgono le seguenti proposizioni:
    \begin{enumerate}
        \item $\mathcal{P} \subseteq \J_b(\R^n)$. Tutti i pluri-intervalli sono misurabili secondo Peano-Jordan e la misura coincide con quella definita in modo elementare.
        \item Se $X_1,\dots,X_k \in \J_b(\R^n)$, allora $\bigcup\limits_{j=1}^k X_j, \bigcap\limits_{j=1}^kX_j \in \J_b(\R^n)$
        \item Se $X,Y \in \J_b(\R^n)\then X \setminus Y \in \J_b(\R^n)$
        \qed
    \end{enumerate}
\end{lemma}

\begin{theorem}
    [Caratterizzazione della misurabilità]
    Sia $X\subseteq \R^n$ limitato. Allora:
    \begin{enumerate}
        \item $X\in \J_b(\R^n) \iff \forall \varepsilon > 0 \ \exists P,Q \in \mathcal{P} \tc P \subseteq X \subseteq Q \e \mu_n(Q)-\mu_n(P) < \varepsilon$
        \item $X \in \J_b(\R^n) \iff \forall \varepsilon > 0 \ \exists Y,Z \in \J_b(\R^n) \tc Y \subseteq X \subseteq Z \e \mu_n(Z)-\mu_n(Y) < \varepsilon$
        \item $X \in \J_b(\R^n)\iff \partial X \in \J_b(\R^n) \e \mu_n(\partial X)=0$
        \qed
    \end{enumerate}
\end{theorem}

\begin{corollary}
    Se $X \in \J_b(\R^n)$, allora $\overline{X},\mathring X \in \J_b(\R^n) \e \mu_n(X)=\mu_n(\overline{X})=\mu_n(\mathring X)$.
    \qed
\end{corollary}

\begin{theorem}
    [Proprietà di $\J_b(\R^n)$]
    Valgono le seguenti proprietà per la misura in $\J_b(\R^n)$:
    \begin{enumerate}
        \item (Additività finita) Se $X_1,\dots,X_k \in \J_b(\R)^n$ sono a due a due disgiunti, allora
        $$
            \mu_n\left(\displaystyle \bigcup_{j=1}^kX_j\right) =   \displaystyle\sum_{j=1}^k\mu_n(X_j)
        $$
        \item (Modularità) Se $X,Y \in \J_b(\R^n)$, allora $\mu_n(X\cup Y)+\mu_n(X\cap Y)=\mu_n(X)+\mu_n(Y)$
        \item (Monotonia) Se $X,Y \in \J_b(\R^n) \e X \subseteq Y$, allora $\mu_n(X) \leq \mu_n(Y)$
        \item (Sottrattività) Se $X,Y \in \J_b(\R^n)$, allora $\mu_n(X\setminus Y)=\mu_n(X)-\mu_n(Y)$. In particolare, se $Y \subseteq X \then \mu_n (X \setminus Y)=\mu_n(X)-\mu_n(Y)$
        \qed
    \end{enumerate}
\end{theorem}

\begin{theorem}
    [Caratterizzazione degli insiemi di misura nulla]\leavevmode
    \begin{enumerate}
        \item $\mathring X = \varnothing \iff \mu_n^{(i)}(X)=0$
        \item $X \in \J_b(\R^n) \e \mu_n(X)=0 \iff \forall \varepsilon >0 \ \exists P \in \mathcal{P} \tc P \supseteq X \e \mu_n(P)< \varepsilon$
        \item $X \in \J_b(\R^n) \e \mu_n(X)=0 \iff \mathring X = \varnothing \e X \in \J_b(\R^n)$
        \qed
    \end{enumerate}
\end{theorem}

\begin{definition}
    [Insieme misurabile secondo Peano-Jordan]
    Sia $X\subseteq \R^n$. $X$ è misurabile secondo Peano-Jordan, ovvero $X \in \J(\R^n)$, se $\forall Y \in \J_b(\R^n), \ Y \cap X \in \J_b(\R^n)$. In tal caso si definisce $\mu_n(X)=\sup\{\mu_n(X\cap Y):Y\in \J_b(\R^n)\}$.
\end{definition}

\begin{remark}
    Questa definizione allarga la nozione di misura agli insiemi non limitati. Un insieme misurabile in $\J_b(\R^n)$ ha la stessa misura se misurato in $\J(\R^n)$.
\end{remark}

\begin{theorem}
    Se $X \in \J(\R^n) \e \{X_k\}_{k \in \N}$ è una successione crescente di elementi in $\J_b(\R^n)$, ovvero:
    \begin{enumerate}
        \item $\forall k \in \N, \ X_k \in \J_b(\R^n)$
        \item $X_k \subseteq X_{k+1}$
        \item $\bigcup\limits_{k=1}^{\infty}X_k = \R^n$,
    \end{enumerate}
    allora $\mu_n(X)=\lim\limits_{k\to + \infty}\mu_n(X\cap X_k)$.
    \qed
\end{theorem}

\begin{theorem}
    $\mu_n: \J(\R^n) \to [0, +\infty]$ e inoltre:
    \begin{enumerate}
        \item $\varnothing \in \J(\R^n)$
        \item Se $X \in \J(\R^n)$, allora $\R^n \setminus X \in \J(\R^n)$
        \item Se $X, Y \in \J(\R^n)$, allora $X \cup Y, X \cap Y, X \setminus Y \in \J(\R^n)$
        \qed
    \end{enumerate}
\end{theorem}

\begin{theorem}
    [Proprietà di $\J(\R^n)$]\label{thm:prop_J}
    $\J(\R^n)$ gode delle seguenti proprietà:
    \begin{enumerate}
        \item (Additività finita) Se $X_1,\dots,X_k \in \J(\R^n), \ X_i \cap X_j = \varnothing \ \forall i, j \in [k]$, allora $\bigcup\limits_{j=1}^kX_j \in \J(\R^n)$ e
        $$
            \mu_n\left(\bigcup_{j=1}^kX_j\right) = \sum_{j=1}^k\mu_n(X_j)
        $$
        \item (Monotonia) Se $X, Y \in \J(\R^n) \e X \subseteq Y$, allora $\mu_n(X) \leq \mu_n(Y)$
        \item (Modularità) Se $X, Y \in \J(\R^n)$, allora
        $$
            \mu_n(X \cup Y) + \mu_n (X \cap Y) = \mu_n(X) + \mu_n(Y)
        $$
        \item (Sottrattività) Se $X, Y \in \J(\R^n) \e \mu_n(Y) < + \infty$, allora
        $$
            \mu_n(X \setminus Y) = \mu_n(X) - \mu_n(X \cap Y)
        $$
        \qed
    \end{enumerate}
\end{theorem}

\begin{remark}
    Nel punto \textit{3.} del teorema \ref{thm:prop_J} è importante non spostare termini dell'espressione da una parte all'altra dell'uguaglianza. Essendo $X \e Y$ insiemi non necessariamente limitati, si rischierebbe di avere una sottrazione $\infty - \infty$.
\end{remark}

\begin{lemma}
    $X \in \J(\R^n) \iff \partial X \in \J(\R^n) \e \mu_n(\partial X)=0$.
    \qed
\end{lemma}

\section{Integrazione secondo Riemann}

\begin{definition}
    [Funzione non negativa integrabile secondo Riemann]
    Siano $A \subseteq \R^n, \ A \in \J(\R^n), \ f: A \to \R$ limitata e tale che $f \geq 0$. Si definisce sottografico di $f$ l'insieme $R(f)=\{(\vb{x}, y)\in A \times \R_{\geq 0} : 0 \leq y \leq f(\vb{x})\}$.
    $f$ si dice integrabile secondo Riemann se $R(f) \in \J(\R^{n+1})$ e in tal caso
    $$
        \idotsint_A f(x_1,\dots,x_n)\dd x_1\cdots \dd x_n=\mu_{n+1}(R(f))
    $$
    $f$ è detta sommabile se $f$ è integrabile e $\idotsint_Af < +\infty$.
\end{definition}

\begin{definition}
    [Parte positiva e parte negativa]\label{def:ppos_pneg}
    Se $f:A \subseteq \R^n \to \R$, si definiscono:
    \begin{itemize}
        \item Parte positiva di $f$, $f_+(\vb{x})=\max\{0,f(\vb{x})\}$
        \item Parte negativa di $f$, $f_-(\vb{x})=\max\{0,-f(\vb{x})\}$
    \end{itemize}
    In base a questa definizione, $f(\vb{x})=f_+(\vb{x})-f_-(\vb{x})$ e $\abs{f(\vb{x})}=f_+(\vb{x})+f_-(\vb{x})$.
\end{definition}

\begin{definition}
    [Funzione integrabile secondo Riemann]
    Siano $A \in \J(\R^n) \e f: A \to \R$ limitata. Se $f_+ \e f_-$ sono integrabili e almeno uno fra $\idotsint_A f_+, \idotsint_A f_-$ è un valore finito, allora $f$ si dice integrabile secondo Riemann e
    $$
        \idotsint_Af(x_1,\dots,x_n)\dd x_1\cdots \dd x_n = \idotsint_A f_+(x_1,\dots,x_n)\dd x_1\cdots \dd x_n - \idotsint_A f_-(x_1,\dots,x_n)\dd x_1\cdots \dd x_n
    $$
    Inoltre, se $f_+ \e f_-$ sono sommabili, allora $f$ è sommabile.
\end{definition}

\begin{theorem}
    Sia $f: A \to \R$ integrabile in $A \in \J(\R^n)$. Allora
    $$
        \abs{\idotsint_A f(x_1,\dots,x_n)\dd x_1\cdots \dd x_n} \leq \idotsint_A\abs{f(x_1,\dots,x_n)}\dd x_1\cdots \dd x_n
    $$
    \qed
\end{theorem}

\begin{theorem}
    [Proprietà delle funzioni sommabili]
    Siano $A \in \J(\R^n), \ f,g: A \to \R$ limitate e sommabili secondo Riemann. Valgono le seguenti proprietà:
    \begin{enumerate}
        \item (Linearità) $\forall a,b \in \R, \ af + bg$ è sommabile e $\idotsint_A(af+bg)=a\idotsint_A f+b\idotsint_A g$
        \item (Monotonia) Se $\forall \vb{x} \in A \ f(\vb{x}) \leq g(\vb{x})$, allora $\idotsint_A f \leq \idotsint_A g$
        \item (Additività) Se $\mu_n(A) < + \infty \e A_1,A_2 \subseteq A \tc A_1 \cup A_2 = A \e A_1,A_2 \in \J(\R^n) \e \mu_n(A_1 \cap A_2)=0$, allora $\idotsint_A f= \idotsint_{A_1} f+ \idotsint_{A_2} f$
        \qed
    \end{enumerate}
\end{theorem}

\begin{theorem}
    [della media integrale]
    Siano $A \subseteq \R^n, \ A \in \J(\R^n), \ \mu_n(A) < +\infty \e f: A \to \R$ limitata e sommabile. Allora:
    $$
        \inf_A f \leq \frac{\idotsint_A f}{\mu_n(A)}\leq \sup_A f
    $$
    \qed
\end{theorem}

\begin{corollary}
    Sotto le ipotesi del teorema, se $A \subseteq \R^n$ è compatto e connesso e $f: A \to \R$ è continua, allora $\exists \vb{c} \in A \tc \displaystyle\frac{\idotsint_A f}{\mu_n(A)}=f(\vb{c})$.
    \qed
\end{corollary}

\subsection{Integrali doppi}

\begin{theorem}
    [di riduzione sui rettangoli]
    Sia $K=[a,b]\times[c,d] \e f\in \C{1}(K,\R)$, allora:
    \begin{enumerate}
        \item $G:[c,d]\to \R$ definita come $G(y)=\int_a^bf(x,y)\dd x$ è continua e $\iint_Kf(x,y)\dd x\dd y=\int_c^dG(y)\dd y=\int_c^d\dd y(\int_a^b f(x,y)\dd x)$
        \item $F:[a,b]\to \R$ definita come $F(x)=\int_c^df(x,y)\dd y$ è continua e $\iint_Kf(x,y)\dd x\dd y=\int_a^bF(x)\dd x=\int_a^b\dd x(\int_c^d f(x,y)\dd y)$
        \item Se $f(x,y)=g(x)h(y)$, allora
        $$
            \iint_Kf(x,y)\dd x\dd y=\iint_Kg(x)h(y)\dd x\dd y=\left(\int_a^bg(x)\dd x\right)\left(\int_c^dh(y)\dd y\right)
        $$
        \qed
    \end{enumerate}
\end{theorem}

\begin{definition}
    [Dominio normale rispetto a un asse]\leavevmode
    \begin{enumerate}[a.]
        \item Siano $\varphi, \psi \in \C{0}([c,d],\R) \tc \varphi(y)\leq \psi(y)\ \forall y \in [c,d]$. Allora
        $$
            A=\{(x,y)\in \R\times[c,d]:\varphi(y)\leq x \leq \psi(y)\}
        $$
        è un dominio normale rispetto all'asse $x$.
        \item Siano $g, h \in \C{0}([a,b],\R) \tc g(x)\leq h(x)\ \forall x \in [a,b]$. Allora
        $$
            B=\{(x,y)\in [a,b]\times\R:g(x)\leq y \leq h(x)\}
        $$
        è un dominio normale rispetto all'asse $y$. 
    \end{enumerate}
\end{definition}

\begin{theorem}
    [di riduzione degli integrali doppi su domini normali]\leavevmode
    \begin{enumerate}[a.]
        \item Siano $\varphi, \psi \in \C{0}([c,d],\R), \ A=\{(x,y)\in \R\times[c,d]:\varphi(y)\leq x \leq \psi(y)\} \e f \in \C{0}(A,\R)$. Allora
        $$
            \iint_Af(x,y)\dd x\dd y = \int_c^d\dd y\left(\int_{\varphi(y)}^{\psi(y)}f(x,y)\dd x\right)
        $$
        \item Siano $g, h \in \C{0}([a,b],\R), \ A=\{(x,y)\in \R\times[a,b]:g(x)\leq y \leq h(x)\} \e f \in \C{0}(A,\R)$. Allora
        $$
            \iint_Af(x,y)\dd x\dd y = \int_a^b\dd x\left(\int_{g(x)}^{h(x)}f(x,y)\dd y\right)
        $$
        \qed
    \end{enumerate}
\end{theorem}

\subsection{Integrali tripli}

\begin{definition}
    [Solido di Cavalieri]
    Sia $K \subseteq \R^3$ compatto e misurabile. Se esiste un asse $\hat{\lambda}$ tale che
    \begin{enumerate}[a.]
        \item $\forall \lambda \in [a,b], \ \sez_\lambda(K)$ è un insieme misurabile e
        \item $\forall \lambda < a \e \lambda > b, \ \sez_\lambda (K) = \varnothing$,
    \end{enumerate}
    allora $K$ è detto solido di Cavalieri.
\end{definition}

\begin{axiom}
    [di Cavalieri]
    Se $V$ e $W$ sono solidi di Cavalieri rispetto allo stesso asse $\hat{\lambda}$, $\mu_2(\sez_\lambda(V))\leq\mu_2(\sez_\lambda(W)) \ \forall \lambda \in [a,b]$ e $\mu_2(\sez_\lambda(V))=\mu_2(\sez_\lambda(W))=0 \ \forall \lambda \notin [a,b]$, allora $\mu_3(V)\leq \mu_3(W)$. Inoltre, se $\mu_2(\sez_\lambda(V))=\mu_2(\sez_\lambda(W))$, allora $\mu_3(V)=\mu_3(W)$.
\end{axiom}

\begin{theorem}
    [di Cavalieri]
    Sia $K \in \J_b(\R^3)$ compatto un solido di Cavalieri rispetto all'asse $\hat{z}$. Allora
    $$
        \mu_3(K)=\int_a^b\dd z\mu_2(\sez_z(K))=\int_a^b\dd z\left(\iint_{\sez_z(K)}\dd x\dd y\right)    
    $$
    Inoltre, se $f \in \C{0}(K,\R)$, allora
    $$
        \iiint_Kf(x,y,z)\dd x\dd y\dd z=\int_a^b\dd z\left(\iint_{\sez_z(K)}f(x,y,z)\dd x\dd y\right)
    $$
    \qed
\end{theorem}

\begin{definition}
    [Domino normale rispetto a un asse in $\R^3$]\label{def:dom_r3}
    Siano $K\subseteq \R^2$ compatto e misurabile, $\varphi,\psi \in \C{0}(K,\R) \tc \varphi(x,y)\leq \psi(x,y) \ \forall (x,y) \in K$. Si dice dominio normale rispetto all'asse $\hat{z}$ l'insieme $A=\{(x,y,z)\in K \times \R : \varphi(x,y)\leq z \leq \psi(x,y)\}$.
\end{definition}

\begin{theorem}
    [di riduzione degli integrali tripli su domini normali]
    Siano $\varphi, \psi, K, A$ come nella definizione \ref{def:dom_r3}. Allora
    $$
        \mu_3(A)=\iint_K\dd x\dd y(\psi(x,y)-\varphi(x,y))
    $$
    Inoltre, se $f \in \C{0}(A,\R)$,
    $$
        \iiint_Af(x,y,z)\dd x\dd y\dd z = \iint_K\dd x\dd y\left(\int_{\varphi(x,y)}^{\psi(x,y)}\dd zf(x,y,z)\right)
    $$
\end{theorem}

\subsection{Cambiamento di variabile nell'integrale multiplo}

Per comprendere il significato del cambiamento di variabile nell'integrale multiplo, è opportuno ricordare le principali caratteristiche delle applicazioni lineari.

\begin{definition}
    [Applicazione lineare]
    Sia $\vb{A}:\R^n\to\R^n$ un'applicazione. Essa è detta lineare se possiede le seguenti proprietà:
    \begin{enumerate}
        \item $\vb{A}(\vb{u}+\vb{v})=\vb{A}(\vb{u})+\vb{A}(\vb{v}) \ \forall \vb{u},\vb{v}\in \R^n$
        \item $\vb{A}(\lambda\vb{v})=\lambda \vb{A}(\vb{v}) \ \forall \lambda \in \R, \ \forall \vb{v} \in \R^n$ 
    \end{enumerate}
    Inoltre, detta $M_A \in \mathcal{M}_{n\times n}$ la matrice associata all'applicazione $\vb{A}$, il valore assoluto del suo determinante rappresenta il fattore di cui viene riscalata la misura di un qualsiasi $Q\subseteq \R^n$ attraverso l'applicazione.
    $$\mu_n(\vb{A}(Q))=\abs{\det M_A}\mu_n(Q)$$
\end{definition}

\begin{remark}
    Si noti che nel caso lineare $\det M_A \neq 0 \then A$ biunivoca da $\R^n$ in $\R^n$.
\end{remark}

Nel caso non lineare, quindi con un'applicazione $\bm\varphi$ qualsiasi, non esiste un fattore di scala valido per tutto lo spazio, ma localmente è possibile approssimare la trasformazione con una trasformazione lineare. Il fattore di scala infinitesimo è rappresentato dal determinante della matrice jacobiana dell'applicazione. Inoltre non vale l'osservazione appena fatta ed è quindi necessario richiedere sia che $\bm\varphi$ sia iniettiva sia che abbia $\det J_{\bm\varphi} \neq 0$.

\begin{theorem}
    Siano $A \subseteq \R^n$ aperto, $\bm\varphi \in \C{1}(A,\R^n), \bm\varphi \inj \e \det J_{\bm\varphi} \neq 0$ in $A$. Allora, se $K \subseteq A$ è compatto e misurabile, anche $\bm\varphi(K)$ è compatto e misurabile e vale
    $$
        \mu_n(\bm\varphi(K))=\idotsint_K \abs{\det J_{\bm\varphi}(u_1,\dots,u_n)}\dd u_1 \cdots \dd u_n
    $$
    Inoltre, se $f \in \C{0}(\bm\varphi(K),\R)$, allora
    $$
        \idotsint_{\bm\varphi(K)}f(x_1,\dots,x_n)\dd x_1 \cdots \dd x_n=\idotsint_K (f \circ \bm\varphi)(u_1,\dots,u_n)\abs{\det J_{\bm\varphi}(u_1,\dots,u_n)}\dd u_1\cdots \dd u_n
    $$
    \qed
\end{theorem}

\begin{remark}
    Il teorema del cambiamento di variabile continua a valere nel caso di perdita di iniettività o di $\det J_{\bm\varphi}\neq 0$ su insiemi di misura nulla che sono trasformati in insiemi di misura nulla.
\end{remark}

\paragraph{Coordinate polari nel piano}
% TODO

\paragraph{Coordinate sferiche in $\R^3$}
% TODO

\paragraph{Coordinate cilindriche in $\R^3$}
% TODO

\chapter{Curve e lavoro}\label{chap:curves}

\section{Curve in forma parametrica}

\begin{definition}
	[Curva]
	Si definisce curva in $\R^n$ una funzione $\vb{r}:[a,b]\to \R^n$ continua. $\vb{r}([a,b])$ è detto sostegno della curva.
\end{definition}

\begin{remark}
	La richiesta della continuità non è sufficiente perché $\vb{r}([a,b])$ sia un sottoinsieme di dimensione $1$ di $\R^n$. Esempio: la curva di Peano.
\end{remark}

\begin{definition}
	[Parametrizzazione regolare]
	Si dice parametrizzazione regolare della curva una funzione $\vb{r}:[a,b]\to \R^n \in \C{1} \tc \vb{r}'(t)\neq\vb{0} \ \forall t \in [a,b]$.
\end{definition}

\begin{definition}
	[Parametrizzazione semplice aperta]
	Si dice parametrizzazione semplice aperta una funzione $\vb{r}: [a,b]\bijarrow\vb{r}([a,b]) \in \C{1}$. Si noti che $\vb{r}$ è un omeomorfismo.
\end{definition}

\begin{definition}
	[Parametrizzazione semplice chiusa]
	Si dice parametrizzazione semplice chiusa una funzione $\vb{r}: (a,b)\bijarrow \vb{r}((a,b)) \e \vb{r}(a)=\vb{r}(b)$.
\end{definition}

\begin{theorem}
	[Invarianza per cambio di parametrizzazione]
	Siano $\vb{r} \in \C{1}([a,b],\gamma) \e \varphi:[\alpha,\beta]\bijarrow[a,b]\\\in \C{1} \with \varphi^{-1} \in \C{1}$ (ovvero, $\varphi$ è un diffeomorfismo). Sia $\bm\rho = \vb{r} \circ \varphi: [\alpha, \beta] \to \R^n$. Allora
	\begin{enumerate}
		\item $\bm\rho([\alpha,\beta])=\gamma$, ovvero il sostegno della curva è invariato
		\item $\bm\rho$ è regolare se e solo se $\vb{r}$ è regolare
		\item $\bm\rho$ è semplice se e solo se $\vb{r}$ è semplice
		\qed
	\end{enumerate}
\end{theorem}

\begin{remark}
	Quando si trattano curve in forma parametrica si considera la classe di equivalenza delle parametrizzazioni con lo stesso sostegno definita come segue: $\vb{r}\sim\bm\rho \iff \exists \varphi \in \C{1}([\alpha,\beta],[a,b])$ diffeomorfismo tale che $\bm\rho=\vb{r}\circ\varphi$.
\end{remark}

\begin{definition}
	[Curva orientabile]\label{def:curve_or}
	$\gamma \subseteq \R^n$ connesso e compatto è orientabile se $\exists \vb{T}: \gamma \to \R^n$ continua tale che $\forall \vb{x} \in \gamma \ \vb{T}(\vb{x})\in T_{\vb{x}}\gamma \e \norm{\vb{T}(\vb{x})}=1$. In altre parole, è necessario che esista un campo continuo di versori tangenti alla curva. Si definisce orientamento indotto dalla parametrizzazione $\vb{r}$ il campo di versori
	$$
		\vb{T}(\vb{x})=\frac{\vb{r}'(r^{-1}(\vb{x}))}{\norm{\vb{r}'(r^{-1}(\vb{x}))}} \in T_{\vb{x}}\gamma
	$$
\end{definition}

\begin{lemma}
	Ogni curva semplice e regolare è orientabile. La semplicità garantisce che $\gamma$ abbia esattamente due orientamenti.\qed
\end{lemma}

\begin{theorem}
	[Effetto del cambio di parametrizzazione sull'orientamento]
	Sia $\gamma$ orientabile e sia $\vb{T}(\vb{x})$ come nella Definizione \ref{def:curve_or}. Sia $\varphi:[\alpha,\beta]\to[a,b]$ un diffeomorfismo $\C{1}$ e sia $\bm\rho=\vb{r}\circ\varphi$ una nuova parametrizzazione di $\gamma$. Vi sono due casi:
	\begin{enumerate}[a.]
		\item $\varphi$ è un diffeomorfismo crescente, allora l'orientamento indotto da $\bm\rho$ è lo stesso di $\vb{r}$
		\item $\varphi$ è un diffeomorfismo decrescente, allora l'orientamento indotto da $\bm\rho$ è opposto a quello indotto da $\vb{r}$
	\end{enumerate}
\end{theorem}

\begin{proof} Si nota immediatamente che il segno dell'orientamento dipende esclusivamente dal segno della derivata di $\varphi$.
	$$
		\bm\rho'(\tau)=\vb{r}'(\varphi(\tau))\frac{\dd \varphi}{\dd \tau}(\tau)
	$$
\end{proof}

\begin{definition}
	[Parametrizzazione regolare a tratti]
	Si definisce parametrizzazione regolare a tratti $\vb{r}:[a,b]\to \R^n$ continua se $\exists a=t_1<\cdots<t_k=b \tc \vb{r}:[t_i, t_{i+1}]\bijarrow \vb{r}([t_i,t_{i+1}]) \in \C{1} \e \vb{r}'(t)\neq \vb{0} \ \forall t \in (t_i,t_{i+1}) \ \forall i \in [k-1]$.
\end{definition}

\begin{definition}
	[Parametrizzazione regolare a tratti orientabile]
	Si definisce parametrizzazione regolare a tratti orientabile $\vb{r}: [a,b]\to\R^n$ continua se $\exists a=t_1<\cdots<t_k=b \tc \vb{r}:[a,b]\setminus \{t_1,\dots,t_k\} \injarrow \gamma, \ \vb{r}:[t_i,t_{i+1}]\bijarrow \vb{r}([t_i,t_{i+1}]) \in \C{1} \e \vb{r}'(t)\neq \vb{0} \ \forall t \in (t_i,t_{i+1}) \ \forall i \in [k-1]$.
\end{definition}

\begin{definition}
	[Punto di arresto]
	Sia $\vb{r} :[a,b] \to \R^n\in \C{1}$. $t \in [a,b]$ è punto di arresto se $\vb{r}'(t) = \vb{0}$.
\end{definition}

\section{Integrali curvilinei}

\begin{definition}
	[Lunghezza di una curva]
	Siano $\gamma$ una curva e sia $\mathcal{D}_P=\vb{x}_0,\dots,\vb{x}_k\in \gamma$ una suddivisione di $\gamma$. Si definisce lunghezza della curva $\gamma$ $L_\gamma = \sup\limits_{P}\sum\limits_{j=1}^{k}\norm{\vb{x}_j-\vb{x}_{j-1}}$.
\end{definition}

\begin{theorem}
	[Lunghezza di una curva regolare]
	Se $\gamma$ è una curva regolare, allora la lunghezza di $\gamma$ è
	$$
		L_\gamma = \int_a^b\norm{\vb{r}'(t)}\dd t
	$$
	dove $\vb{r}: [a,b]\suarrow \gamma$ è una parametrizzazione regolare.
\end{theorem}

\begin{proof}
	Si dimostra solo la disuguaglianza $L_P \leq \int\limits_{a}^{b}\norm{\vb{r}'(t)}\dd t \ \forall P$. Siano $a=t_0<t_1<\cdots<t_k=b$ i valori nell'intervallo $[a,b]$ corrispondenti ai punti della suddivisione $\mathcal{D}_P$.
	\begin{gather*}
		\vb{x}_j-\vb{x}_{j-1}=\vb{r}(t_j)-\vb{r}(t_{j-1})=\int_{t_{j-1}}^{t_j}\vb{r}'(t)\dd t\\
		\then L_P=\sum_{j=1}^{k}\norm{\vb{x}_j-\vb{x}_{j-1}}=\sum_{j=1}^{k}\norm{\int_{t_{j-1}}^{t_j}\vb{r}'(t)\dd t}\leq\\
		\leq\sum_{j=1}^{k}\int_{t_{j-1}}^{t_j}\norm{\vb{r}'(t)}\dd t=\int_{a}^{b}\norm{\vb{r}'(t)}\dd t
	\end{gather*}
\end{proof}

\begin{prop}\label{prop:length_invariance}
	La lunghezza di una curva regolare non dipende dalla parametrizzazione.
\end{prop}

\begin{proof}
	Siano $\vb{r}\in\C{1}([a,b],\R^n)$ una parametrizzazione regolare di $\gamma$, $\varphi:[\alpha,\beta]\to[a,b]$ un diffeomorfismo $\C{1}$ e $\bm{\rho}=\vb{r}\circ\varphi$.
	\begin{equation}\label{eq:length_curve}
		\int_{\alpha}^{\beta}\norm{\frac{\dd \bm\rho}{\dd \tau}(\tau)}\dd \tau
		=\int_{\alpha}^{\beta}\abs{\frac{\dd \varphi}{\dd \tau}(\tau)}\norm{\frac{\dd \vb{r}}{\dd t}(\varphi(\tau))}\dd \tau
	\end{equation}
	Si presentano due casi:
	\begin{enumerate}[a.]
		\item Se $\varphi$ è un diffeomorfismo crescente, la \eqref{eq:length_curve} diventa come segue:
		$$
			\int_{\alpha}^{\beta}\frac{\dd \varphi}{\dd \tau}(\tau)\norm{\frac{\dd \vb{r}}{\dd t}(\varphi(\tau))}\dd \tau=\int_{a}^{b}\norm{\frac{\dd \vb{r}}{\dd t}(t)}\dd t=L_\gamma
		$$
		\item Se $\varphi$ è un diffeomorfismo crescente, il valore assoluto cambia il segno della sua derivata e la \eqref{eq:length_curve} diventa come segue:
		$$
		\int_{\alpha}^{\beta}-\frac{\dd \varphi}{\dd \tau}(\tau)\norm{\frac{\dd \vb{r}}{\dd t}(\varphi(\tau))}\dd \tau=\int_{a}^{b}\norm{\frac{\dd \vb{r}}{\dd t}(t)}\dd t=L_\gamma
		$$
	\end{enumerate}
	Dove in entrambi i casi è stato usato il teorema del cambiamento di variabile nell'integrale.
\end{proof}

\begin{prop}
	Tutte le curve regolari a tratti sono rettificabili, ovvero hanno lunghezza finita.
	\qed
\end{prop}

\begin{definition}
	[Integrale di linea di una funzione]
	Siano $f: A\subseteq\R^n \to \R$ continua con $A$ aperto, $\gamma \subseteq A$ una curva regolare (o regolare a tratti) e $\vb{r}:[a,b]\suarrow \gamma \in \C{1}$ una sua parametrizzazione, allora
	$$
		\int_\gamma f \dd s = \int_a^b f(\vb{r}(t))\norm{\vb{r}'(t)}\dd t
	$$
\end{definition}

\begin{definition}
	[Ascissa curvilinea]
	Sia $\gamma$ una curva rettificabile e $\vb{r}:[a,b]\bijarrow\gamma \in \C{1}$ una sua parametrizzazione. Sia
	$$
		s(t)=\int_a^t\norm{\vb{r}'(u)}\dd u
	$$
	$s(t)$ è detta ascissa curvilinea.
\end{definition}
$s\in\C{1}([a,b],[0,L_\gamma])$ è iniettiva e suriettiva. Inoltre $s'(t)=\norm{\vb{r}'(t)}\neq0 \ \forall t \in [a,b]$. Per questo $s$ è invertibile con inversa a sua volta $\C{1}$. $s^{-1}$ è quindi un cambiamento di variabile ammissibile e solitamente negli integrali si sottintende di star utilizzando l'ascissa curvilinea.

\section{Lavoro}

\begin{definition}
	[Campo vettoriale]
	Sia $A\subseteq \R^n$ un aperto connesso. Una funzione $\vecf: A \to \R^n$ continua è detta campo vettoriale.
\end{definition}

\begin{definition}
	[$1$-forma differenziale]\label{def:1form}
	Si definisce 1-forma differenziale $\omega=\ip{\vecf(\vb{x})}{\dd \vb{x}}$.
\end{definition}

\begin{definition}
	[Lavoro]
	Sia $\gamma \subseteq A$ una curva regolare (a tratti) orientabile con orientamento $\hat{\bm\tau}$. Il lavoro di $\vecf$ lungo $\gamma$ è
	$$
		L_{\gamma,\hat{\bm\tau}}=\int_{\gamma,\hat{\bm\tau}}\omega=\int_{\gamma, \hat{\bm\tau}}\ip{\vecf(\vb{x})}{\dd \vb{x}}=\int_\gamma \ip{\vecf}{\hat{\bm\tau}}\dd s
	$$
	dove $s$ è l'ascissa curvilinea.
\end{definition}

\begin{remark}
	Se cambia l'orientamento da $(\gamma, \hat{\bm\tau})$ in $(\gamma, -\hat{\bm\tau})$, allora $L_{\gamma,-\hat{\bm\tau}}$=$-L_{\gamma,\hat{\bm\tau}}$.
\end{remark}

Se si considera la parametrizzazione $\vb{r}: [a,b]\suarrow \gamma$, dove si assume che l'orientamento indotto dalla parametrizzazione sia compatibile con $\hat{\bm\tau}$, allora il lavoro si può calcolare come segue:
$$
	L_{\gamma, \hat{\bm\tau}}
	=\int_{a}^{b}\ip{\vecf(\vb{r}(t))}{\frac{\vb{r}'(t)}{\norm{\vb{r}'(t)}}}\norm{\vb{r}'(t)}\dd t
	=\int_a^b\ip{\vecf(\vb{r} (t))}{\frac{\dd \vb{r}}{\dd t} (t)}\dd t
$$

\begin{theorem}
	[Cambio di parametrizzazione e lavoro]
	Sia $\vb{r}:[a,b]\suarrow \gamma$ una parametrizzazione regolare che induce l'orientamento $\hat{\bm\tau}_r$ su $\gamma$ e sia $\bm\rho = \vb{r} \circ \varphi$ una nuova parametrizzazione con $\hat{\bm\tau}_\rho$ l'orientamento indotto da essa. Allora:
	\begin{enumerate} [a.]
		\item Se $\varphi$ è un diffeomorfismo crescente, $L_{\gamma, \hat{\bm\tau}_\rho} = L_{\gamma, \hat{\bm\tau}_r}$
		\item Se $\varphi$ è un diffeomorfismo crescente, $L_{\gamma, \hat{\bm\tau}_\rho} = -L_{\gamma, \hat{\bm\tau}_r}$
	\end{enumerate}
\end{theorem}

\begin{proof}
	La dimostrazione è identica a quella della proposizione \ref{prop:length_invariance} in cui si applica il cambiamento di variabile all'integrale, con la differenza che la derivata del diffeomorfismo non è all'interno del valore assoluto per cui il segno del lavoro varia a seconda del segno di quest'ultima.
	$$
		L_{\gamma,\hat{\bm\tau}_\rho}=\int_{\alpha}^{\beta}\ip{\vecf(\vb{r}(\varphi(u)))}{\frac{\dd \vb{r}}{\dd t}(\varphi(u))}\frac{\dd \varphi}{\dd u}(u)\dd u
	$$
\end{proof}

\section{Campi vettoriali conservativi}

\begin{definition}
	[Campo vettoriale conservativo]
	Siano $A \subseteq \R^n$ un aperto connesso e $\vecf \in \C{0}(A,\R^n)$. $\vecf$ è un campo vettoriale conservativo se $\exists U \in \C{1}(A,\R) \tc$
	$$
		\vecf(\vb{x})=\grad U(\vb{x}) \ \forall \vb{x} \in A
	$$
	In tal caso $U$ è detto potenziale del campo $\vecf$.
\end{definition}

\begin{prop}
	Siano $\vecf \in \C{0}(A,\R^n)$ con $A$ aperto connesso e $U \in \C{1}(A,\R)$ un suo potenziale. Allora $V\in \C{1}(A,\R)$ è un potenziale di $\vecf \iff \exists k \in \R \tc V(\vb{x})=U(\vb{x}) + k \ \forall \vb{x} \in A$.
	\qed
\end{prop}

\begin{theorem}
	[Campi vettoriali conservativi e lavoro]\label{thm:cvc_work}
	Sia $\vecf \in \C{0}(A,\R^n)$ un campo conservativo definito in $A\subseteq\R^n$ aperto e connesso e sia $U \in \C{1}(A,\R)$ un suo potenziale. Allora, se $(\gamma,\hat{\bm\tau}) \subseteq A$ è una curva regolare a tratti orientabile con primo estremo $\vb{x}_i$ e secondo estremo $\vb{x}_f$,
	$$
		L_{\gamma, \hat{\bm\tau}}=U(\vb{x}_f)-U(\vb{x}_i)
	$$
\end{theorem}

\begin{proof}
	Sia $\vb{r}:[a,b]\suarrow\gamma$ una parametrizzazione regolare a tratti che induce $\hat{\bm\tau}$.
	\begin{align*}
		L_{\gamma,\hat{\bm\tau}}
		&=\int_{a}^{b}\ip{\vecf (\vb{r}(t))}{\vb{r}'(t)}\dd t=\\
		&=\int_{a}^{b}\ip{\grad U (\vb{r}(t))}{\vb{r}'(t)}\dd t=\\
		&=\int_{a}^{b}\frac{\dd}{\dd t}(U\circ \vb{r})(t)\dd t=\\
		&=U(\vb{x}_f)-U(\vb{x}_i)
	\end{align*}
\end{proof}

\begin{theorem}
	[Caratterizzazione dei campi vettoriali conservativi]
	Siano $A\subseteq\R^n$ un aperto connesso, $\vecf \in \C{0}(A,\R^n)$. Allora le seguenti affermazioni sono equivalenti:
	\begin{enumerate}
		\item $\vecf$ è un campo vettoriale conservativo
		\item Per ogni coppia di curve regolari a tratti orientate $(\gamma_1,\hat{\bm\tau_1}), (\gamma_2,\hat{\bm\tau_2})$ con estremi coincidenti e $\gamma_1,\gamma_2 \subseteq A$ vale
		$$
			L_{\gamma_1,\hat{\bm\tau_1}}=L_{\gamma_2,\hat{\bm\tau_2}}
		$$
		\item Per ogni curva chiusa regolare a tratti orientabile con sostegno $\gamma \subseteq A$ e orientamento $\hat{\bm\tau}$ vale $L_{\gamma, \hat{\bm\tau}}=0$
		\qed
	\end{enumerate}
\end{theorem}

\begin{remark}
	Il punto \textit{1} implica il punto \textit{2} per il teorema \ref{thm:cvc_work}. I punti \textit{2} e \textit{3} sono ovviamente equivalenti. Dimostrare che il punto \textit{2} implica il punto \textit{1} comporta verificare che $U(\vb{x})=\int_{\gamma,\hat{\bm\tau}}\ip{\vecf}{\hat{\bm\tau}}\dd s$ è $\C{1}$ in $A$ e che è un potenziale per $\vecf$. Occorre utilizzare piccoli spostamenti e studiare come varia $\vecf$.
\end{remark}

\begin{definition}
	[Campo vettoriale irrotazionale]
	Siano $A\subseteq\R^n$ un aperto connesso e $\vecf \in \C{1}(A,\R^n)$. $\vecf$ è detto irrotazionale se ha matrice jacobiana simmetrica, ovvero se
	$$
		\frac{\partial f_i}{\partial x_j}(\vb{x})=\frac{\partial f_j}{\partial x_i}(\vb{x}) \ \forall \vb{x} \in A \ \forall i,j \in [n]
	$$
\end{definition}

\begin{theorem}
	Se $\vecf \in \C{1}(A,\R^n)$ con $A$ aperto connesso è conservativo, allora è irrotazionale.
\end{theorem}

\begin{proof}
	$U(\vb{x})$ è $\C{2}$, quindi vale il teorema di Schwarz (Thm. \ref{thm:schwarz}, Cap. \ref{chap:nvars}).
	$$
		\frac{\partial f_i}{\partial x_j}(\vb{x})
		=\frac{\partial}{\partial x_j}\left(\frac{\partial U}{\partial x_i}(\vb{x})\right)=\frac{\partial}{\partial x_i}\left(\frac{\partial U}{\partial x_j}(\vb{x})\right)
		=\frac{\partial f_j}{\partial x_i}(\vb{x})
	$$
\end{proof}

\begin{definition}
	[Insieme convesso]
	$A \subseteq \R^n$ aperto è un insieme convesso se $\forall \vb{x},\vb{y} \in A$ il segmento $[\vb{x},\vb{y}] \subseteq A$.
\end{definition}

\begin{definition}
	[Insieme stellato rispetto a un punto]
	Siano $A \subseteq \R^n$ e $\vb{x_0} \in A$. $A$ è stellato rispetto a $\vb{x_0}$ se $\forall \vb{x} \in A, [\vb{x_0},\vb{x}]\subseteq A$.
\end{definition}

\begin{definition}
	[Insieme semplicemente connesso]
	$A \subseteq \R^n$ è semplicemente connesso se ogni curva regolare semplice chiusa contenuta in $A$ può essere deformata con continuità a un punto in $A$ restando in $A$.
\end{definition}

\begin{remark}
	Convesso $\then$ stellato $\then$ semplicemente connesso $\then$ connesso per archi $\then$ connesso.
\end{remark}

\begin{lemma}
	[di Poincarè]
	Sia $A \subseteq \R^n$ aperto convesso oppure stellato rispetto a un punto oppure semplicemente connesso. Sia $\vecf\in \C{1}(A,\R^n)$ un campo vettoriale irrotazionale. Allora $\vecf$ è conservativo in $A$.
	\qed
\end{lemma}

\begin{corollary}\label{cor:loc_cons}
	Siano $A \subseteq \R^n$ un aperto connesso e $\vecf \in \C{1}(A,\R^n)$ un campo vettoriale irrotazionale. Allora $\forall B\subseteq A$ connesso oppure stellato oppure semplicemente connesso $\vecf$ è conservativo se ristretto a $B$.
\end{corollary}

\begin{theorem}
	Sia $\vecf \in \C{1}(\R^2\setminus{(0,0)}, \R^2)$ un campo vettoriale irrotazionale. Se, detta $\gamma$ la circonferenza di raggio unitario centrata in $(0,0)$ e orientamento arbitrario, $L_{\gamma, \hat{\bm\tau}}=0$, allora $\vecf$ è conservativo.
\end{theorem}

\begin{proof}
	Sia $\lambda$ una qualsiasi curva chiusa nel piano. L'unico caso da analizzare è quello in cui l'insieme che ha come bordo $\lambda$ contiene al suo interno il punto $(0,0)$, poiché si sa già che $\vecf$ è conservativo su qualsiasi curva che non racchiude $(0,0)$ per il corollario \ref{cor:loc_cons}. Si definisca la curva $\Gamma_\varepsilon$\footnote{Qui sarebbe stato opportuno un disegno, ma per questioni di tempo non è stato possibile aggiungerlo.}, ottenuta separando di $\varepsilon>0$ due estremi della curva $\lambda$ e due estremi della curva $\gamma$, per poi connetterli fra loro con due tratti chiamati $s_\varepsilon^1$ e $s_\varepsilon^2$. Il lavoro di $\vecf$ lungo $\Gamma_\varepsilon$ è nullo perché $\Gamma_\varepsilon$ è contenuta in un insieme semplicemente connesso (non racchiude il punto $(0,0)$). Il lavoro su $\Gamma_\varepsilon$ è la somma dei lavori su $\lambda_\varepsilon,\gamma_\varepsilon,s_\varepsilon^1 \e s_\varepsilon^2$. Mandando $\varepsilon \to 0$, $\Gamma_\varepsilon \to \Gamma$, cioè $s_\varepsilon^1,s_\varepsilon^2\to \tilde{s}, \lambda_\varepsilon \to \lambda \e \gamma_\varepsilon \to \gamma$.
	$$
		\int_{s_{\varepsilon}^1}\ip{\vecf}{\hat{\bm\tau}_\varepsilon^1}+\int_{s_{\varepsilon}^2}\ip{\vecf}{\hat{\bm\tau}_\varepsilon^2}
		\xrightarrow[s_\varepsilon^1,s_\varepsilon^2\to \tilde{s}]{\varepsilon\to 0}
		\int_{\tilde{s}}\ip{\vecf}{\hat{\bm\tau}}\dd s+\int_{\tilde{s}}\ip{\vecf}{-\hat{\bm\tau}}\dd s=0
	$$
	Di conseguenza,
	\begin{gather*}
		0=\int_{\Gamma_\varepsilon}\ip{\vecf}{\hat{\bm\tau}_\varepsilon}\xrightarrow[\lambda_\varepsilon\to\lambda,\gamma_\varepsilon\to\gamma]
		{\varepsilon\to 0}\int_\gamma \ip{\vecf}{\hat{\bm\tau}}\dd s+\int_\lambda\ip{\vecf}{\hat{\bm\tau}}\dd s\\
		\then \int_\lambda \ip{\vecf}{\hat{\bm\tau}}\dd s=0
	\end{gather*}
	Dove nell'ultimo passaggio è stata usata l'ipotesi che il lavoro su $\gamma$ sia nullo.
\end{proof}

\chapter{Integrali di superficie}

\section{Superfici}

\begin{definition}
	[Aperto regolare in $\R^2$]
	$\Omega \subseteq \R^2$ aperto, connesso e limitato è un aperto regolare di $\R^2$ se $\partial \overline{\Omega}=\partial \Omega \e \partial \Omega$ è l'unione finita e disgiunta di curve regolari a tratti, semplici e chiuse.
\end{definition}

\begin{definition}
	[Superficie]
	Siano $\Omega \subseteq \R^2$ un aperto regolare e $\vb{r}:\overline{\Omega}\to \R^3 \in \C{1}$. Si definisce superficie $\Sigma=\vb{r}(\overline{\Omega})$.
	$\Sigma$ è detta regolare se $\rank{J_{\vb{r}}(u,v)} = 2 \ \forall (u,v) \in \Omega$.
	$\Sigma$ è detta semplice se $\vb{r}:\Omega \injarrow \Sigma$.
\end{definition}

\begin{remark}
	La superficie è la classe di equivalenza delle parametrizzazioni che si ottengono le une dalle altre per composizione con un diffeomorfismo $\C{1}$ fra aperti regolari di $\R^2$.
\end{remark}

\begin{definition}
	[Superficie orientabile]
	$\Sigma \in \R^3$ è una superficie orientabile se $\exists \vb{N}:\Sigma \to \R^3 \in \C{0}$ tale che $\vb{N}(x,y,z)\in N_{(x,y,z)}\Sigma \e \norm{\vb{N}(x,y,z)}=1 \ \forall (x,y,z) \in \Sigma$, ovvero se esiste un campo continuo di versori normali a $\Sigma$.
\end{definition}

\begin{remark}
	Diffeomorfismi crescenti, ovvero con $\det J_\varphi>0$, non variano l'orientamento della superficie. Diffeomorfismi decrescenti inducono l'orientamento opposto.
\end{remark}

Sia $\vb{r}:\overline{\Omega} \to \Sigma$ una parametrizzazione regolare. Le derivate parziali di $\vb{r}$ rispetto a $u \e v$ appartengono allo spazio tangente a $\Sigma$ in $\vb{r}(u,v)$, allora il loro prodotto vettoriale $\vb{n}(u,v)=\left(\frac{\partial \vb{r}}{\partial u}\wedge\frac{\partial \vb{r}}{\partial v}\right)(u,v) \in N_{\vb{r}(u,v)}\Sigma$ e si può definire il campo di versori normale alla superficie come
$$
	\vb{N}(u,v)=\frac{(\partial_u\vb{r}\wedge\partial_v\vb{r})(u,v)}{\norm{(\partial_u\vb{r}\wedge\partial_v\vb{r})(u,v)}}\in N_{\vb{r}(u,v)}\Sigma
$$
Applicando un diffeomorfismo $\bm\varphi$ all'aperto regolare $\Omega$ si vede che l'espressione di $\vb{N}$ dipende dal segno di $\det J_{\bm\varphi}$.

\begin{definition}
	[Superficie regolare con bordo]
	Siano $\Omega \subseteq \R^2$ un aperto regolare e $\vb{r} \in \C{1}(\overline{\Omega},\R^3)$. $\Sigma = \vb{r}(\overline{\Omega})$ è una superficie regolare con bordo se:
	\begin{enumerate}
		\item $\vb{r}:\overline{\Omega}\bijarrow\Sigma$
		\item $\rank J_{\vb{r}}(u,v) = 2 \ \forall (u,v) \in \overline{\Omega}$
	\end{enumerate}
\end{definition}

\begin{theorem}
	Ogni superficie regolare con bordo è orientabile. \qed
\end{theorem}

\begin{definition}
	[Frontiera orientata canonicamente]
	Sia $\Omega \subseteq \R^2$ un aperto regolare e siano $\vb{T}(u,v)\in T_{(u,v)}\Omega$ e $\vb{N}(u,v)=(T_2(u,v),-T_1(u,v))$. $(\partial \Omega, \vb{T})$ è orientata canonicamente se $\forall (u,v) \in \partial \Omega \ \exists \lambda > 0 \tc \forall \varepsilon \in (0,\lambda)$
	\begin{enumerate}[a.]
		\item $(u,v)+\varepsilon\vb{N}(u,v) \notin \overline{\Omega}$
		\item $(u,v)-\varepsilon\vb{N}(u,v) \in \Omega$
	\end{enumerate}
\end{definition}

\begin{definition}
	[Orientamento canonico di $\partial \Sigma$]
	Sia $\vb{r}: \overline{\Omega} \bijarrow \Sigma \in \C{1}$ tale che $\rank J_{\vb{r}}(u,v)=2 \ \forall (u,v) \in \overline{\Omega}$ una parametrizzazione che induce l'orientamento $\hat{\bm\nu}$ su $\Sigma$. Siano inoltre $\gamma_1 \subset \partial\Omega$ e $\bm\varphi: [a,b]\to \gamma_1 \in \C{1}$ una parametrizzazione regolare semplice a tratti che induce l'orientamento canonico $\vb{T}$ su $\gamma_1$.
	Allora $\vb{r}\circ \bm\varphi:[a,b]\to \partial\Sigma_1$ è una parametrizzazione regolare semplice a tratti che induce l'orientamento $\hat{\bm\tau}$ canonico rispetto a $\hat{\bm\nu}$.
\end{definition}

\begin{definition}
	[Superficie regolare a tratti]
	Siano $\Sigma_1,\dots,\Sigma_p$ superfici regolari con bordo, allora $\Sigma = \bigcup\limits_{j=1}^p \Sigma_j$ è una superficie regolare a tratti se $\forall i,j \in [p] \with i \neq j \ \Sigma_i \cap \Sigma_j \subseteq \bigcup\limits_{k=1}^p\partial \Sigma_k$.
\end{definition}

\begin{definition}
	[Superficie regolare a tratti orientabile]
	$\Sigma$ superficie regolare a tratti è detta orientabile se ciascuna componente di $\partial \Sigma_j$ è orientabile in modo tale che sugli spigoli comuni questi abbiano orientamento opposto.
\end{definition}

\begin{definition}
	[Superficie chiusa]
	Le superfici senza bordo e limitate in $\R^3$ si dicono chiuse. Una superficie regolare a tratti è chiusa se $\partial \Sigma = \varnothing$ e se è limitata.
\end{definition}

\begin{definition}
	[Area di una superficie]
	Sia $\vb{r}:\overline{\Omega} \bijarrow \Sigma \tc \rank J_{\vb{r}}(u,v) = 2 \ \forall (u,v) \in \overline{\Omega}$.
	$$
		\Area (\Sigma) = \iint_\Sigma \dd S = \iint_{\overline{\Omega}} \norm{\frac{\partial \vb{r}}{\partial u}\wedge \frac{\partial \vb{r}}{\partial v}}\dd u \dd v
	$$
	Inoltre, sia $f: \Sigma \to \R$ una funzione continua.
	$$
		\iint_\Sigma f \dd S = \iint_{\overline{\Omega}}(f\circ\vb{r})(u,v)\norm{\frac{\partial \vb{r}}{\partial u}\wedge \frac{\partial \vb{r}}{\partial v}}\dd u \dd v
	$$
\end{definition}

\begin{definition}
	[Flusso]
	Sia $A\subseteq \R^3$ un aperto e sia $\Sigma \subseteq A$ una superficie orientabile con orientamento $\hat{\bm \nu}$ indotto dalla parametrizzazione $\vb{r}:\overline{\Omega}\to\Sigma$. Sia inoltre $\vecf \in \C{0}(A,\R^3)$ un campo vettoriale. Il flusso di $\vecf$ attraverso $\Sigma$ è
	\begin{align*}
		\iint_\Sigma \ip{\vecf}{\hat{\bm\nu}}\dd S 
		&=\iint_{\overline{\Omega}}\innerproduct{(\vecf\circ\vb{r})(u,v)}{\frac{(\partial_u\vb{r}\wedge\partial_v\vb{r})(u,v)}{\norm{(\partial_u\vb{r}\wedge\partial_v\vb{r})(u,v)}}}\norm{(\partial_u\vb{r}\wedge\partial_v\vb{r})(u,v)}\dd u \dd v=\\
		&= \iint_{\overline{\Omega}}\ip{(\vecf\circ\vb{r})(u,v)}{\frac{\partial \vb{r}}{\partial u} \wedge \frac{\partial \vb{r}}{\partial v}(u,v)}\dd u \dd v\\
	\end{align*}
\end{definition}

\begin{remark}
	Se $\vb{r}$ induce l'orientamento opposto,
	$$
		\iint_\Sigma \ip{\vecf}{\hat{\bm\nu}}\dd S = -\iint_{\overline{\Omega}}\ip{(\vecf\circ\vb{r}(u,v))}{\frac{\partial \vb{r}}{\partial u}\wedge \frac{\partial \vb{r}}{\partial v}}\dd u \dd v
	$$
\end{remark}

\section{Teorema di Stokes}

\begin{definition}
	[Rotore]
	Sia $\vecf\in \C{1}(A\subseteq \R^3,\R^3)$ con $A$ aperto un campo vettoriale. Si definisce rotore di $\vecf$
	$$
		\rot \vecf=\curl \vecf=\det
		\begin{bmatrix}
			\hat{\vb{i}} & \hat{\vb{j}} & \hat{\vb{k}}\\
			\partial_x & \partial_y & \partial_z\\
			f_1 & f_2 & f_3
		\end{bmatrix}
		\in \R^3
	$$
\end{definition}

\begin{theorem}
	[di Stokes o del rotore]
	Siano $A \subseteq \R^3$ un aperto, $\vecf \in \C{1}(A,\R^3)$, $\Sigma \subseteq A$ una superficie regolare con bordo con orientamento $\hat{\bm \nu}$ e $(\partial \Sigma,\hat{\bm \tau})$ il suo bordo con orientamento indotto canonicamente. Allora
	$$
		\iint_\Sigma \ip{\rot \vecf}{\hat{\bm \nu}}\dd \sigma = \int_{\partial\Sigma}\ip{\vecf}{\hat{\bm \tau}}\dd s
	$$
\end{theorem}

\begin{proof}
	[Dimostrazione in un caso particolare.]
	Si consideri il rettangolo $W=\{(x,y,0)\in[a,b]\times[c,d]\}$ posto nel piano $xy$ con orientamento $\hat{\vb{k}}$. Il flusso del rotore di $\vecf$ attraverso il rettangolo è
	\begin{align*}
		\iint_W \ip{\rot \vecf}{\hat{\bm \nu}}\dd \sigma
		&= \iint_{W}\left(\frac{\partial f_2}{\partial x}(x,y,0)-\frac{\partial f_1}{\partial y}(x,y,0)\right)\dd x \dd y=\\
		&=\int_c^d \dd y\int_a^b\frac{\partial f_2}{\partial x}(x,y,0)\dd x-
		\int_a^b\dd x\int_c^d\frac{\partial f_1}{\partial y}(x,y,0)\dd y=\\
		&=\int_c^d(f_2(b,y,0)-f_2(a,y,0))\dd y+\int_a^b(f_1(x,c,0)-f_1(x,d,0))\dd x
	\end{align*}
	Il lavoro di $\vecf$ sul bordo del rettangolo $W$ orientato canonicamente rispetto a $\hat{\vb{k}}$ è la somma dei lavori sui tratti del bordo, ovvero
	\begin{align*}
		&\int_a^bf_1(x,c,0)\dd x+\int_c^df_2(b,y,0)+\int_b^af_1(x,d,0)+\int_d^cf_2(a,y,0)=\\
		=&\int_a^b(f_1(x,c,0)-f_1(x,d,0))\dd x+\int_c^d(f_2(b,y,0)-f_2(a,y,0))\dd y
	\end{align*}
	I due integrali coincidono.
\end{proof}

\paragraph{Teorema di Stokes tramite le forme differenziali}

Il teorema di Stokes può essere espresso anche tramite l'integrazione di una forma differenziale d'area (o 2-forma), ricavata applicando l'operatore differenziale esterno alla 1-forma associata al lavoro (Def. \ref{def:1form}, Cap. \ref{chap:curves}) usando le tre regole seguenti:
\begin{enumerate}[a.]
	\item $\dd(\dd \alpha)=0$
	\item $\dd \alpha \wedge \dd \alpha=0$
	\item $\dd \alpha \wedge \dd \beta = -\dd \beta \wedge \dd \alpha$
\end{enumerate}
\begin{align*}
	\dd \omega =& \dd (f_1 \dd x+f_2 \dd y+f_3 \dd z)=\\
	&=\dd f_1\wedge \dd x +\dd f_2 \wedge \dd y +\dd f_3 \wedge \dd z=\\
	&=\left( \frac{\partial f_1}{\partial x}\dd x + \frac{\partial f_1}{\partial y}\dd y + \frac{\partial f_1}{\partial z}\dd z \right)\wedge \dd x+\\
	&+\left( \frac{\partial f_2}{\partial x}\dd x + \frac{\partial f_2}{\partial y}\dd y + \frac{\partial f_2}{\partial z}\dd z \right)\wedge \dd y +\\ 
	&+\left(\frac{\partial f_3}{\partial x}\dd x + \frac{\partial f_3}{\partial y}\dd y + \frac{\partial f_3}{\partial z}\dd z\right) \wedge \dd z= \\
	&=\left(\frac{\partial f_3}{\partial y}-\frac{\partial f_2}{\partial z}\right)\dd y \wedge \dd z+
	\left(\frac{\partial f_1}{\partial z}-\frac{\partial f_3}{\partial x}\right)\dd z \wedge \dd x+
	\left(\frac{\partial f_2}{\partial x}-\frac{\partial f_1}{\partial y}\right)\dd x \wedge \dd y
	=\\
	&=\ip{\rot \vecf}{\hat{\bm \nu}} \dd u \wedge \dd v
\end{align*}
Di conseguenza,
$$
	\iint\limits_{\Sigma,\hat{\bm \nu}} \dd \omega = \int\limits_{\partial \Sigma,\hat{\bm \tau}} \omega
$$

\begin{remark}
	Per passare da $\dd u \wedge \dd v$ a $\dd u \dd v$ nell'integrale è necessario verificare l'orientamento indotto dalla parametrizzazione utilizzata. Se la parametrizzazione induce l'orientamento corretto, $\dd u \wedge \dd v = \dd u \dd v$, altrimenti $\dd u \wedge \dd v= -\dd u \dd v$.  
\end{remark}

\section{Teorema di Gauss}

\begin{definition}
	[Aperto regolare in $\R^3$]
	$A\subseteq \R^3$ è un aperto regolare se è aperto, limitato, connesso, $\mathring{\overline A}=A \e \partial A$ è l'unione finita di superfici regolari a tratti chiuse e orientabili a due a due disgiunte. Se $A \subseteq \R^3$ è un aperto regolare, $\partial A$ è orientata canonicamente se $\forall (x,y,z) \in \partial A \ \exists \varepsilon > 0 \tc \forall \lambda \in (0,\varepsilon)$
	\begin{enumerate}[a.]
		\item $(x,y,z) + \lambda \hat{\bm \nu}(x,y,z) \notin \overline{A}$
		\item $(x,y,z) - \lambda \hat{\bm \nu}(x,y,z) \in A$
	\end{enumerate}
\end{definition}

\begin{definition}
	[Divergenza]
	Sia $\vecf \in \C{1}(A\subseteq\R^3,\R^3)$ un campo vettoriale. Si definisce divergenza di $\vecf$
	$$
		\divop \vecf=\divergence \vecf=\frac{\partial f_1}{\partial x} + \frac{\partial f_2}{\partial y} + \frac{\partial f_3}{\partial z} \in \R
	$$
\end{definition}

\begin{theorem}
	[di Gauss o della divergenza]
	Siano $A\subseteq \R^3$ un aperto regolare, $\vecf \in \C{1}(\overline{A},\R^3) \e (\partial A, \hat{\bm \nu})$ la frontiera di $A$ orientata canonicamente. Allora
	$$
		\iiint_A \divop \vecf(x,y,z)\dd x \dd y \dd z = \iint_{\partial A}\ip{\vecf}{\hat{\bm \nu}}\dd \sigma
	$$
\end{theorem}

\begin{proof}
	[Giustificazione in un caso semplice.]
	Si consideri il cubo $Q=\{(x,y,z)\in[0,L]\times[0,L]\times[0,L]\}$. Il flusso del campo $\vecf$ attraverso le facce di $\partial Q$ parallele al piano $xz$ è
	$$
		\iint_{\partial Q_y}\innerproduct{\vecf}{\hat{\bm\nu}}=-\iint_{[0,L]\times[0,L]}f_2(x,0,z)\dd x \dd z
		+\iint_{[0,L]\times[0,L]}f_2(x,L,z)\dd x \dd z
	$$
	L'integrale del termine della divergenza dipendente da $y$ è
	$$
		\iiint_{Q}\frac{\partial f_2}{\partial y}(x,y,z)\dd x \dd y \dd z=\iint_{[0,L]\times[0,L]}(f_2(x,L,z)-f_2(x,0,z))\dd x \dd z
	$$
	I due integrali coincidono ed è possibile applicare lo stesso ragionamento alle variabili $x$ e $z$.
\end{proof}

\paragraph{Teorema di Gauss tramite le forme differenziali}

Anche nel caso del teorema di Gauss è possibile esprimere l'enunciato del teorema in termini di forme differenziali. Sia $\omega = f_1 \dd y \wedge \dd z + f_2 \dd z \wedge \dd x + f_3 \dd x \wedge \dd y$ la forma differenziale d'area associata al flusso del campo $\vecf$. Applicando l'operatore differenziale esterno, si ottiene che
\begin{align*}
	\dd \omega &=\dd(f_1 \dd y \wedge \dd z + f_2 \dd z \wedge \dd x + f_3 \dd x \wedge \dd y)=\\
	&=\dd f_1 \wedge \dd y \wedge \dd z+ \dd f_2\wedge\dd z \wedge \dd x+\dd f_3\dd\wedge x \wedge \dd y=\\
	&= \left(\frac{\partial f_1}{\partial x} + \frac{\partial f_2}{\partial y} + \frac{\partial f_3}{\partial z}\right)\dd x \wedge \dd y \wedge \dd z\\
\end{align*}
Di conseguenza, si può scrivere
$$
	\iiint\limits_A \dd \omega = \iint\limits_{\partial A^+}\omega
$$
dove $\partial A^+$ è la frontiera di $A$ orientata canonicamente.

\begin{remark}
	Si noti che la somiglianza fra il teorema di Gauss e il teorema di Stokes è dovuta al fatto che essi sono casi particolari dello stesso teorema applicato a dimensioni diverse.
\end{remark}

\chapter{Misura e integrazione secondo Lebesgue}

\section{Misura di Lebesgue}

\begin{definition}
	[Misura esterna]
	Siano $X$ un insieme e $\calP(X)=\{A:A\subseteq X\}$ l'insieme delle parti di $X$. $\mu^*:\calP(X)\to [0,+\infty]$ è una misura esterna se soddisfa le seguenti condizioni:
	\begin{enumerate}
		\item $\mu^*(\varnothing)=0$
		\item (Monotonia) Se $A\subseteq B$, allora $\mu^*(A) \leq \mu^*(B)$
		\item (Sub-additività numerabile) $\mu^*\left(\bigcup\limits_{i=1}^\infty A_i\right)\leq \sum\limits_{i=1}^\infty \mu^*(A_i)$
	\end{enumerate}
\end{definition}

\begin{definition}
	[$\sigma$-algebra]
	Sia $X$ un insieme. $\A \subseteq \calP(X)$ è una $\sigma$-algebra se:
	\begin{enumerate}
		\item $\varnothing, X \in \A$
		\item Se $A\subset \A$, allora $X \setminus A \in \A$
		\item Se $A_j \in \A \with j \in \N$, allora $\bigcup\limits_{i=1}^\infty A_i, \bigcap\limits_{i=1}^\infty A_i \in \A$
	\end{enumerate}
\end{definition}

\begin{prop}
	Se $C \subseteq \calP(X)$, allora $C$ può essere sempre completato a una $\sigma$-algebra ed esiste il completamento minimo $\A(C)$ che è la $\sigma$-algebra più piccola in $\calP(X)$ contenente $C$.
	$$
		\A(C)=\cap\{\A:\A \text{ è una } \sigma \text{-algebra in } \calP(X), C \subseteq \A\}
	$$
	\qed
\end{prop}

\begin{remark}\leavevmode
	\begin{enumerate}
		\item $\{\varnothing,X\}$ è la più piccola $\sigma$-algebra associata ad $X$
		\item $\calP(X)$ è la più grande $\sigma$-algebra associata ad $X$
	\end{enumerate}
\end{remark}

\begin{definition}
	[$\sigma$-algebra di Borel]
	Sia $(X,\tau)$ uno spazio topologico. Si definisce $\sigma$-algebra di Borel la più piccola $\sigma$-algebra in $X$ contenente $\tau$.
\end{definition}

\begin{definition}
	[Spazio misurabile]
	Si definisce spazio misurabile la coppia $(X,\A)$, dove $X$ è un insieme e $\A$ una $\sigma$-algebra contenuta in $\calP(X)$.
\end{definition}

\begin{definition}
	[Misura]
	Sia $(X,\A)$ uno spazio misurabile. $\mu:\A \to [0,+\infty]$ è una misura se soddisfa le seguenti richieste:
	\begin{enumerate}
		\item $\mu(\varnothing)=0$
		\item (Additività numerabile) Se $A_i \in \A \ \forall i \in \N$ sono insiemi a due a due disgiunti, allora $$\displaystyle\mu\left(\bigcup_{i=1}^\infty A_i \right) = \sum_{i=1}^\infty \mu(A_i)$$
	\end{enumerate}
	$\mu$ si dice finita se $\mu(X)<+\infty$. $\mu$ si dice $\sigma$-finita se esistono $A_i \with i \in \N$ insiemi misurabili di misura finita e $X=\bigcup\limits_{i=1}^\infty A_i$.
\end{definition}

\begin{remark}
	La proprietà di monotonia discende dal fatto che la misura è sempre definita a partire da una misura esterna.
\end{remark}

\begin{definition}
	[Spazio di misura]
	$(X,\A,\mu)$ è detto spazio di misura.
\end{definition}

\begin{definition}
	[Spazio di misura completo]
	$(X,\A,\mu)$ è uno spazio di misura completo se $\forall N \in \A \tc \mu(N)=0$ tutti i suoi sottoinsiemi sono misurabili, cioè $\forall M\subseteq N, \ M\in \A$.
\end{definition}

\begin{theorem}
	[Successioni monotone in uno spazio di misura]
	Sia $(X,\A,\mu)$ uno spazio di misura.
	\begin{enumerate}
		\item Se $A_i \in \A$ è una successione crescente, cioè $A_i \subseteq A_{i+1} \ \forall i \in \N$, allora $\mu\left(\bigcup\limits_{i=1}^\infty A_i\right) = \lim\limits_{i\to + \infty}\mu(A_i)$
		\item Se $A_i \in \A$ è una successione decrescente, cioè $A_i \supseteq A_{i+1} \ \forall i \in \N$ e $\exists j \in \N \tc \mu(A_j) < + \infty$, allora $\mu\left(\bigcap\limits_{i=1}^\infty A_i\right)=\lim\limits_{i\to +\infty}\mu(A_i)$
	\end{enumerate}
	\qed
\end{theorem}

\begin{definition}
	[Intervallo]
	Sia $(\R^n,\tau)$ lo spazio topologico euclideo. Si definisce intervallo $R=[a_1,b_1]\times\cdots[a_n,b_n] \subseteq \R^n$. La misura elementare di $R$ è $\mu(R)=\prod\limits_{j=1}^n(b_j-a_j)$. Si indica con $\calR$ l'insieme degli intervalli di $\R^n$: $\calR=\{A \in \calP(\R^n):A \text{ è un intervallo}\}$.
\end{definition}

\begin{definition}
	[Misura esterna di Lebesgue]
	Si definisce la misura esterna di Lebesgue $\mu^*:\calP(\R^n)\to [0,+\infty]$ tale che, se $E \in \calP(\R^n)$,
	$$
		\mu^*(E)=\inf \left\{\sum\limits_{j=1}^\infty \mu(R_j):R_j \in \calR \ \forall j \in \N \e E \subseteq \bigcup\limits_{j=1}^\infty R_j\right\}
	$$
\end{definition}

\begin{theorem}
	$\mu^*$ è una misura esterna, cioè
	\begin{enumerate}
		\item $\mu^*(\varnothing)=0$
		\item Se $A \subseteq B$, allora $\mu^*(A)\leq \mu^*(B)$
		\item Se $A_i \subseteq \R^n, i \in \N$, allora $\mu^*\left(\bigcup\limits_{i=1}^\infty A_i\right)\leq \sum\limits_{i=1}^\infty \mu^*(A_i)$
	\end{enumerate}
\end{theorem}

\begin{proof}
	[Dimostrazione del punto 3.]
	Sia $\varepsilon>0$. $\forall i \in \N$ è possibile trovare un ricoprimento di $A_i$ tale che
	$$A_i\subseteq \bigcup\limits_{j=1}^\infty R_{ij} \e \sum\limits_{j=1}^\infty \mu^*(R_{ij})\leq \mu^*(A_i)+\frac{\varepsilon}{2^i}$$
	Si noti che l'unico caso da dimostrare è quello in cui $\mu^*(A_i)<+\infty \ \forall i \e \sum\limits_{i=1}^\infty \mu^*(A_i)<+\infty$, perché altrimenti è ovvio che $\mu^*\left(\bigcup\limits_{i=1}^\infty A_i\right)\leq +\infty$. $\bigcup\limits_{i,j=1}^\infty R_{ij}$ è un ricoprimento di $E=\bigcup\limits_{i=1}^\infty A_i$.
	\begin{gather*}
		\mu^*(E)=\mu^*\left(\bigcup_{i=1}^\infty A_i\right)\leq \sum_{i,j=1}^{\infty}\mu^*(R_{ij})=\sum_{i=1}^\infty\left(\sum_{j=1}^{\infty}\mu^*(R_{ij})\right)\leq\\
		\leq\sum_{i=1}^{\infty}\left(\mu^*(A_i)+\frac{\varepsilon}{2^i}\right)=\sum_{i=1}^\infty \mu^*(A_i)+\varepsilon\\
		\then \mu^*\left(\bigcup_{i=1}^\infty A_i\right)\leq \sum_{i=1}^\infty \mu^*(A_i)
	\end{gather*}
\end{proof}

\begin{prop}
	La misura elementare di un intervallo $R \in \calR$ coincide con la misura esterna di Lebesgue dell'intervallo.
	\qed
\end{prop}

\subsection{Metodo di Carathéodory}

\begin{definition}
	[Insieme misurabile secondo Carathéodory]
	Siano $X$ un insieme e $\mu^*:\calP(X) \to [0,+\infty]$ una misura esterna su $X$. $A\subseteq X$ è misurabile secondo Carathéodory se $\forall E \subseteq X, \ \mu^*(E)=\mu^*(A\cap E) + \mu^*\left(A^C \cap E\right)$, dove $A^C=X\setminus A$.
\end{definition}

\begin{theorem}
	[Costruzione della misura]
	Siano $X$ un insieme e $\mu^*$ una misura esterna su $\calP(X)$. Allora l'insieme $\A \subseteq \calP(X)$ degli insiemi misurabili secondo Carathéodory è una $\sigma$-algebra e $\restr{\mu^*}{\A}$ è una misura.
	\qed
\end{theorem}

\begin{definition}
	[Misura di Lebesgue in $\R^n$]
	Sia $\mathcal{L}(\R^n)$ la $\sigma$-algebra degli insiemi misurabili secondo Carathéodory rispetto alla misura esterna di Lebesgue. La misura di Lebesgue è $\mu=\restr{\mu^*}{\mathcal{L}(\R^n)}$.
\end{definition}

\begin{theorem}
	[Completezza della misura di Lebesgue]
	La misura di Lebesgue è una misura completa, cioè:
	\begin{enumerate}
		\item $\forall N \subseteq \R^n$ con misura esterna nulla, $N \in \mathcal{L}(\R^n)$
		\item $\forall M \subseteq N \with \mu(N)=0, \ M \in \mathcal{L}(\R^n)$
	\end{enumerate}
\end{theorem}

\begin{proof}
	Si dimostra ciascun punto separatamente.
	\begin{enumerate}
		\item Siano $\mu^*(N)=0 \e E \subseteq \R^n$. Si noti che per la proprietà di subadditività è ovvio che $\mu^*(E)\leq \mu^*(E\cap N)+\mu^*\left(E\cap N^C\right)$, quindi resta da dimostrare solo che $\mu^*(E)\geq\mu^*\left(E\cap N^C\right)+\mu^*(E\cap N)$. $E \cap N \subseteq N\then \mu^*(E\cap N)\leq\mu^*(N)=0\then \mu^*(E\cap N)=0$. Inoltre $E\supseteq E\cap N^C$, quindi $\mu^*(E)\geq \mu^*\left(E\cap N^C\right)=\mu^*\left(E\cap N^C\right)+\mu^*(E\cap N)$. Quindi $N$ è misurabile secondo Carathéodory.
		\item $\forall M\subseteq N \ \mu^*(M)\leq\mu^*(N)=0$, quindi anche $M\in \mathcal{L}(\R^n)$.
	\end{enumerate}
\end{proof}

\begin{remark}
	Nella dimostrazione non viene usata la definizione di misura esterna di Lebesgue. Tutte le misure costruite con il metodo di Carathéodory sono complete.
\end{remark}

\begin{definition}
	[Insieme di misura nulla]
	$E \subseteq \R^n$ ha misura di Lebesgue nulla se e solo se $\forall \varepsilon > 0$ esiste un ricoprimento numerabile di $E$ con intervalli tale che $E \subseteq \bigcup\limits_{j=1}^\infty R_j, R_j \in \calR \e \sum\limits_{j=1}^\infty \mu(R_j)<\varepsilon$.
\end{definition}

\begin{theorem}
	[Relazione fra $\mathcal{B}(\R^n)$ e $\mathcal{L}(\R^n)$]
	Sia $\mathcal{B}(\R^n)$ la $\sigma$-algebra di Borel in $\R^n$. Allora
	\begin{enumerate}
		\item $\mathcal{B}(\R^n)$ è la più piccola $\sigma$-algebra che contenga $\mathcal{R}$, cioè gli intervalli di $\R^n$
		\item $\mathcal{B}(\R^n)\subset \mathcal{L}(\R^n)$
		\item $\mathcal{B}(\R^n)$ non è completo rispetto alla misura di Lebesgue
		\qed
	\end{enumerate}
\end{theorem}

\begin{theorem}
	[Misura con aperti e compatti]\leavevmode
	\begin{enumerate}
		\item Sia $A\subseteq\R^n$, allora $\mu^*(A)=\inf\{\mu(U):A\subseteq U \with U \text{ aperto}\}$. In particolare, se $A$ è misurabile allora $\mu^*(A)=\mu(A)$
		\item Se $A \in \mathcal{L}(\R^n)$, allora $\mu(A)=\sup\{\mu(K):K \text{ compatto}, K \subseteq A\}$
		\qed
	\end{enumerate}
\end{theorem}

\begin{corollary}
	$A \in \mathcal{L}(\R^n) \iff \forall \varepsilon > 0 \ \exists V \text{ chiuso},\ U \text{ aperto},\ V \subseteq A \subseteq U \tc \mu(U\setminus V) < \varepsilon$.\qed
\end{corollary}

\begin{definition}
	[$G_\delta \e F_\sigma$]
	Si definiscono i seguenti insiemi:
	\begin{itemize}
		\item $G_\delta = \left\{G \subseteq \R^n:G=\bigcap\limits_{j=1}^\infty U_j, U_j \text{ aperto } \forall j \in \N\right\} \subseteq \mathcal{B}(\R^n)$
		\item $F_\sigma = \left\{F \subseteq \R^n:F=\bigcup\limits_{j=1}^\infty C_j, C_j \text{ chiuso } \forall j \in \N\right\} \subseteq \mathcal{B}(\R^n)$
	\end{itemize}
\end{definition}

\begin{theorem}
	$\forall A \in \mathcal{L}(\R^n) \ \exists G \in G_\delta, F \in F_\sigma \tc F \subseteq A \subseteq G \e \mu(G\setminus A)=\mu(A\setminus F)=0$. In altre parole, ogni insieme misurabile secondo Lebesgue è approssimato a meno di insiemi di misura nulla da un insieme in $F_\sigma$ per difetto e da un insieme in $G_\delta$ per eccesso.
	\qed
\end{theorem}

\begin{corollary}
	Il completamento della $\sigma$-algebra di Borel è la $\sigma$-algebra degli insiemi misurabili secondo Lebesgue. Per ottenerla vengono aggiunti gli insiemi di misura nulla.\qed
\end{corollary}

\begin{theorem}
	[Invarianza della misura di Lebesgue per isometrie]\leavevmode
	\begin{enumerate}
		\item Siano $A\subseteq \R^n$ e $\vb{h} \in \R^n$. Sia $A_{\vb{h}}=\{\vb{x}+\vb{h}:\vb{x}\in A\}$ il traslato di $A$. Allora $A \in \mathcal{L}(\R^n)\iff A_{\vb{h}} \in \mathcal{L}(\R^n)$ e in tal caso $\mu(A)=\mu(A_{\vb{h}})$.
		\item Siano $Q\in \mathcal{O}(n)$ (il gruppo delle matrici ortogonali in $\R^n$) e $E\subseteq \R^n$. Allora $E \in \mathcal{L}(\R^n) \iff QE \in \mathcal{L}(\R^n)$ e in tal caso $\mu(E)=\mu(QE)$.
		\item Siano $T: \R^n \to \R^n$ un'applicazione lineare con $\det T \neq 0$ e $A \subseteq \R^n$. Allora $A \in \mathcal{L}(\R^n) \iff TA \in \mathcal{L}(\R^n)$ e in tal caso $\mu(TA) = \abs{\det T}\mu(A)$.
		\qed
	\end{enumerate}
\end{theorem}

\section{Integrale di Lebesgue}

\begin{definition}
	[Funzione semplice]
	Siano $E_j \subseteq \R^n \with j=1,\dots,k$ insiemi misurabili secondo Lebesgue e $C_j \in \R$. Sia inoltre $\chi_{E_j}$ la funzione caratteristica dell'insieme $E_k$ definita come segue:
	$$
		\chi_{E_j}(\vb{x}) =
		\begin{cases}
			1 & \vb{x} \in E_j\\
			0 & \vb{x} \notin E_j
		\end{cases}
	$$
	$\varphi: \R^n \to [0, +\infty)$ definita come $\varphi(\vb{x})=\sum\limits_{j=1}^k C_j \chi_{E_j}(\vb{x})$ è detta funzione semplice. $\varphi$ è una funzione semplice positiva se $C_j \geq 0 \ \forall j=1,\dots,k$.
\end{definition}

\begin{definition}
	[Integrale secondo Lebesgue di una funzione semplice positiva]
	Sia $\varphi(\vb{x})=\sum\limits_{j=1}^k C_j \chi_{E_j}(\vb{x})$ con $C_j \geq 0 \ \forall j=1,\dots,k$. Si definisce integrale secondo Lebesgue di $\varphi$
	$$
		\idotsint_{\R^n}\varphi(x_1,\dots,x_n)\dd x_1 \cdots \dd x_n = \sum_{j=1}^kC_j\mu(E_j)
	$$
	con la convenzione che se $C_j=0$ e $\mu(E_j)=+\infty$, allora $C_j\mu(E_j)=0$.
	Inoltre $\varphi$ si dice sommabile se $\int_{\R^n} \varphi < + \infty$.
\end{definition}

\begin{theorem}
	[Proprietà dell'integrale di Lebesgue sulle funzioni semplici positive]\leavevmode
	\begin{enumerate}
		\item (Linearità) Se $\varphi,\psi$ sono funzioni semplici positive e $c\in \R^+$, allora $c\varphi + \psi$ è una funzione semplice positiva e $\int(c\varphi + \psi)=c\int\varphi +\int\psi$
		\item (Monotonia) Se $\varphi, \psi$ sono funzioni semplici positive e $\varphi \leq \psi \ \forall \vb{x} \in \R^n$, allora  $\int \varphi \leq \int \psi$
		\qed
	\end{enumerate}
\end{theorem}

\begin{definition}
	[Funzione misurabile secondo Lebesgue]
	Siano $A \in \mathcal{L}(\R^n) \e f:A\to \R$. $f$ è misurabile secondo Lebesgue se $\forall \beta \in \R, \ \{\vb{x}\in A:f(\vb{x})<\beta\} \in \mathcal{L}(\R^n)$.
\end{definition}

\begin{remark}
	Le funzioni semplici sono misurabili secondo Lebesgue.
\end{remark}

\begin{theorem}
	[Caratterizzazione delle funzioni misurabili positive tramite funzioni semplici positive]
	Siano $A \subseteq \R^n$ misurabile secondo Lebesgue e $f:A\to [0,+\infty]$ misurabile secondo Lebesgue. Allora esiste una successione crescente di funzioni semplici non negative $\varphi_k:A\to [0,+\infty)$ convergenti ad $f$ puntualmente. Inoltre se $f$ è limitata, la convergenza delle $\varphi_k$ a $f$ è uniforme.
\end{theorem}

\begin{proof}
	Per ogni $k\in \N$, si consideri l'intervallo $[0,2^k]$ nell'insieme dei valori di $f$ e lo si divida in $2^{2k}$ sottointervalli di ampiezza $2^{-k}$.
	$$
		I_{k,l}=\left[\frac{l}{2^k},\frac{l+1}{2^k}\right) \with l=1,\dots,2^{2k}-1
	$$
	Si definisca $J_k=[2^k,+\infty]$. Siano inoltre
	$$
		E_{k,l}=f^{-1}(I_{k,l}) \e F_k=f^{-1}(J_k)
	$$
	$E_{k,l} \e F_k$ sono misurabili. Allora la successione
	$$
		\varphi_k=\sum_{l=0}^{2^{2k}-1}\frac{l}{2^k}\chi_{E_{k,l}}+2^k\chi_{F_k}
	$$
	soddisfa alla tesi del teorema.   
\end{proof}

\begin{definition}
	[Integrale di Lebesgue di una funzione misurabile non negativa]
	Se $A\in \mathcal{L}(\R^n) \e f:A\to [0,+\infty)$ è misurabile, allora $\int_Af=\sup\{\int_A \varphi_k:\varphi_k \leq f, \varphi_k \text{ è una funzione semplice positiva}\}$. In particolare, $f$ è detta sommabile se $\int_A f<+\infty$.
\end{definition}

\begin{theorem}
	[Proprietà dell'integrale di funzioni misurabili non negative]\label{thm:int_nonneg}
	\leavevmode
	\begin{enumerate}
		\item\label{item:mis_int} Ogni funzione misurabile non negativa definita in un insieme misurabile è integrabile secondo Lebesgue
		\item (Linearità) Se $f,g:A\to [0,+\infty]$ sono misurabili in $A\in \mathcal{L}(\R^n)$, allora $f+cg \with c \in \R^+$ è misurabile e $\int_A (f+cg)=\int_Af+c\int_Ag$
		\item (Monotonia) Se $f,g:A\to[0,+\infty]$ sono misurabili in $A\in\mathcal{L}(\R^n) \e f\leq g$, allora $\int_Af\leq\int_Ag$
		\item Se $B\subseteq A \with B\in \mathcal{L}(\R^n)$, allora $\int_Bf=\int_A\chi_Bf$
		\qed
	\end{enumerate}
\end{theorem}

\begin{theorem}
	[Proprietà delle funzioni misurabili]
	Siano $A \in \mathcal{L}(\R^n), \ f,g:A\to\R$ misurabili e $c \in \R$. Allora $f+g,\ cf,\ f/g,\ \abs{f},\ \max\{f,g\} \e \min\{f,g\}$ sono misurabili in $A$. Nel caso della divisione, $g(\vb{x})\neq 0 \ \forall \vb{x} \in A$.
	\qed
\end{theorem}

\begin{theorem}
	[Misurabilità e convergenza puntuale]\label{thm:mis_conv}
	Siano $A \in \mathcal{L}(\R^n)$, $f_k:A\to\R \with k \in \N$ una successione di funzioni misurabili e $f(\vb{x})=\lim\limits_{k\to+\infty}f_k(\vb{x}) \ \forall \vb{x} \in A$. Allora $f$ è misurabile secondo Lebesgue.
	\qed
\end{theorem}

\begin{definition}
	[Funzione integrabile di segno variabile]
	Siano $A\in\mathcal{L}(\R^n) \e f:A\to\R$ misurabile secondo Lebesgue. Si definiscono la parte positiva $f_+$ e la parte negativa $f_-$ di $f$ come nella Definizione \ref{def:ppos_pneg} del Capitolo \ref{chap:peano_jordan}. $f_+ \e f_-$ sono misurabili. Se almeno uno fra $\int_A f_+ \e \int_A f_-$ è finito, si definisce $\int_A f= \int_A f_+ - \int_A f_-$. Se $\int_A f_+,\int_A f_- <+\infty, \ f$ è detta sommabile.
\end{definition}

\begin{theorem}[Proprietà dell'integrale di Lebesgue]
	Se $A \in \mathcal{L}(\R^n), \ f,g:A \to \overline{\R}$ sono misurabili e sommabili secondo Lebesgue, allora:
	\begin{enumerate}
		\item (Linearità) $\forall c \in \R, \ f+cg$ è sommabile e $\int_A(f+cg)=\int_Af+c\int_Ag$
		\item (Monotonia) Se $f\leq g$, allora $\int_Af\leq\int_Ag$
		\item $\abs{f}$ è sommabile e $\abs{\int_Af}\leq\int_A\abs{f}$
		\qed
	\end{enumerate}
\end{theorem}

\begin{theorem}[Integrale di Riemann e di Lebesgue]\leavevmode
	\begin{enumerate}
		\item Se $f:[a,b]\to\R$ è integrabile secondo Riemann, allora $f$ è integrabile secondo Lebesgue e $$\int_{a(\mathcal{R})}^bf=\int_{a(\mathcal{L})}^{b}f$$
		\item Sia $f:[a,b]\to\R$ misurabile secondo Lebesgue. Allora $f$ è integrabile secondo Riemann se e solo se soddisfa queste condizioni:
		\begin{enumerate}[a.]
			\item $f$ è limitata
			\item L'insieme dei punti di discontinuità di $f$ ha misura di Lebesgue nulla
			\qed
		\end{enumerate}
	\end{enumerate}
\end{theorem}

Le funzioni $f:[a,b]\to\R$ continue in $[a,b]$ oppure monotone in $[a,b]$ sono integrabili secondo Riemann e quindi anche secondo Lebesgue. La funzione di Dirichlet ($\chi_{\mathbb{Q}\cap[0,1]}$) non è integrabile secondo Riemann, ma lo è secondo Lebesgue e ha integrale nullo. La funzione
$$
	f(x)=\frac{\dd}{\dd x}\left[x\cos(\frac{1}{x})\right] \with x \in [0,1]
$$
non è integrabile secondo Lebesgue, ma è integrabile in senso generalizzato secondo Riemann.

\begin{theorem}
	[di Fubini]
	Sia $f:\R^n\times\R^m\to\R$ misurabile secondo Lebesgue. $\abs{f}$ è sommabile in $\R^n\times\R^m$ se e solo se almeno uno dei seguenti integrali esiste finito: $\int_{\R^n}\left(\int_{\R^m}\abs{f}\right)<+\infty$ oppure $\int_{\R^m}\left(\int_{\R^n}\abs{f}\right)<+\infty$. In tal caso $f$ è sommabile e 
	$$
		\iint_{\R^n\times\R^m}f(\vb{x},\vb{y})\dd \vb{x} \dd \vb{y} = \int_{\R^n}\dd \vb{x}\left(\int_{\R^m} \dd \vb{y} f(\vb{x},\vb{y})\right)=\int_{\R^m}\dd \vb{y} \left(\int_{\R^n}\dd \vb{x} f(\vb{x},\vb{y})\right)
	$$
	\qed
\end{theorem}

\subsection{Lebesgue e i limiti di successioni}

\begin{theorem}
	[Successioni di funzioni sommabili uniformemente convergenti]
	Siano $A\in \mathcal{L}(\R^n)$ limitato e $f_k,f:A\to\R \with k \in \N$ misurabili. Se $f_k \rightrightarrows f$ in $A$ e $f_k$ è sommabile $\forall k \in \N$, allora $f$ è sommabile e $\displaystyle\int_A f = \lim\limits_{k\to+\infty}\int_Af_k$.
\end{theorem}

\begin{proof}
	$f$ è sommabile:
	$$\displaystyle\abs{\int_A (f-f_k)}\leq \int_A\abs{f-f_k}\leq \int_A\sup_{x \in A}\abs{f-f_k}=M_k\mu_n(A)\xrightarrow{k\to+\infty}0$$
	Inoltre
	$$
		\int_Af=\int_A(f-f_k)+\int_Af_k\then \int_Af=\lim_{k\to+\infty}\int_Af_k
	$$
\end{proof}

\begin{theorem}
	[Convergenza monotona]
	Siano $A \in \mathcal{L}(\R^n)$ e $f_k:A\to[0,+\infty)$ una successione crescente di funzioni misurabili. Allora
	\begin{enumerate}
		\item $f_k$ convergono puntualmente a $f: A\to[0,+\infty]$: $$f(\vb{x})=\lim\limits_{k\to+\infty}f_k(\vb{x})=\sup_kf_k(\vb{x})$$
		\item $f$ è integrabile e $\displaystyle\int_Af=\lim_{k\to+\infty}\int_Af_k$
		\qed
	\end{enumerate}
\end{theorem}

\begin{remark}
	Si noti che le $f_k$ sono integrabili per il punto \textit{\ref{item:mis_int}} del teorema \ref{thm:int_nonneg}. Inoltre $f$ è misurabile per il teorema \ref{thm:mis_conv} e quindi integrabile per il punto \textit{\ref{item:mis_int}} del teorema \ref{thm:int_nonneg}.
\end{remark}

\begin{theorem}
	[della convergenza dominata]
	Siano $A\in \mathcal{L}(\R^n)$ e $f_n:A\to\R$ misurabili. Sotto le seguenti ipotesi:
	\begin{enumerate}[a.]
		\item $\exists g:A\to[0,+\infty)$ sommabile tale che $\abs{f_n}\leq g\ \forall n \in \N \e \int_Ag<+\infty$
		\item $f_n\xrightarrow[n\to+\infty]{A}f$, cioè che $f$ sia il limite puntuale di $f_n$
	\end{enumerate}
	allora $f_n,f$ sono sommabili e $\displaystyle\int_Af=\lim\limits_{n\to+\infty}\int_Af_n$.
	\qed
\end{theorem}


\end{document}
