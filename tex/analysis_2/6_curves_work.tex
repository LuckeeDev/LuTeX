\chapter{Curve e lavoro}\label{chap:curves}

\section{Curve in forma parametrica}

\begin{definition}
	[Curva]
	Si definisce curva in $\R^n$ una funzione $\vb{r}:[a,b]\to \R^n$ continua. $\vb{r}([a,b])$ è detto sostegno della curva.
\end{definition}

\begin{remark}
	La richiesta della continuità non è sufficiente perché $\vb{r}([a,b])$ sia un sottoinsieme di dimensione $1$ di $\R^n$. Esempio: la curva di Peano.
\end{remark}

\begin{definition}
	[Parametrizzazione regolare]
	Si dice parametrizzazione regolare della curva una funzione $\vb{r}:[a,b]\to \R^n \in \C{1} \tc \vb{r}'(t)\neq\vb{0} \ \forall t \in [a,b]$.
\end{definition}

\begin{definition}
	[Parametrizzazione semplice aperta]
	Si dice parametrizzazione semplice aperta una funzione $\vb{r}: [a,b]\bijarrow\R^n \in \C{1}$. Si noti che $\vb{r}$ è un omeomorfismo.
\end{definition}

\begin{definition}
	[Parametrizzazione semplice chiusa]
	Si dice parametrizzazione semplice chiusa una funzione $\vb{r}: [a,b]\bijarrow \vb{r}((a,b)) \e \vb{r}(a)=\vb{r}(b)$.
\end{definition}

\begin{theorem}
	[Invarianza per cambio di parametrizzazione]
	Siano $\vb{r} \in \C{1}([a,b],\gamma) \e \varphi:[\alpha,\beta]\bijarrow[a,b]\\\in \C{1} \with \varphi^{-1} \in \C{1}$ (ovvero, $\varphi$ è un diffeomorfismo). Sia $\bm\rho = \vb{r} \circ \varphi: [\alpha, \beta] \to \R^n$. Allora
	\begin{enumerate}
		\item $\bm\rho([\alpha,\beta])=\gamma$, ovvero il sostegno della curva è invariato
		\item $\bm\rho$ è regolare se e solo se $\vb{r}$ è regolare
		\item $\bm\rho$ è semplice se e solo se $\vb{r}$ è semplice
		\qed
	\end{enumerate}
\end{theorem}

\begin{definition}
	[Curva orientabile]
	$\gamma \subseteq \R^n$ connesso e compatto è orientabile se $\exists \vb{T}: \gamma \to \R^n \tc \forall \vb{x} \in \gamma \ \vb{T}(\vb{x})\in T_{\vb{x}}\gamma \e \norm{\vb{T}(\vb{x})}=1$. In altre parole, è necessario che esista un campo continuo di versori tangenti alla curva. Si definisce orientamento indotto dalla parametrizzazione $\vb{r}$ il campo di versori
	$$
		T(x)=\frac{\vb{r}'(r^{-1}(\vb{x}))}{\norm{\vb{r}'(r^{-1}(\vb{x}))}} \in T_{\vb{x}}\gamma
	$$
\end{definition}

\begin{lemma}
	Ogni curva semplice e regolare è orientabile. La semplicità garantisce che $\gamma$ abbia esattamente due orientamenti.\qed
\end{lemma}

\begin{theorem}
	[Effetto del cambio di parametrizzazione sull'orientamento]
	Sia $\gamma$ orientabile e sia $\vb{T}(\vb{x})=\frac{\vb{r}'(r^{-1}(\vb{x}))}{\norm{\vb{r}'(r^{-1}(\vb{x}))}}$. Vi sono due casi:
	\begin{enumerate}[a.]
		\item $\varphi$ è un diffeomorfismo crescente, quindi l'orientamento indotto da $\bm\rho$ è lo stesso di $\vb{r}$
		\item $\varphi$ è un diffeomorfismo decrescente, quindi l'orientamento indotto da $\bm\rho$ è opposto a quello indotto da $\vb{r}$
	\end{enumerate}
\end{theorem}

\begin{proof}
	% TODO
\end{proof}

\begin{definition}
	[Parametrizzazione regolare a tratti]
	Si definisce parametrizzazione regolare a tratti $\vb{r}:[a,b]\to \R^n$ continua se $\exists a=t_1<\cdots<t_k=b \tc \vb{r}:[t_i, t_{i+1}]\bijarrow \vb{r}([t_i,t_{i+1}]) \in \C{1} \e \vb{r}'(t)\neq \vb{0} \ \forall t \in (t_i,t_{i+1}) \ \forall i \in [k-1]$.
\end{definition}

\begin{definition}
	[Parametrizzazione regolare a tratti orientabile]
	Si definisce parametrizzazione regolare a tratti orientabile $\vb{r}: [a,b]\to\R^n$ continua se $\exists a=t_1<\cdots<t_k=b \tc \vb{r}:[a,b]\setminus \{t_1,\dots,t_k\} \injarrow \gamma, \ \vb{r}:[t_i,t_{i+1}]\bijarrow \vb{r}([t_i,t_{i+1}]) \in \C{1} \e \vb{r}'(t)\neq \vb{0} \ \forall t \in (t_i,t_{i+1}) \ \forall i \in [k-1]$.
\end{definition}

\begin{definition}
	[Punto di arresto]
	Sia $\vb{r} :[a,b] \to \R^n\in \C{1}$. $t \in [a,b]$ è punto di arresto se $\vb{r}'(t) = \vb{0}$.
\end{definition}

\section{Integrali curvilinei}

\begin{theorem}
	[Lunghezza di una curva]
	Se $\gamma$ è una curva regolare, allora la lunghezza di $\gamma$ è
	$$
		L_\gamma = \int_a^b\norm{\vb{r}'(t)}\dd t
	$$
	dove $\vb{r}: [a,b]\suarrow \gamma$ è una parametrizzazione regolare.
\end{theorem}

\begin{proof}
	% TODO
\end{proof}

\begin{prop}
	La lunghezza di una curva regolare non dipende dalla parametrizzazione.
\end{prop}

\begin{proof}
	% TODO
\end{proof}

\begin{prop}
	Tutte le curve regolari a tratti sono rettificabili, ovvero hanno lunghezza finita.
	\qed
\end{prop}

\begin{definition}
	[Integrale di linea di una funzione]
	Siano $f: A\subseteq\R^n \to \R$ continua con $A$ aperto, $\gamma \subseteq A$ una curva regolare (o regolare a tratti) e $\vb{r}:[a,b]\suarrow \gamma \in \C{1}$ una sua parametrizzazione, allora
	$$
		\int_\gamma f \dd s = \int_a^b f(\vb{r}(t))\norm{\vb{r}'(t)}\dd t
	$$
\end{definition}

\begin{definition}
	[Ascissa curvilinea]
	Sia $\gamma$ una curva rettificabile e $\vb{r}:[a,b]\bijarrow\gamma \in \C{1}$ una sua parametrizzazione. Sia
	$$
		s(t)=\int_a^t\norm{\vb{r}'(u)}\dd u
	$$
	$s(t)$ è detta ascissa curvilinea e $s^{-1}(t)$ è un cambio di parametrizzazione ammissibile per $\gamma$, essendo un diffeomorfismo crescente da $[0,L_\gamma]$ in $[a,b]$.
\end{definition}

\section{Lavoro}

\begin{definition}
	[Campo vettoriale]
	Sia $A\subseteq \R^n$ un aperto connesso. Una funzione $\vecf: A \to \R^n$ continua è detta campo vettoriale.
\end{definition}

\begin{definition}
	[$1$-forma differenziale]\label{def:1form}
	Si definisce 1-forma differenziale $\omega=\ip{\vecf(\vb{x})}{\dd \vb{x}}$.
\end{definition}

\begin{definition}
	[Lavoro]
	Sia $\gamma \subseteq A$ una curva regolare (a tratti) orientabile con orientamento $\hat{\bm\tau}$. Il lavoro di $\vecf$ lungo $\gamma$ è
	$$
		L_{\gamma,\hat{\bm\tau}}=\int_{\gamma,\hat{\bm\tau}}\omega=\int_{\gamma, \hat{\bm\tau}}\ip{\vecf(\vb{x})}{\dd \vb{x}}=\int_\gamma \ip{\vecf}{\hat{\bm\tau}}\dd s
	$$
	dove $s$ è l'ascissa curvilinea.
\end{definition}

\begin{remark}
	Se cambia l'orientamento da $(\gamma, \hat{\bm\tau})$ in $(\gamma, -\hat{\bm\tau})$, allora $L_{\gamma,-\hat{\bm\tau}}$=$-L_{\gamma,\hat{\bm\tau}}$.
\end{remark}

Se si considera la parametrizzazione $\vb{r}: [a,b]\suarrow \gamma$, dove si assume che l'orientamento indotto dalla parametrizzazione sia compatibile con $\hat{\bm\tau}$, allora il lavoro si può calcolare come segue:
$$
	L_{\gamma, \hat{\bm\tau}} = \int_a^b\ip{\vecf(\vb{r} (t))}{\frac{\dd \vb{r}}{\dd t} (t)}\dd t
$$

\begin{theorem}
	[Cambio di parametrizzazione e lavoro]
	Sia $\vb{r}:[a,b]\suarrow \gamma$ una parametrizzazione regolare che induce l'orientamento $\hat{\bm\tau}_r$ su $\gamma$ e sia $\bm\rho = \vb{r} \circ \varphi$ una nuova parametrizzazione con $\hat{\bm\tau}_\rho$ l'orientamento indotto da essa. Allora:
	\begin{enumerate} [a.]
		\item Se $\varphi$ è un diffeomorfismo crescente, $L_{\gamma, \hat{\bm\tau}_\rho} = L_{\gamma, \hat{\bm\tau}_r}$
		\item Se $\varphi$ è un diffeomorfismo crescente, $L_{\gamma, \hat{\bm\tau}_\rho} = -L_{\gamma, \hat{\bm\tau}_r}$
	\end{enumerate}
\end{theorem}

\begin{proof}
	% TODO
\end{proof}

\section{Campi vettoriali conservativi}

\begin{definition}
	[Campo vettoriale conservativo]
	Siano $A \subseteq \R^n$ un aperto connesso e $\vecf \in \C{0}(A,\R^n)$. $\vecf$ è un campo vettoriale conservativo se $\exists U \in \C{1}(A,\R) \tc$
	$$
		\vecf(\vb{x})=\grad U(\vb{x}) \ \forall \vb{x} \in A
	$$
	In tal caso $U$ è detto potenziale del campo $\vecf$.
\end{definition}

\begin{prop}
	Siano $\vecf \in \C{0}(A,\R^n)$ con $A$ aperto connesso e $U \in \C{1}(A,\R)$ un suo potenziale. Allora $V\in \C{1}(A,\R)$ è un potenziale di $\vecf \iff \exists k \in \R \tc V(\vb{x})=U(\vb{x}) + k \ \forall \vb{x} \in A$.
	\qed
\end{prop}

\begin{theorem}
	[Campi vettoriali conservativi e lavoro]
	Sia $\vecf \in \C{0}(A,\R^n)$ un campo conservativo definito in $A\subseteq\R^n$ aperto e connesso e sia $U \in \C{1}(A,\R)$ un suo potenziale. Allora, se $(\gamma,\hat{\bm\tau}) \subseteq A$ è una curva regolare a tratti orientabile con primo estremo $\vb{x}_i$ e secondo estremo $\vb{x}_f$,
	$$
		L_{\gamma, \hat{\bm\tau}}=U(\vb{x}_f)-U(\vb{x}_i)
	$$
\end{theorem}

\begin{proof}
	% TODO
\end{proof}

\begin{theorem}
	[Caratterizzazione dei campi vettoriali conservativi]
	Siano $A\subseteq\R^n$ un aperto connesso, $\vecf \in \C{0}(A,\R^n)$. Allora le seguenti affermazioni sono equivalenti:
	\begin{enumerate}
		\item $\vecf$ è un campo vettoriale conservativo
		\item Per ogni coppia di curve regolari a tratti orientate $(\gamma_1,\hat{\bm\tau_1}), (\gamma_2,\hat{\bm\tau_2})$ con estremi coincidenti e $\gamma_1,\gamma_2 \subseteq A$ vale
		$$
			L_{\gamma_1,\hat{\bm\tau_1}}=L_{\gamma_2,\hat{\bm\tau_2}}
		$$
		\item Per ogni curva chiusa regolare a tratti orientabile con sostegno $\gamma \subseteq A$ e orientamento $\hat{\bm\tau}$ vale $L_{\gamma, \hat{\bm\tau}}=0$
	\end{enumerate}
\end{theorem}

\begin{proof}
	% TODO
\end{proof}

\begin{definition}
	[Campo vettoriale irrotazionale]
	Siano $A\subseteq\R^n$ un aperto connesso e $\vecf \in \C{1}(A,\R^n)$. $\vecf$ è detto irrotazionale se ha matrice jacobiana simmetrica, ovvero se
	$$
		\frac{\partial f_i}{\partial x_j}(\vb{x})=\frac{\partial f_j}{\partial x_i}(\vb{x}) \ \forall \vb{x} \in A \ \forall i,j \in [n]
	$$
\end{definition}

\begin{theorem}
	Se $\vecf \in \C{1}(A,\R^n)$ con $A$ aperto connesso è conservativo, allora è irrotazionale.
\end{theorem}

\begin{proof}
	% TODO
\end{proof}

\begin{theorem}
	Sia $\vecf \in \C{1}(\R^2\setminus{(0,0)}, \R^2)$ un campo vettoriale irrotazionale. Se, detta $\gamma$ la circonferenza di raggio unitario centrata in $(0,0)$ e orientamento arbitrario, $L_{\gamma, \hat{\bm\tau}}=0$, allora $\vecf$ è conservativo.
\end{theorem}

\begin{proof}
	% TODO
\end{proof}

\begin{definition}
	[Insieme convesso]
	$A \subseteq \R^n$ aperto è un insieme convesso se $\forall \vb{x},\vb{y} \in A$ il segmento $[\vb{x},\vb{y}] \subseteq A$.
\end{definition}

\begin{definition}
	[Insieme stellato rispetto a un punto]
	Siano $A \subseteq \R^n$ e $\vb{x_0} \in A$. $A$ è stellato rispetto a $\vb{x_0}$ se $\forall \vb{x} \in A, [\vb{x_0},\vb{x}]\subseteq A$.
\end{definition}

\begin{definition}
	[Insieme semplicemente connesso]
	$A \subseteq \R^n$ è semplicemente connesso se ogni curva regolare semplice chiusa contenuta in $A$ può essere deformata con continuità a un punto in $A$ restando in $A$.
\end{definition}

\begin{remark}
	Convesso $\then$ stellato $\then$ semplicemente connesso $\then$ connesso per archi $\then$ connesso.
\end{remark}

\begin{lemma}
	[di Poincarè]
	Sia $A \subseteq \R^n$ aperto convesso oppure stellato rispetto a un punto oppure semplicemente connesso. Sia $\vecf\in \C{1}(A,\R^n)$ un campo vettoriale irrotazionale. Allora $\vecf$ è conservativo in $A$.
	\qed
\end{lemma}

\begin{corollary}
	Siano $A \subseteq \R^n$ un aperto connesso e $\vecf \in \C{1}(A,\R^n)$ un campo vettoriale irrotazionale. Allora $\forall B\subseteq A$ connesso oppure stellato oppure semplicemente connesso $\vecf$ è conservativo se ristretto a $B$.
\end{corollary}