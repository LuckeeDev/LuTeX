\chapter{Integrali di superficie}

\section{Superfici}

\begin{definition}
	[Aperto regolare in $\R^2$]
	$\Omega \subseteq \R^2$ aperto, connesso e limitato è un aperto regolare di $\R^2$ se $\partial \overline{\Omega}=\partial \Omega \e \partial \Omega$ è l'unione finita e disgiunta di curve regolari a tratti, semplici e chiuse.
\end{definition}

\begin{definition}
	[Superficie]
	Siano $\Omega \subseteq \R^2$ un aperto regolare e $\vb{r}:\overline{\Omega}\to \R^3 \in \C{1}$. Si definisce superficie $\Sigma=\vb{r}(\overline{\Omega})$.
	$\Sigma$ è detta regolare se $\rank{J_{\vb{r}}(u,v)} = 2 \ \forall (u,v) \in \Omega$.
	$\Sigma$ è detta semplice se $\vb{r}:\Omega \injarrow \Sigma$.
\end{definition}

\begin{remark}
	La superficie è la classe di equivalenza delle parametrizzazioni che si ottengono le une dalle altre per composizione con un diffeomorfismo $\C{1}$ fra aperti regolari di $\R^2$.
\end{remark}

\begin{definition}
	[Superficie orientabile]
	$\Sigma \in \R^3$ è una superficie orientabile se $\exists \vb{N}:\Sigma \to \R^3 \in \C{0}$ tale che $\vb{N}(x,y,z)\in N_{(x,y,z)}\Sigma \e \norm{\vb{N}(x,y,z)}=1 \ \forall (x,y,z) \in \Sigma$, ovvero se esiste un campo continuo di versori normali a $\Sigma$.
\end{definition}

\begin{remark}
	Diffeomorfismi crescenti, ovvero con $\det J_\varphi>0$, non variano l'orientamento della superficie. Diffeomorfismi decrescenti inducono l'orientamento opposto.
\end{remark}

Sia $\vb{r}:\overline{\Omega} \to \Sigma$ una parametrizzazione regolare. Le derivate parziali di $\vb{r}$ rispetto a $u \e v$ appartengono allo spazio tangente a $\Sigma$ in $\vb{r}(u,v)$, allora il loro prodotto vettoriale $\vb{n}(u,v)=\left(\frac{\partial \vb{r}}{\partial u}\wedge\frac{\partial \vb{r}}{\partial v}\right)(u,v) \in N_{\vb{r}(u,v)}\Sigma$ e si può definire il campo di versori normale alla superficie come
$$
	\vb{N}(u,v)=\frac{(\partial_u\vb{r}\wedge\partial_v\vb{r})(u,v)}{\norm{(\partial_u\vb{r}\wedge\partial_v\vb{r})(u,v)}}\in N_{\vb{r}(u,v)}\Sigma
$$
Applicando un diffeomorfismo $\bm\varphi$ all'aperto regolare $\Omega$ si vede che l'espressione di $\vb{N}$ dipende dal segno di $\det J_{\bm\varphi}$.

\begin{definition}
	[Superficie regolare con bordo]
	Siano $\Omega \subseteq \R^2$ un aperto regolare e $\vb{r} \in \C{1}(\overline{\Omega},\R^3)$. $\Sigma = \vb{r}(\overline{\Omega})$ è una superficie regolare con bordo se:
	\begin{enumerate}
		\item $\vb{r}:\overline{\Omega}\bijarrow\Sigma$
		\item $\rank J_{\vb{r}}(u,v) = 2 \ \forall (u,v) \in \overline{\Omega}$
	\end{enumerate}
\end{definition}

\begin{theorem}
	Ogni superficie regolare con bordo è orientabile. \qed
\end{theorem}

\begin{definition}
	[Frontiera orientata canonicamente]
	Sia $\Omega \subseteq \R^2$ un aperto regolare e siano $\vb{T}(u,v)\in T_{(u,v)}\Omega$ e $\vb{N}(u,v)=(T_2(u,v),-T_1(u,v))$. $(\partial \Omega, \vb{T})$ è orientata canonicamente se $\forall (u,v) \in \partial \Omega \ \exists \lambda > 0 \tc \forall \varepsilon \in (0,\lambda)$
	\begin{enumerate}[a.]
		\item $(u,v)+\varepsilon\vb{N}(u,v) \notin \overline{\Omega}$
		\item $(u,v)-\varepsilon\vb{N}(u,v) \in \Omega$
	\end{enumerate}
\end{definition}

\begin{definition}
	[Orientamento canonico di $\partial \Sigma$]
	Sia $\vb{r}: \overline{\Omega} \bijarrow \Sigma \in \C{1}$ tale che $\rank J_{\vb{r}}(u,v)=2 \ \forall (u,v) \in \overline{\Omega}$ una parametrizzazione che induce l'orientamento $\hat{\bm\nu}$ su $\Sigma$. Siano inoltre $\gamma_1 \subset \partial\Omega$ e $\bm\varphi: [a,b]\to \gamma_1 \in \C{1}$ una parametrizzazione regolare semplice a tratti che induce l'orientamento canonico $\vb{T}$ su $\gamma_1$.
	Allora $\vb{r}\circ \bm\varphi:[a,b]\to \partial\Sigma_1$ è una parametrizzazione regolare semplice a tratti che induce l'orientamento $\hat{\bm\tau}$ canonico rispetto a $\hat{\bm\nu}$.
\end{definition}

\begin{definition}
	[Superficie regolare a tratti]
	Siano $\Sigma_1,\dots,\Sigma_p$ superfici regolari con bordo, allora $\Sigma = \bigcup\limits_{j=1}^p \Sigma_j$ è una superficie regolare a tratti se $\forall i,j \in [p] \with i \neq j \ \Sigma_i \cap \Sigma_j \subseteq \bigcup\limits_{k=1}^p\partial \Sigma_k$.
\end{definition}

\begin{definition}
	[Superficie regolare a tratti orientabile]
	$\Sigma$ superficie regolare a tratti è detta orientabile se ciascuna componente di $\partial \Sigma_j$ è orientabile in modo tale che sugli spigoli comuni questi abbiano orientamento opposto.
\end{definition}

\begin{definition}
	[Superficie chiusa]
	Le superfici senza bordo e limitate in $\R^3$ si dicono chiuse. Una superficie regolare a tratti è chiusa se $\partial \Sigma = \varnothing$ e se è limitata.
\end{definition}

\begin{definition}
	[Area di una superficie]
	Sia $\vb{r}:\overline{\Omega} \bijarrow \Sigma \tc \rank J_{\vb{r}}(u,v) = 2 \ \forall (u,v) \in \overline{\Omega}$.
	$$
		\Area (\Sigma) = \iint_\Sigma \dd S = \iint_{\overline{\Omega}} \norm{\frac{\partial \vb{r}}{\partial u}\wedge \frac{\partial \vb{r}}{\partial v}}\dd u \dd v
	$$
	Inoltre, sia $f: \Sigma \to \R$ una funzione continua.
	$$
		\iint_\Sigma f \dd S = \iint_{\overline{\Omega}}(f\circ\vb{r})(u,v)\norm{\frac{\partial \vb{r}}{\partial u}\wedge \frac{\partial \vb{r}}{\partial v}}\dd u \dd v
	$$
\end{definition}

\begin{definition}
	[Flusso]
	Sia $A\subseteq \R^3$ un aperto e sia $\Sigma \subseteq A$ una superficie orientabile con orientamento $\hat{\bm \nu}$ indotto dalla parametrizzazione $\vb{r}:\overline{\Omega}\to\Sigma$. Sia inoltre $\vecf \in \C{0}(A,\R^3)$ un campo vettoriale. Il flusso di $\vecf$ attraverso $\Sigma$ è
	\begin{align*}
		\iint_\Sigma \ip{\vecf}{\hat{\bm\nu}}\dd S 
		&=\iint_{\overline{\Omega}}\ip{(\vecf\circ\vb{r})(u,v)}{\frac{(\partial_u\vb{r}\wedge\partial_v\vb{r})(u,v)}{\norm{(\partial_u\vb{r}\wedge\partial_v\vb{r})(u,v)}}}\norm{(\partial_u\vb{r}\wedge\partial_v\vb{r})(u,v)}\dd u \dd v=\\
		&= \iint_{\overline{\Omega}}\ip{(\vecf\circ\vb{r})(u,v)}{\frac{\partial \vb{r}}{\partial u} \wedge \frac{\partial \vb{r}}{\partial v}(u,v)}\dd u \dd v\\
	\end{align*}
\end{definition}

\begin{remark}
	Se $\vb{r}$ induce l'orientamento opposto,
	$$
		\iint_\Sigma \ip{\vecf}{\hat{\bm\nu}}\dd S = -\iint_{\overline{\Omega}}\ip{(\vecf\circ\vb{r}(u,v))}{\frac{\partial \vb{r}}{\partial u}\wedge \frac{\partial \vb{r}}{\partial v}}\dd u \dd v
	$$
\end{remark}

\section{Teorema di Stokes}

\begin{definition}
	[Rotore]
	Sia $\vecf\in \C{1}(A\subseteq \R^3,\R^3)$ con $A$ aperto un campo vettoriale. Si definisce rotore di $\vecf$
	$$
		\rot \vecf=\curl \vecf=\det
		\begin{bmatrix}
			\hat{\vb{i}} & \hat{\vb{j}} & \hat{\vb{k}}\\
			\partial_x & \partial_y & \partial_z\\
			f_1 & f_2 & f_3
		\end{bmatrix}
		\in \R^3
	$$
\end{definition}

\begin{theorem}
	[di Stokes o del rotore]
	Siano $A \subseteq \R^3$ un aperto, $\vecf \in \C{1}(A,\R^3)$, $\Sigma \subseteq A$ una superficie regolare con bordo con orientamento $\hat{\bm \nu}$ e $(\partial \Sigma,\hat{\bm \tau})$ il suo bordo con orientamento indotto canonicamente. Allora
	$$
		\iint_\Sigma \ip{\rot \vecf}{\hat{\bm \nu}}\dd \sigma = \int_{\partial\Sigma}\ip{\vecf}{\hat{\bm \tau}}\dd s
	$$
\end{theorem}

\begin{proof}
	[Dimostrazione in un caso particolare.]
	Si consideri il rettangolo $W=\{(x,y,0)\in[a,b]\times[c,d]\}$ posto nel piano $xy$ con orientamento $\hat{\vb{k}}$. Il flusso del rotore di $\vecf$ attraverso il rettangolo è
	\begin{align*}
		\iint_W \ip{\rot \vecf}{\hat{\bm \nu}}\dd \sigma
		&= \iint_{W}\left(\frac{\partial f_2}{\partial x}(x,y,0)-\frac{\partial f_1}{\partial y}(x,y,0)\right)\dd x \dd y=\\
		&=\int_c^d \dd y\int_a^b\frac{\partial f_2}{\partial x}(x,y,0)\dd x-
		\int_a^b\dd x\int_c^d\frac{\partial f_1}{\partial y}(x,y,0)\dd y=\\
		&=\int_c^d(f_2(b,y,0)-f_2(a,y,0))\dd y+\int_a^b(f_1(x,c,0)-f_1(x,d,0))\dd x
	\end{align*}
	Il lavoro di $\vecf$ sul bordo del rettangolo $W$ orientato canonicamente rispetto a $\hat{\vb{k}}$ è la somma dei lavori sui tratti del bordo, ovvero
	\begin{align*}
		&\int_a^bf_1(x,c,0)\dd x+\int_c^df_2(b,y,0)+\int_b^af_1(x,d,0)+\int_d^cf_2(a,y,0)=\\
		=&\int_a^b(f_1(x,c,0)-f_1(x,d,0))\dd x+\int_c^d(f_2(b,y,0)-f_2(a,y,0))\dd y
	\end{align*}
	I due integrali coincidono.
\end{proof}

\paragraph{Teorema di Stokes tramite le forme differenziali}

Il teorema di Stokes può essere espresso anche tramite l'integrazione di una forma differenziale d'area (o 2-forma), ricavata applicando l'operatore differenziale esterno alla 1-forma associata al lavoro (Def. \ref{def:1form}, Cap. \ref{chap:curves}) usando le tre regole seguenti:
\begin{enumerate}[a.]
	\item $\dd(\dd \alpha)=0$
	\item $\dd \alpha \wedge \dd \alpha=0$
	\item $\dd \alpha \wedge \dd \beta = -\dd \beta \wedge \dd \alpha$
\end{enumerate}
\begin{align*}
	\dd \omega =& \dd (f_1 \dd x+f_2 \dd y+f_3 \dd z)=\\
	&=\dd f_1\wedge \dd x +\dd f_2 \wedge \dd y +\dd f_3 \wedge \dd z=\\
	&=\left( \frac{\partial f_1}{\partial x}\dd x + \frac{\partial f_1}{\partial y}\dd y + \frac{\partial f_1}{\partial z}\dd z \right)\wedge \dd x+\\
	&+\left( \frac{\partial f_2}{\partial x}\dd x + \frac{\partial f_2}{\partial y}\dd y + \frac{\partial f_2}{\partial z}\dd z \right)\wedge \dd y +\\ 
	&+\left(\frac{\partial f_3}{\partial x}\dd x + \frac{\partial f_3}{\partial y}\dd y + \frac{\partial f_3}{\partial z}\dd z\right) \wedge \dd z= \\
	&=\left(\frac{\partial f_3}{\partial y}-\frac{\partial f_2}{\partial z}\right)\dd y \wedge \dd z+
	\left(\frac{\partial f_1}{\partial z}-\frac{\partial f_3}{\partial x}\right)\dd z \wedge \dd x+
	\left(\frac{\partial f_2}{\partial x}-\frac{\partial f_1}{\partial y}\right)\dd x \wedge \dd y
	=\\
	&=\ip{\rot \vecf}{\hat{\bm \nu}} \dd u \wedge \dd v
\end{align*}
Di conseguenza,
$$
	\iint\limits_{\Sigma,\hat{\bm \nu}} \dd \omega = \int\limits_{\partial \Sigma,\hat{\bm \tau}} \omega
$$

\begin{remark}
	Per passare da $\dd u \wedge \dd v$ a $\dd u \dd v$ nell'integrale è necessario verificare l'orientamento indotto dalla parametrizzazione utilizzata. Se la parametrizzazione induce l'orientamento corretto, $\dd u \wedge \dd v = \dd u \dd v$, altrimenti $\dd u \wedge \dd v= -\dd u \dd v$.  
\end{remark}

\section{Teorema di Gauss}

\begin{definition}
	[Aperto regolare in $\R^3$]
	$A\subseteq \R^3$ è un aperto regolare se è aperto, limitato, connesso, $\mathring{\overline A}=A \e \partial A$ è l'unione finita di superfici regolari a tratti chiuse e orientabili a due a due disgiunte. Se $A \subseteq \R^3$ è un aperto regolare, $\partial A$ è orientata canonicamente se $\forall (x,y,z) \in \partial A \ \exists \varepsilon > 0 \tc \forall \lambda \in (0,\varepsilon)$
	\begin{enumerate}[a.]
		\item $(x,y,z) + \lambda \hat{\bm \nu}(x,y,z) \notin \overline{A}$
		\item $(x,y,z) - \lambda \hat{\bm \nu}(x,y,z) \in A$
	\end{enumerate}
\end{definition}

\begin{definition}
	[Divergenza]
	Sia $\vecf \in \C{1}(A\subseteq\R^3,\R^3)$ un campo vettoriale. Si definisce divergenza di $\vecf$
	$$
		\divop \vecf=\divergence \vecf=\frac{\partial f_1}{\partial x} + \frac{\partial f_2}{\partial y} + \frac{\partial f_3}{\partial z} \in \R
	$$
\end{definition}

\begin{theorem}
	[di Gauss o della divergenza]
	Siano $A\subseteq \R^3$ un aperto regolare, $\vecf \in \C{1}(\overline{A},\R^3) \e (\partial A, \hat{\bm \nu})$ la frontiera di $A$ orientata canonicamente. Allora
	$$
		\iiint_A \divop \vecf(x,y,z)\dd x \dd y \dd z = \iint_{\partial A}\ip{\vecf}{\hat{\bm \nu}}\dd \sigma
	$$
\end{theorem}

\begin{proof}
	[Giustificazione in un caso semplice.]
	Si consideri il cubo $Q=\{(x,y,z)\in[0,L]\times[0,L]\times[0,L]\}$. Il flusso del campo $\vecf$ attraverso le facce di $\partial Q$ parallele al piano $xz$ è
	$$
		\iint_{\partial Q_y}\ip{\vecf}{\hat{\bm\nu}}=-\iint_{[0,L]\times[0,L]}f_2(x,0,z)\dd x \dd z
		+\iint_{[0,L]\times[0,L]}f_2(x,L,z)\dd x \dd z
	$$
	L'integrale del termine della divergenza dipendente da $y$ è
	$$
		\iiint_{Q}\frac{\partial f_2}{\partial y}(x,y,z)\dd x \dd y \dd z=\iint_{[0,L]\times[0,L]}(f_2(x,L,z)-f_2(x,0,z))\dd x \dd z
	$$
	I due integrali coincidono ed è possibile applicare lo stesso ragionamento alle variabili $x$ e $z$.
\end{proof}

\paragraph{Teorema di Gauss tramite le forme differenziali}

Anche nel caso del teorema di Gauss è possibile esprimere l'enunciato del teorema in termini di forme differenziali. Sia $\omega = f_1 \dd y \wedge \dd z + f_2 \dd z \wedge \dd x + f_3 \dd x \wedge \dd y$ la forma differenziale d'area associata al flusso del campo $\vecf$. Applicando l'operatore differenziale esterno, si ottiene che
\begin{align*}
	\dd \omega &=\dd(f_1 \dd y \wedge \dd z + f_2 \dd z \wedge \dd x + f_3 \dd x \wedge \dd y)=\\
	&=\dd f_1 \wedge \dd y \wedge \dd z+ \dd f_2\wedge\dd z \wedge \dd x+\dd f_3\dd\wedge x \wedge \dd y=\\
	&= \left(\frac{\partial f_1}{\partial x} + \frac{\partial f_2}{\partial y} + \frac{\partial f_3}{\partial z}\right)\dd x \wedge \dd y \wedge \dd z\\
\end{align*}
Di conseguenza, si può scrivere
$$
	\iiint\limits_A \dd \omega = \iint\limits_{\partial A^+}\omega
$$
dove $\partial A^+$ è la frontiera di $A$ orientata canonicamente.

\begin{remark}
	Si noti che la somiglianza fra il teorema di Gauss e il teorema di Stokes è dovuta al fatto che essi sono casi particolari dello stesso teorema applicato a dimensioni diverse.
\end{remark}