\chapter{Teoria della misura e dell'integrazione}\label{chap:peano_jordan}

\section{Misura di Peano-Jordan}

\begin{definition}
    [Intervallo semi-aperto superiormente]
    Si definisce intervallo semi-aperto superiormente di $\R^n$ il prodotto cartesiano di $n$ intervalli chiusi inferiormente e aperti superiormente:
    $$
        I=[a_1,b_1)\times\cdots\times[a_n,b_n) \with a_i \leq b_i \ \forall i \in [n]
    $$
    La misura elementare di $I$ è
    $$
        \mu_n(I)=\prod_{j=1}^n(b_j-a_j)
    $$
\end{definition}

\begin{definition}
    [Pluri-intervallo]
    Siano $I_1,\dots,I_k$ intervalli semi-aperti superiormente a due a due disgiunti. Si definisce pluri-intervallo l'unione $P=\bigcup\limits_{j=1}^kI_j$. La misura di $P$ è $\mu_n(P)=\sum\limits_{j=1}^k\mu_n(I_j)$. 
    L'insieme dei pluri-intervalli di $\R^n$ viene indicato con $\mathcal{P}$.
\end{definition}

\begin{lemma}
    Valgono le seguenti proposizioni:
    \begin{itemize}
        \item Se $P_1,\dots,P_k \in \mathcal{P}$, allora $\bigcup\limits_{j=1}^k P_j, \bigcap\limits_{j=1}^kP_j \in \mathcal{P}$
        \item Se $P_1,P_2 \in \mathcal{P}$, allora $P_2 \setminus P_1 \in \mathcal{P}$
        \qed
    \end{itemize}
\end{lemma}

\begin{theorem}
    [Caratterizzazione della misura in $\mathcal{P}$]
    Per la misura in $\mathcal{P}$ valgono le seguenti proprietà:
    \begin{enumerate}
        \item (Additività finita) Se $P_1,\dots,P_k \in \mathcal{P}$ sono a due a due disgiunti, allora
        $$
            \mu_n\left( \bigcup_{j=1}^k P_j \right) = \sum_{j=1}^k \mu_n(P_j)
        $$
        \item (Monotonia) Se $P, Q \in \mathcal{P} \e P \subseteq Q \then \mu_n(P)\leq \mu_n(Q)$
        \item (Modularità) Se $P,Q \in \mathcal{P}$, allora
        $$
            \mu_n(P\cup Q) + \mu_n (P \cap Q) = \mu_n (P) + \mu_n (Q)
        $$
        \item (Sottrattività) Se $P,Q \in \mathcal{P} \then \mu_n(P\setminus Q)=\mu_n(P)-\mu_n(P\cap Q)$. In particolare, se $Q \subseteq P$, allora $\mu_n(P\setminus Q) = \mu_n (P)-\mu_n(Q)$
        \qed
    \end{enumerate}
\end{theorem}

\begin{corollary}
    Se $P_1,\dots,P_k \in \mathcal{P}$, allora $\mu_n\left(\bigcup\limits_{j=1}^kP_j\right) \leq \sum\limits_{j=1}^k\mu_n(P_j)$. Ovvero, vale la proprietà di subadditività finita per $\mu_n$ su $\mathcal{P}$.
    \qed
\end{corollary}

\begin{lemma}
    Se $P \in \mathcal{P}$, allora $\partial P \in \mathcal{P} \e \mu_n(\partial P)=0$.
    \qed
\end{lemma}

\begin{corollary}
    Se $P \in \mathcal{P}$, allora $\mathring P, \overline{P}$ sono misurabili e $\mu_n(\mathring P)=\mu_n(\overline{P})=\mu_n(P)$.
    \qed
\end{corollary}

\begin{definition}
    [Misura interna ed esterna]
    Si definiscono gli insiemi dei pluri-intervalli interni all'insieme $X$, $\mathcal{P}^{(i)}$, e dei pluri-intervalli esterni all'insieme $X$, $\mathcal{P}^{(e)}$:
    \begin{align*}
        \mathcal{P}^{(i)}&=\{P\in \mathcal{P}:P\subseteq X\}\\
        \mathcal{P}^{(e)}&=\{Q \in \mathcal{P}: Q \supseteq X \}
    \end{align*}
    La misura interna di $X$ è $
        \mu_n^{(i)}(X)=\sup\limits_{P\in \mathcal{P}^{(i)}}\mu_n(P)
    $
    e la misura esterna di $X$ è $
        \mu_n^{(e)}(X)=\inf\limits_{Q \in \mathcal{P}^{(e)}}\mu_n(Q)
    $
\end{definition}

\begin{lemma}
    $\mu_n^{(i)}(X), \mu_n^{(e)}(X) \in \R_{\geq 0} \e \mu_n^{(i)}(X)\leq\mu_n^{(e)}(X)$.
    \qed
\end{lemma}

\begin{definition}
    [Insieme limitato misurabile secondo Peano-Jordan]
    Sia $X \subseteq \R^n$ un insieme limitato. Si dice che $X$ è misurabile secondo Peano-Jordan se $\mu_n^{(i)}(X)=\mu_n^{(e)}(X)$ e in tal caso si dice misura $n$-dimensionale di $X$ tale valore.
    L'insieme degli insiemi limitati misurabili secondo Peano-Jordan si indica con $\J_b(\R^n)$.
\end{definition}

\begin{lemma} Valgono le seguenti proposizioni:
    \begin{enumerate}
        \item $\mathcal{P} \subseteq \J_b(\R^n)$. Tutti i pluri-intervalli sono misurabili secondo Peano-Jordan e la misura coincide con quella definita in modo elementare.
        \item Se $X_1,\dots,X_k \in \J_b(\R^n)$, allora $\bigcup\limits_{j=1}^k X_j, \bigcap\limits_{j=1}^kX_j \in \J_b(\R^n)$
        \item Se $X,Y \in \J_b(\R^n)\then X \setminus Y \in \J_b(\R^n)$
        \qed
    \end{enumerate}
\end{lemma}

\begin{theorem}
    [Caratterizzazione della misurabilità]
    Sia $X\subseteq \R^n$ limitato. Allora:
    \begin{enumerate}
        \item $X\in \J_b(\R^n) \iff \forall \varepsilon > 0 \ \exists P,Q \in \mathcal{P} \tc P \subseteq X \subseteq Q \e \mu_n(Q)-\mu_n(P) < \varepsilon$
        \item $X \in \J_b(\R^n) \iff \forall \varepsilon > 0 \ \exists Y,Z \in \J_b(\R^n) \tc Y \subseteq X \subseteq Z \e \mu_n(Z)-\mu_n(Y) < \varepsilon$
        \item $X \in \J_b(\R^n)\iff \partial X \in \J_b(\R^n) \e \mu_n(\partial X)=0$
        \qed
    \end{enumerate}
\end{theorem}

\begin{corollary}
    Se $X \in \J_b(\R^n)$, allora $\overline{X},\mathring X \in \J_b(\R^n) \e \mu_n(X)=\mu_n(\overline{X})=\mu_n(\mathring X)$.
    \qed
\end{corollary}

\begin{theorem}
    [Proprietà di $\J_b(\R^n)$]
    Valgono le seguenti proprietà per la misura in $\J_b(\R^n)$:
    \begin{enumerate}
        \item (Additività finita) Se $X_1,\dots,X_k \in \J_b(\R)^n$ sono a due a due disgiunti, allora
        $$
            \mu_n\left(\displaystyle \bigcup_{j=1}^kX_j\right) =   \displaystyle\sum_{j=1}^k\mu_n(X_j)
        $$
        \item (Modularità) Se $X,Y \in \J_b(\R^n)$, allora $\mu_n(X\cup Y)+\mu_n(X\cap Y)=\mu_n(X)+\mu_n(Y)$
        \item (Monotonia) Se $X,Y \in \J_b(\R^n) \e X \subseteq Y$, allora $\mu_n(X) \leq \mu_n(Y)$
        \item (Sottrattività) Se $X,Y \in \J_b(\R^n)$, allora $\mu_n(X\setminus Y)=\mu_n(X)-\mu_n(Y)$. In particolare, se $Y \subseteq X \then \mu_n (X \setminus Y)=\mu_n(X)-\mu_n(Y)$
        \qed
    \end{enumerate}
\end{theorem}

\begin{theorem}
    [Caratterizzazione degli insiemi di misura nulla]\leavevmode
    \begin{enumerate}
        \item $\mathring X = \varnothing \iff \mu_n^{(i)}(X)=0$
        \item $X \in \J_b(\R^n) \e \mu_n(X)=0 \iff \forall \varepsilon >0 \ \exists P \in \mathcal{P} \tc P \supseteq X \e \mu_n(P)< \varepsilon$
        \item $X \in \J_b(\R^n) \e \mu_n(X)=0 \iff \mathring X = \varnothing \e X \in \J_b(\R^n)$
        \qed
    \end{enumerate}
\end{theorem}

\begin{definition}
    [Insieme misurabile secondo Peano-Jordan]
    Sia $X\subseteq \R^n$. $X$ è misurabile secondo Peano-Jordan, ovvero $X \in \J(\R^n)$, se $\forall Y \in \J_b(\R^n), \ Y \cap X \in \J_b(\R^n)$. In tal caso si definisce $\mu_n(X)=\sup\{\mu_n(X\cap Y):Y\in \J_b(\R^n)\}$.
\end{definition}

\begin{remark}
    Questa definizione allarga la nozione di misura agli insiemi non limitati. Un insieme misurabile in $\J_b(\R^n)$ ha la stessa misura se misurato in $\J(\R^n)$.
\end{remark}

\begin{theorem}
    Se $X \in \J(\R^n) \e \{X_k\}_{k \in \N}$ è una successione crescente di elementi in $\J_b(\R^n)$, ovvero:
    \begin{enumerate}
        \item $\forall k \in \N, \ X_k \in \J_b(\R^n)$
        \item $X_k \subseteq X_{k+1}$
        \item $\bigcup\limits_{k=1}^{\infty}X_k = \R^n$,
    \end{enumerate}
    allora $\mu_n(X)=\lim\limits_{k\to + \infty}\mu_n(X\cap X_k)$.
    \qed
\end{theorem}

\begin{theorem}
    $\mu_n: \J(\R^n) \to [0, +\infty]$ e inoltre:
    \begin{enumerate}
        \item $\varnothing \in \J(\R^n)$
        \item Se $X \in \J(\R^n)$, allora $\R^n \setminus X \in \J(\R^n)$
        \item Se $X, Y \in \J(\R^n)$, allora $X \cup Y, X \cap Y, X \setminus Y \in \J(\R^n)$
        \qed
    \end{enumerate}
\end{theorem}

\begin{theorem}
    [Proprietà di $\J(\R^n)$]\label{thm:prop_J}
    $\J(\R^n)$ gode delle seguenti proprietà:
    \begin{enumerate}
        \item (Additività finita) Se $X_1,\dots,X_k \in \J(\R^n), \ X_i \cap X_j = \varnothing \ \forall i, j \in [k]$, allora $\bigcup\limits_{j=1}^kX_j \in \J(\R^n)$ e
        $$
            \mu_n\left(\bigcup_{j=1}^kX_j\right) = \sum_{j=1}^k\mu_n(X_j)
        $$
        \item (Monotonia) Se $X, Y \in \J(\R^n) \e X \subseteq Y$, allora $\mu_n(X) \leq \mu_n(Y)$
        \item (Modularità) Se $X, Y \in \J(\R^n)$, allora
        $$
            \mu_n(X \cup Y) + \mu_n (X \cap Y) = \mu_n(X) + \mu_n(Y)
        $$
        \item (Sottrattività) Se $X, Y \in \J(\R^n) \e \mu_n(Y) < + \infty$, allora
        $$
            \mu_n(X \setminus Y) = \mu_n(X) - \mu_n(X \cap Y)
        $$
        \qed
    \end{enumerate}
\end{theorem}

\begin{remark}
    Nel punto \textit{3.} del teorema \ref{thm:prop_J} è importante non spostare termini dell'espressione da una parte all'altra dell'uguaglianza. Essendo $X \e Y$ insiemi non necessariamente limitati, si rischierebbe di avere una sottrazione $\infty - \infty$.
\end{remark}

\begin{lemma}
    $X \in \J(\R^n) \iff \partial X \in \J(\R^n) \e \mu_n(\partial X)=0$.
    \qed
\end{lemma}

\section{Integrazione secondo Riemann}

\begin{definition}
    [Funzione non negativa integrabile secondo Riemann]
    Siano $A \subseteq \R^n, \ A \in \J(\R^n), \ f: A \to \R$ limitata e tale che $f \geq 0$. Si definisce sottografico di $f$ l'insieme $R(f)=\{(\vb{x}, y)\in A \times \R_{\geq 0} : 0 \leq y \leq f(\vb{x})\}$.
    $f$ si dice integrabile secondo Riemann se $R(f) \in \J(\R^{n+1})$ e in tal caso
    $$
        \idotsint_A f(x_1,\dots,x_n)\dd x_1\cdots \dd x_n=\mu_{n+1}(R(f))
    $$
    $f$ è detta sommabile se $f$ è integrabile e $\idotsint_Af < +\infty$.
\end{definition}

\begin{definition}
    [Parte positiva e parte negativa]\label{def:ppos_pneg}
    Se $f:A \subseteq \R^n \to \R$, si definiscono:
    \begin{itemize}
        \item Parte positiva di $f$, $f_+(\vb{x})=\max\{0,f(\vb{x})\}$
        \item Parte negativa di $f$, $f_-(\vb{x})=\max\{0,-f(\vb{x})\}$
    \end{itemize}
    In base a questa definizione, $f(\vb{x})=f_+(\vb{x})-f_-(\vb{x})$ e $\abs{f(\vb{x})}=f_+(\vb{x})+f_-(\vb{x})$.
\end{definition}

\begin{definition}
    [Funzione integrabile secondo Riemann]
    Siano $A \in \J(\R^n) \e f: A \to \R$ limitata. Se $f_+ \e f_-$ sono integrabili e almeno uno fra $\idotsint_A f_+, \idotsint_A f_-$ è un valore finito, allora $f$ si dice integrabile secondo Riemann e
    $$
        \idotsint_Af(x_1,\dots,x_n)\dd x_1\cdots \dd x_n = \idotsint_A f_+(x_1,\dots,x_n)\dd x_1\cdots \dd x_n - \idotsint_A f_-(x_1,\dots,x_n)\dd x_1\cdots \dd x_n
    $$
    Inoltre, se $f_+ \e f_-$ sono sommabili, allora $f$ è sommabile.
\end{definition}

\begin{theorem}
    Sia $f: A \to \R$ integrabile in $A \in \J(\R^n)$. Allora
    $$
        \abs{\idotsint_A f(x_1,\dots,x_n)\dd x_1\cdots \dd x_n} \leq \idotsint_A\abs{f(x_1,\dots,x_n)}\dd x_1\cdots \dd x_n
    $$
    \qed
\end{theorem}

\begin{theorem}
    [Proprietà delle funzioni sommabili]
    Siano $A \in \J(\R^n), \ f,g: A \to \R$ limitate e sommabili secondo Riemann. Valgono le seguenti proprietà:
    \begin{enumerate}
        \item (Linearità) $\forall a,b \in \R, \ af + bg$ è sommabile e $\idotsint_A(af+bg)=a\idotsint_A f+b\idotsint_A g$
        \item (Monotonia) Se $\forall \vb{x} \in A \ f(\vb{x}) \leq g(\vb{x})$, allora $\idotsint_A f \leq \idotsint_A g$
        \item (Additività) Se $\mu_n(A) < + \infty \e A_1,A_2 \subseteq A \tc A_1 \cup A_2 = A \e A_1,A_2 \in \J(\R^n) \e \mu_n(A_1 \cap A_2)=0$, allora $\idotsint_A f= \idotsint_{A_1} f+ \idotsint_{A_2} f$
        \qed
    \end{enumerate}
\end{theorem}

\begin{theorem}
    [della media integrale]
    Siano $A \subseteq \R^n, \ A \in \J(\R^n), \ \mu_n(A) < +\infty \e f: A \to \R$ limitata e sommabile. Allora:
    $$
        \inf_A f \leq \frac{\idotsint_A f}{\mu_n(A)}\leq \sup_A f
    $$
    \qed
\end{theorem}

\begin{corollary}
    Sotto le ipotesi del teorema, se $A \subseteq \R^n$ è compatto e connesso e $f: A \to \R$ è continua, allora $\exists \vb{c} \in A \tc \displaystyle\frac{\idotsint_A f}{\mu_n(A)}=f(\vb{c})$.
    \qed
\end{corollary}

\subsection{Integrali doppi}

\begin{theorem}
    [di riduzione sui rettangoli]
    Sia $K=[a,b]\times[c,d] \e f\in \C{1}(K,\R)$, allora:
    \begin{enumerate}
        \item $G:[c,d]\to \R$ definita come $G(y)=\int_a^bf(x,y)\dd x$ è continua e $\iint_Kf(x,y)\dd x\dd y=\int_c^dG(y)\dd y=\int_c^d\dd y(\int_a^b f(x,y)\dd x)$
        \item $F:[a,b]\to \R$ definita come $F(x)=\int_c^df(x,y)\dd y$ è continua e $\iint_Kf(x,y)\dd x\dd y=\int_a^bF(x)\dd x=\int_a^b\dd x(\int_c^d f(x,y)\dd y)$
        \item Se $f(x,y)=g(x)h(y)$, allora
        $$
            \iint_Kf(x,y)\dd x\dd y=\iint_Kg(x)h(y)\dd x\dd y=\left(\int_a^bg(x)\dd x\right)\left(\int_c^dh(y)\dd y\right)
        $$
        \qed
    \end{enumerate}
\end{theorem}

\begin{definition}
    [Dominio normale rispetto a un asse]\leavevmode
    \begin{enumerate}[a.]
        \item Siano $\varphi, \psi \in \C{0}([c,d],\R) \tc \varphi(y)\leq \psi(y)\ \forall y \in [c,d]$. Allora
        $$
            A=\{(x,y)\in \R\times[c,d]:\varphi(y)\leq x \leq \psi(y)\}
        $$
        è un dominio normale rispetto all'asse $x$.
        \item Siano $g, h \in \C{0}([a,b],\R) \tc g(x)\leq h(x)\ \forall x \in [a,b]$. Allora
        $$
            B=\{(x,y)\in [a,b]\times\R:g(x)\leq y \leq h(x)\}
        $$
        è un dominio normale rispetto all'asse $y$. 
    \end{enumerate}
\end{definition}

\begin{theorem}
    [di riduzione degli integrali doppi su domini normali]\leavevmode
    \begin{enumerate}[a.]
        \item Siano $\varphi, \psi \in \C{0}([c,d],\R), \ A=\{(x,y)\in \R\times[c,d]:\varphi(y)\leq x \leq \psi(y)\} \e f \in \C{0}(A,\R)$. Allora
        $$
            \iint_Af(x,y)\dd x\dd y = \int_c^d\dd y\left(\int_{\varphi(y)}^{\psi(y)}f(x,y)\dd x\right)
        $$
        \item Siano $g, h \in \C{0}([a,b],\R), \ A=\{(x,y)\in \R\times[a,b]:g(x)\leq y \leq h(x)\} \e f \in \C{0}(A,\R)$. Allora
        $$
            \iint_Af(x,y)\dd x\dd y = \int_a^b\dd x\left(\int_{g(x)}^{h(x)}f(x,y)\dd y\right)
        $$
        \qed
    \end{enumerate}
\end{theorem}

\subsection{Integrali tripli}

\begin{definition}
    [Solido di Cavalieri]
    Sia $K \subseteq \R^3$ compatto e misurabile. Se esiste un asse $\hat{\lambda}$ tale che
    \begin{enumerate}[a.]
        \item $\forall \lambda \in [a,b], \ \sez_\lambda(K)$ è un insieme misurabile e
        \item $\forall \lambda < a \e \lambda > b, \ \sez_\lambda (K) = \varnothing$,
    \end{enumerate}
    allora $K$ è detto solido di Cavalieri.
\end{definition}

\begin{axiom}
    [di Cavalieri]
    Se $V$ e $W$ sono solidi di Cavalieri rispetto allo stesso asse $\hat{\lambda}$, $\mu_2(\sez_\lambda(V))\leq\mu_2(\sez_\lambda(W)) \ \forall \lambda \in [a,b]$ e $\mu_2(\sez_\lambda(V))=\mu_2(\sez_\lambda(W))=0 \ \forall \lambda \notin [a,b]$, allora $\mu_3(V)\leq \mu_3(W)$. Inoltre, se $\mu_2(\sez_\lambda(V))=\mu_2(\sez_\lambda(W))$, allora $\mu_3(V)=\mu_3(W)$.
\end{axiom}

\begin{theorem}
    [di Cavalieri]
    Sia $K \in \J_b(\R^3)$ compatto un solido di Cavalieri rispetto all'asse $\hat{z}$. Allora
    $$
        \mu_3(K)=\int_a^b\dd z\mu_2(\sez_z(K))=\int_a^b\dd z\left(\iint_{\sez_z(K)}\dd x\dd y\right)    
    $$
    Inoltre, se $f \in \C{0}(K,\R)$, allora
    $$
        \iiint_Kf(x,y,z)\dd x\dd y\dd z=\int_a^b\dd z\left(\iint_{\sez_z(K)}f(x,y,z)\dd x\dd y\right)
    $$
    \qed
\end{theorem}

\begin{definition}
    [Domino normale rispetto a un asse in $\R^3$]\label{def:dom_r3}
    Siano $K\subseteq \R^2$ compatto e misurabile, $\varphi,\psi \in \C{0}(K,\R) \tc \varphi(x,y)\leq \psi(x,y) \ \forall (x,y) \in K$. Si dice dominio normale rispetto all'asse $\hat{z}$ l'insieme $A=\{(x,y,z)\in K \times \R : \varphi(x,y)\leq z \leq \psi(x,y)\}$.
\end{definition}

\begin{theorem}
    [di riduzione degli integrali tripli su domini normali]
    Siano $\varphi, \psi, K, A$ come nella definizione \ref{def:dom_r3}. Allora
    $$
        \mu_3(A)=\iint_K\dd x\dd y(\psi(x,y)-\varphi(x,y))
    $$
    Inoltre, se $f \in \C{0}(A,\R)$,
    $$
        \iiint_Af(x,y,z)\dd x\dd y\dd z = \iint_K\dd x\dd y\left(\int_{\varphi(x,y)}^{\psi(x,y)}\dd zf(x,y,z)\right)
    $$
\end{theorem}

\subsection{Cambiamento di variabile nell'integrale multiplo}

Per comprendere il significato del cambiamento di variabile nell'integrale multiplo, è opportuno ricordare le principali caratteristiche delle applicazioni lineari.

\begin{definition}
    [Applicazione lineare]
    Sia $\vb{A}:\R^n\to\R^n$ un'applicazione. Essa è detta lineare se possiede le seguenti proprietà:
    \begin{enumerate}
        \item $\vb{A}(\vb{u}+\vb{v})=\vb{A}(\vb{u})+\vb{A}(\vb{v}) \ \forall \vb{u},\vb{v}\in \R^n$
        \item $\vb{A}(\lambda\vb{v})=\lambda \vb{A}(\vb{v}) \ \forall \lambda \in \R, \ \forall \vb{v} \in \R^n$ 
    \end{enumerate}
    Inoltre, detta $M_A \in \mathcal{M}_{n\times n}$ la matrice associata all'applicazione $\vb{A}$, il valore assoluto del suo determinante rappresenta il fattore di cui viene riscalata la misura di un qualsiasi $Q\subseteq \R^n$ attraverso l'applicazione.
    $$\mu_n(\vb{A}(Q))=\abs{\det M_A}\mu_n(Q)$$
\end{definition}

\begin{remark}
    Si noti che nel caso lineare $\det M_A \neq 0 \then A$ biunivoca da $\R^n$ in $\R^n$.
\end{remark}

Nel caso non lineare, quindi con un'applicazione $\bm\Phi$ qualsiasi, non esiste un fattore di scala valido per tutto lo spazio, ma localmente è possibile approssimare la trasformazione con una trasformazione lineare. Il fattore di scala infinitesimo è rappresentato dal determinante della matrice jacobiana dell'applicazione. Inoltre non vale l'osservazione appena fatta ed è quindi necessario richiedere sia che $\bm\Phi$ sia iniettiva sia che abbia $\det J_{\bm\Phi} \neq 0$.

\begin{theorem}
    [Cambiamento di variabile nell'integrale multiplo]\label{thm:int_var}
    Siano $A \subseteq \R^n$ aperto, $\bm\Phi \in \C{1}(A,\R^n)$, $\bm\Phi$ iniettiva e $\det J_{\bm\Phi} \neq 0$ in $A$. Allora, se $K \subseteq A$ è compatto e misurabile, anche $\bm\Phi(K)$ è compatto e misurabile e vale
    $$
        \mu_n(\bm\Phi(K))=\idotsint_K \abs{\det J_{\bm\Phi}(u_1,\dots,u_n)}\dd u_1 \cdots \dd u_n
    $$
    Inoltre, se $f \in \C{0}(\bm\Phi(K),\R)$, allora
    $$
        \idotsint_{\bm\Phi(K)}f(x_1,\dots,x_n)\dd x_1 \cdots \dd x_n=\idotsint_K (f \circ \bm\Phi)(u_1,\dots,u_n)\abs{\det J_{\bm\Phi}(u_1,\dots,u_n)}\dd u_1\cdots \dd u_n
    $$
    \qed
\end{theorem}

\begin{remark}
    Il teorema del cambiamento di variabile continua a valere nel caso di perdita di iniettività o di $\det J_{\bm\Phi}\neq 0$ su insiemi di misura nulla che sono trasformati in insiemi di misura nulla.
\end{remark}

\paragraph{Coordinate polari nel piano}
La trasformazione in coordinate polari nel piano $\bm\Phi:[0,+\infty)\times[0,2\pi]\to\R^2$ è definita come segue:
$$\bm\Phi(\rho,\theta)=
\begin{cases}
    x=\rho \cos \theta\\
    y=\rho \sin \theta
\end{cases}
$$

Il determinante della matrice jacobiana di $\bm\Phi(\rho,\theta)$ è $\rho$. Di conseguenza, $\det J_{\bm\Phi}=0$ solo sul segmento $\{0\}\times[0,2\pi]$, che ha misura nulla nello spazio $(\rho,\theta)$ e che viene trasformato nel punto $(0,0)\in\R^2$, che ha misura nulla in $\R^2$.
Inoltre $\bm\Phi(\rho,\theta)$ non è iniettiva sul segmento $\{0\}\times[0,2\pi]$ e sull'insieme $[0,+\infty)\times\{0,2\pi\}$, che hanno entrambi misura nulla nello spazio $(\rho,\theta)$. Il primo viene trasformato nel punto $(0,0)\in\R^2$ e il secondo nel semiasse $x$ positivo, ovvero $[0,+\infty)\times\{0\} \in \R^2$. Entrambi hanno misura nulla in $\R^2$.

Per quanto detto, $\bm\Phi$ è un cambiamento di variabile ammissibile per il teorema \ref{thm:int_var}. Sia $f\in \C{0}(\bm\Phi(K),\R^2)$ con $K\subseteq[0,+\infty)\times[0,2\pi]$ compatto e misurabile, allora si ha
$$\iint_{\bm\Phi(K)} f(x,y)\dd x \dd y=\iint_K (f\circ \bm\Phi)(\rho,\theta)\rho\dd \rho \dd \theta$$

\paragraph{Coordinate sferiche in $\R^3$}
La trasformazione in coordinate sferiche nello spazio $\bm\Phi:[0,+\infty)\times[0,\pi]\times[0,2\pi] \to \R^3$ è definita come segue:
$$
\bm\Phi(\rho,\theta,\varphi)=
\begin{cases}
    x=\rho\sin\theta\cos\varphi\\
    y=\rho\sin\theta\sin\varphi\\
    z=\rho\cos\theta
\end{cases}
$$

Il determinante della matrice jacobiana di $\bm\Phi(\rho,\theta,\varphi)$ è $\rho^2\sin\theta$. Il determinante si annulla quindi nel rettangolo $Z=\{0\}\times[0,\pi]\times[0,2\pi]$ e sull'insieme $[0,+\infty)\times\{\frac{\pi}{2},\frac{3}{2}\pi\}\times[0,2\pi]$, che hanno misura nulla e sono trasformati in insiemi di misura nulla. Inoltre $\bm\Phi(\rho,\theta,\varphi)$ non è iniettiva sul rettangolo $Z$ e sull'insieme $[0,+\infty)\times[0,\pi]\times\{0,2\pi\}$, tuttavia entrambi hanno misura nulla nello spazio $(\rho,\theta,\varphi)$ e sono trasformati in insiemi di misura nulla in $\R^3$, rispettivamente l'origine e il semipiano positivo del piano $xz$.

$\bm\Phi$ è un cambiamento di variabile ammissibile per il teorema \ref{thm:int_var}. Sia $f \in \C{0}(\bm\Phi(K),\R^2)$ con $K\subseteq[0,+\infty)\times[0,\pi]\times[0,2\pi]$ compatto e misurabile, allora si ha
$$\iiint_{\bm\Phi(K)}f(x,y,z)\dd x \dd y \dd z=\iiint_K(f\circ\bm\Phi)(\rho,\theta,\varphi)\rho^2\sin\theta  \dd \rho \dd \theta \dd \varphi$$

\paragraph{Coordinate cilindriche in $\R^3$}
% TODO