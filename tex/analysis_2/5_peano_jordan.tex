\chapter{Teoria della misura e dell'integrazione}

\section{Misura di Peano-Jordan}

\begin{definition}
    [Intervallo semi-aperto superiormente]
    Si definisce intervallo semi-aperto superiormente di $\R^n$ il prodotto cartesiano di $n$ intervalli chiusi inferiormente e aperti superiormente:
    \begin{equation*}
        I=[a_1,b_1)\times\cdots\times[a_n,b_n) \with a_i \leq b_i \ \forall i \in [n]
    \end{equation*}
    La misura elementare di $I$ è
    \begin{equation*}
        \mu_n(I)=\prod_{j=1}^n(b_j-a_j)
    \end{equation*}
\end{definition}

\begin{definition}
    [Pluri-intervallo]
    Siano $I_1,\dots,I_k$ intervalli semi-aperti superiormente a due a due disgiunti. Si definisce pluri-intervallo l'unione $\displaystyle P=\bigcup_{j=1}^kI_j$. La misura di $P$ è $\mu_n(P)=\displaystyle \sum_{j=1}^k\mu_n(I_j)$. 
    L'insieme dei pluri-intervalli di $\R^n$ viene indicato con $\mathcal{P}$.
\end{definition}

\begin{lemma}
    Valgono le seguenti proposizioni:
    \begin{itemize}
        \item Se $P_1,\dots,P_k \in \mathcal{P}$, allora $\displaystyle \bigcup_{j=1}^k P_j, \bigcap_{j=1}^kP_j \in \mathcal{P}$
        \item Se $P_1,P_2 \in \mathcal{P}$, allora $P_2 \setminus P_1 \in \mathcal{P}$
        \qed
    \end{itemize}
\end{lemma}

\begin{theorem}
    [Caratterizzazione della misura in $\mathcal{P}$]
    Per la misura in $\mathcal{P}$ valgono le seguenti proprietà:
    \begin{enumerate}
        \item (Additività finita) Se $P_1,\dots,P_k \in \mathcal{P}$ sono a due a due disgiunti, allora
        \begin{equation*}
            \mu_n\left( \bigcup_{j=1}^k P_j \right) = \sum_{j=1}^k \mu_n(P_j)
        \end{equation*}
        \item (Monotonia) Se $P, Q \in \mathcal{P} \e P \subseteq Q \then \mu_n(P)\leq \mu_n(Q)$
        \item (Modularità) Se $P,Q \in \mathcal{P}$, allora
        \begin{equation*}
            \mu_n(P\cup Q) + \mu_n (P \cap Q) = \mu_n (P) + \mu_n (Q)
        \end{equation*}
        \qed
    \end{enumerate}
\end{theorem}

\begin{corollary}
    Se $P_1,\dots,P_k \in \mathcal{P}$, allora $\displaystyle \mu_n\left(\bigcup_{j=1}^kP_j\right) \leq \sum_{j=1}^k\mu_n(P_j)$. Ovvero, vale la proprietà di subadditività finita per $\mu_n$ su $\mathcal{P}$.
    \qed
\end{corollary}

\begin{corollary}
    Se $P,Q \in \mathcal{P} \then \mu_n(P\setminus Q)=\mu_n(P)-\mu_n(P\cap Q)$. In particolare, se $Q \subseteq P$, allora $\mu_n(P\setminus Q) = \mu_n (P)-\mu_n(Q)$.
    \qed
\end{corollary}

\begin{lemma}
    Se $P \in \mathcal{P}$, allora $\partial P \in \mathcal{P} \e \mu_n(\partial P)=0$.
    \qed
\end{lemma}

\begin{corollary}
    Se $P \in \mathcal{P}$, allora $\mathring P, \overline{P}$ sono misurabili e $\mu_n(\mathring P)=\mu_n(\overline{P})=\mu_n(P)$.
    \qed
\end{corollary}

\begin{definition}
    [Misura interna ed esterna]
    Si definiscono gli insiemi dei pluri-intervalli interni all'insieme $X$, $\mathcal{P}^{(i)}$, e dei pluri-intervalli esterni all'insieme $X$, $\mathcal{P}^{(e)}$:
    \begin{align*}
        \mathcal{P}^{(i)}&=\{P\in \mathcal{P}:P\subseteq X\}\\
        \mathcal{P}^{(e)}&=\{Q \in \mathcal{P}: Q \supseteq X \}
    \end{align*}
    La misura interna di $X$ è $
        \mu_n^{(i)}(X)=\sup\limits_{P\in \mathcal{P}^{(i)}}\mu_n(P)
    $
    e la misura esterna di $X$ è $
        \mu_n^{(e)}(X)=\inf\limits_{Q \in \mathcal{P}^{(e)}}\mu_n(Q)
    $
\end{definition}

\begin{lemma}
    $\mu_n^{(i)}(X), \mu_n^{(e)}(X) \in \R_{\geq 0} \e \mu_n^{(i)}(X)\leq\mu_n^{(e)}(X)$.
    \qed
\end{lemma}

\begin{definition}
    [Insieme limitato misurabile secondo Peano-Jordan]
    Sia $X \subseteq \R^n$ un insieme limitato. Si dice che $X$ è misurabile secondo Peano-Jordan se $\mu_n^{(i)}(X)=\mu_n^{(e)}(X)$ e in tal caso si dice misura $n$-dimensionale di $X$ tale valore.
    L'insieme degli insiemi limitati misurabili secondo Peano-Jordan si indica con $\mathcal{J}_b(\R^n)$.
\end{definition}

\begin{lemma} Valgono le seguenti proposizioni:
    \begin{enumerate}
        \item $\mathcal{P} \subseteq \mathcal{J}_b(\R^n)$. Tutti i pluri-intervalli sono misurabili secondo Peano-Jordan e la misura coincide con quella definita in modo elementare.
        \item Se $X_1,\dots,X_k \in \mathcal{J}_b(\R^n)$, allora $\displaystyle\bigcup_{j=1}^k X_j, \bigcap_{j=1}^kX_j \in \mathcal{J}_b(\R^n)$
        \item Se $X,Y \in \mathcal{J}_b(\R^n)\then X \setminus Y \in \mathcal{J}_b(\R^n)$
        \qed
    \end{enumerate}
\end{lemma}

\begin{theorem}
    [Caratterizzazione della misurabilità]
    Sia $X\subseteq \R^n$ limitato. Allora:
    \begin{enumerate}
        \item $X\in \mathcal{J}_b(\R^n) \iff \forall \varepsilon > 0 \ \exists P,Q \in \mathcal{P} \tc P \subseteq X \subseteq Q \e \mu_n(Q)-\mu_n(P) < \varepsilon$
        \item $X \in \mathcal{J}_b(\R^n) \iff \forall \varepsilon > 0 \ \exists Y,Z \in \mathcal{J}_b(\R^n) \tc Y \subseteq X \subseteq Z \e \mu_n(Z)-\mu_n(Y) < \varepsilon$
        \item $X \in \mathcal{J}_b(\R^n)\iff \partial X \in \mathcal{J}_b(\R^n) \e \mu_n(\partial X)=0$
        \qed
    \end{enumerate}
\end{theorem}

\begin{corollary}
    Se $X \in \mathcal{J}_b(\R^n)$, allora $\overline{X},\mathring X \in \mathcal{J}_b(\R^n) \e \mu_n(X)=\mu_n(\overline{X})=\mu_n(\mathring X)$.
    \qed
\end{corollary}

\begin{theorem}
    [Proprietà di $\mathcal{J}_b(\R^n)$]
    Valgono le seguenti proprietà per la misura in $\mathcal{J}_b(\R^n)$:
    \begin{enumerate}
        \item (Additività finita) Se $X_1,\dots,X_k \in \mathcal{J}_b(\R)^n$ sono a due a due disgiunti, allora
        \begin{equation*}
            \mu_n\left(\displaystyle \bigcup_{j=1}^kX_j\right) =   \displaystyle\sum_{j=1}^k\mu_n(X_j)
        \end{equation*}
        \item (Modularità) Se $X,Y \in \mathcal{J}_b(\R^n)$, allora $\mu_n(X\cup Y)+\mu_n(X\cap Y)=\mu_n(X)+\mu_n(Y)$
        \item (Monotonia) Se $X,Y \in \mathcal{J}_b(\R^n) \e X \subseteq Y$, allora $\mu_n(X) \leq \mu_n(Y)$
        \item Se $X,Y \in \mathcal{J}_b(\R^n)$, allora $\mu_n(X\setminus Y)=\mu_n(X)-\mu_n(Y)$. In particolare, se $Y \subseteq X \then \mu_n (X \setminus Y)=\mu_n(X)-\mu_n(Y)$
        \qed
    \end{enumerate}
\end{theorem}

\begin{theorem}
    [Caratterizzazione degli insiemi di misura nulla]\leavevmode
    \begin{enumerate}
        \item $\mathring X = \varnothing \iff \mu_n^{(i)}=0$
        \item $X \in \mathcal{J}_b(\R^n) \e \mu_n(X)=0 \iff \forall \varepsilon >0 \ \exists P \in \mathcal{P} \tc P \supseteq X \e \mu_n(P)< \varepsilon$
        \item $X \in \mathcal{J}_b(\R^n) \e \mu_n(X)=0 \iff \mathring = \varnothing \e X \in \mathcal{J}_b(\R^n)$
        \qed
    \end{enumerate}
\end{theorem}

\begin{definition}
    [Insieme misurabile secondo Peano-Jordan]
    Sia $X\subseteq \R^n$. $X$ è misurabile secondo Peano-Jordan, ovvero $X \in \mathcal{J}(\R^n)$, se $\forall Y \in \mathcal{J}_b(\R^n), \ Y \cap X \in \mathcal{J}_b(\R^n)$. In tal caso si definisce $\mu_n(X)=\sup\{\mu_n(X\cap Y):Y\in \mathcal{J}_b(\R^n)\}$.
\end{definition}

\begin{remark}
    Questa definizione allarga la nozione di misura agli insiemi non limitati. Un insieme misurabile in $\mathcal{J}_b(\R^n)$ ha la stessa misura se misurato in $\mathcal{J}(\R^n)$.
\end{remark}

\begin{theorem}
    Se $X \in \mathcal{J}(\R^n) \e \{X_k\}_{k \in \N}$ è una successione crescente di elementi in $\mathcal{J}_b(\R^n)$, ovvero:
    \begin{enumerate}
        \item $\forall k \in \N, \ X_k \in \mathcal{J}_b(\R^n)$
        \item $X_k \subseteq X_{k+1}$
        \item $\bigcup_{k=1}^{+\infty}X_k = \R^n$,
    \end{enumerate}
    allora $\mu_n(X)=\displaystyle\lim_{k\to + \infty}\mu_n(X\cap X_k)$.
    \qed
\end{theorem}

\begin{theorem}
    $\mu_n: \mathcal{J}(\R^n) \to [0, +\infty]$ e inoltre:
    \begin{enumerate}
        \item $\varnothing \in \mathcal{J}(\R^n)$
        \item Se $X \in \mathcal{J}(\R^n)$, allora $\R^n \setminus X \in \mathcal{J}(\R^n)$
        \item Se $X, Y \in \mathcal{J}(\R^n)$, allora $X \cup Y, X \cap Y, X \setminus Y \in \mathcal{J}(\R^n)$
        \qed
    \end{enumerate}
\end{theorem}

\begin{theorem}
    [Proprietà di $\mathcal{J}(\R^n)$]
    $\mathcal{J}(\R^n)$ gode delle seguenti proprietà:
    \begin{enumerate}
        \item (Additività finita) Se $X_1,\dots,X_k \in \mathcal{J}(\R^n), \ X_i \cap X_j = \varnothing \ \forall i, j \in [k]$, allora $\bigcup_{j=1}^kX_j \in \mathcal{J}(\R^n)$ e
        \begin{equation*}
            \mu_n\left(\bigcup_{j=1}^kX_j\right) = \sum_{j=1}^k\mu_n(X_j)
        \end{equation*}
        \item (Monotonia) Se $X, Y \in \mathcal{J}(\R^n) \e X \subseteq Y$, allora $\mu_n(X) \leq \mu_n(Y)$
        \item (Modularità) Se $X, Y \in \mathcal{J}(\R^n)$, allora
        \begin{equation*}
            \mu_n(X \cup Y) + \mu_n (X \cap Y) = \mu_n(X) + \mu_n(Y)
        \end{equation*}
        \item (Sottrattività) Se $X, Y \in \mathcal{J}(\R^n) \e \mu_n(Y) < + \infty$, allora
        \begin{equation*}
            \mu_n(X \setminus Y) = \mu_n(X) - \mu_n(X \cap Y)
        \end{equation*}
        \qed
    \end{enumerate}
\end{theorem}

\begin{lemma}
    $X \in \mathcal{J}(\R^n) \iff \partial X \in \mathcal{J}(\R^n) \e \mu_n(\partial X)=0$.
    \qed
\end{lemma}

\section{Integrazione secondo Riemann}

\begin{definition}
    [Funzione non negativa integrabile secondo Riemann]
    Siano $A \subseteq \R^n, \ A \in \mathcal{J}(\R^n), \ f: A \to \R$ limitata e tale che $f \geq 0$. Si definisce sottografico di $f$ l'insieme $R(f)=\{(\vb{x}, y)\in A \times \R_{\geq 0} : 0 \leq y \leq f(\vb{x})\}$.
    $f$ si dice integrabile secondo Riemann se $R(f) \in \mathcal{J}(\R^{n+1})$ e in tal caso
    \begin{equation*}
        \idotsint_A f(x_1,\dots,x_n)dx_1\cdots dx_n=\mu_{n+1}(R(f))
    \end{equation*}
\end{definition}

\begin{definition}
    [Parte positiva e parte negativa]
    Se $f:A \subseteq \R^n \to \R$, si definiscono:
    \begin{itemize}
        \item Parte positiva di $f$, $f_+(x)=\max\{0,f(x)\}$
        \item Parte negativa di $f$, $f_-(x)=\max\{0,-f(x)\}$
    \end{itemize}
    In base a questa definizione, $f(x)=f_+(x)-f_-(x)$ e $\abs{f(x)}=f_+(x)+f_-(x)$.
\end{definition}

\begin{definition}
    [Funzione integrabile secondo Riemann]
\end{definition}